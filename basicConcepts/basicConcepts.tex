\documentclass{ximera}

%% You can put user macros here
%% However, you cannot make new environments

\listfiles

\graphicspath{{./}{firstExample/}{secondExample/}}

\usepackage{tikz}
\usepackage{tkz-euclide}
\usepackage{tikz-3dplot}
\usepackage{tikz-cd}
\usetikzlibrary{shapes.geometric}
\usetikzlibrary{arrows}
\usetikzlibrary{decorations.pathmorphing,patterns}
\usetkzobj{all}
\pgfplotsset{compat=1.13} % prevents compile error.

\renewcommand{\vec}[1]{\mathbf{#1}}
\newcommand{\RR}{\mathbb{R}}
\newcommand{\dfn}{\textit}
\newcommand{\dotp}{\cdot}
\newcommand{\id}{\text{id}}
\newcommand\norm[1]{\left\lVert#1\right\rVert}
 
\newtheorem{general}{Generalization}
\newtheorem{initprob}{Exploration Problem}

\tikzstyle geometryDiagrams=[ultra thick,color=blue!50!black]

\usepackage{mathtools}

\title{Basic Concepts}

\begin{document}%\label{Module 0}

\begin{abstract}
We define ordinary differential equations and what it means for a function to be a solution to such an equation.
\end{abstract}

\maketitle

\section*{Basic Concepts}
\subsection*{What is a Differential Equation?}
A \dfn{differential equation} is an equation that contains one or more
derivatives of an unknown function. The \dfn{order} of a
differential equation is the order of the highest
derivative that it contains. A differential equation is an
\dfn{ordinary differential equation} if it involves an unknown
function of only one variable, or a \dfn{partial differential
equation} if it involves partial derivatives of a function of more
than one variable. For now we'll consider only ordinary differential
equations, and we'll just call them \dfn{differential equations}.

Throughout this text, all variables and constants are real numbers unless it's
stated otherwise. We'll usually use $x$ for the independent variable
unless the independent variable is time; then we'll use $t$.

The simplest differential equations are first order equations of the
form
$$
\frac{dy}{dx}=f(x) \quad \text{or, equivalently,} \quad y'=f(x),
$$
where $f$  is a known function of $x$. We already know from calculus
how to find functions that satisfy this kind of equation. For example,
if
$$
y'=x^3,
$$
then
$$
y=\int x^3\, dx=\frac{x^4}{4}+c,
$$
 where $c$ is an arbitrary constant.  If $n>1$
we can find functions $y$ that satisfy
equations of the form
\begin{equation} \label{eq:1.2.1}
y^{(n)}=f(x)
\end{equation}
by repeated integration. Again, this is a calculus problem.

Except for illustrative purposes in this section, there's no need to
consider differential equations like \eqref{eq:1.2.1}.  We'll
usually consider differential equations that can be written as
\begin{equation} \label{eq:1.2.2}
y^{(n)}=f(x,y,y', \dots,y^{(n-1)}),
\end{equation}
where at least one of the functions $y$, $y'$, \dots, $y^{(n-1)}$ actually
appears on the right. Here are some examples:
$$
\begin{array}{rcll}
\frac{dy}{dx}-x^2&=&0&\mbox{ (first order)}  \\
\frac{dy}{dx}+2xy^2&=&-2&\mbox{ (first order)}   \\
\frac{d^2y}{dx^2}+2\frac{dy}{dx}+y&=&2x&\mbox{ (second order)}
\\
xy'''+y^2&=&\sin x  &\mbox{ (third order)}
\\
y^{(n)}+xy'+3y&=&x&\mbox{ ($n$-th order)}
\end{array}
$$
Although none of these equations is  written as in
\eqref{eq:1.2.2}, all of them can be written in this form:
$$
\begin{array}{rcl}
y'&=&x^2  \\
y'&=&-2-2xy^2   \\
y''&=&2x-2y'-y  \\
y'''&=&\frac{\sin x-y^2}{x}
\\ y^{(n)}&=&x-xy'-3y
\end{array}
$$

\subsection*{Solutions of Differential Equations}

A \dfn{solution} of a differential equation is a function that
satisfies the differential equation on some open interval;   thus, $y$
is a solution of \eqref{eq:1.2.2} if $y$ is $n$ times differentiable and
$$
y^{(n)}(x)=f(x,y(x),y'(x), \dots,y^{(n-1)}(x))
$$
for all $x$ in some open interval $(a,b)$. In this case, we also say
that $y$ \dfn{is a solution of} $\eqref{eq:1.2.2}$ \dfn{on} $(a,b)$. Functions
that satisfy a differential equation at isolated points are not
interesting. For example, $y=x^2$ satisfies
$$
xy'+x^2=3x
$$
if and only if $x=0$ or $x=1$, but it's not a solution of this
differential equation because it does not satisfy the equation on an
open interval.

The graph of a solution of a differential equation is  a \dfn{
solution curve}. More generally, a curve $C$ is said to be an \dfn{
integral curve} of a differential equation if every function
$y=y(x)$ whose graph is a segment of $C$ is a solution of the
differential equation. Thus, any solution curve of a differential
equation is an integral curve, but an integral curve need not be a
solution curve.

\begin{example}
\label{example:1.2.1}
If $a$ is any positive constant,  the circle
\begin{equation} \label{eq:1.2.3}
x^2+y^2=a^2
\end{equation}
is an integral curve of
\begin{equation} \label{eq:1.2.4}
y'=-\frac{x}{y}.
\end{equation}
To see this, note that the only functions whose graphs are segments of
\eqref{eq:1.2.3} are
$$
y_1=\sqrt{a^2-x^2}\quad\text{and}\quad y_2=-\sqrt{a^2-x^2}.
$$
We leave it to you to verify that these functions both satisfy
\eqref{eq:1.2.4} on the open interval $(-a,a)$. However, the graph of \eqref{eq:1.2.3} is
not a solution
curve of \eqref{eq:1.2.4}, since it's not the graph of a function.
\end{example}

\begin{example}\label{example:1.2.2}
Verify that
\begin{equation} \label{eq:1.2.5}
y=\frac{x^2}{3}+\frac{1}{x}
\end{equation}
is a solution of
\begin{equation} \label{eq:1.2.6}
xy'+y=x^2
\end{equation}
 on $(0,\infty)$ and on $(-\infty,0)$.
 
\begin{explanation}
Substituting \eqref{eq:1.2.5} and
$$
y'=\frac{2x}{3} - \frac{1}{x^2}
$$
into \eqref{eq:1.2.6}  yields
$$
xy'(x)+y(x)=x \left(\frac{2x}{3} - \frac{1}{x^2}\right)+
\left(\frac{x^2}{3}+\frac{1}{x}\right)=x^2
$$
for all $x\neq 0$. Therefore $y$ is a solution of \eqref{eq:1.2.6}
on $(-\infty,0)$ and $(0,\infty)$.
 However, $y$ isn't  a solution of the differential
equation on any open interval that contains $x=0$, since  $y$ is
not defined at  $x=0$.

The graph of \eqref{eq:1.2.5} appears below

\desmos{zbsiqyq1os}{800}{600}

The part
of the graph of \eqref{eq:1.2.5} on $(0,\infty)$ is a solution curve of
\eqref{eq:1.2.6}, as is the part of the graph on $(-\infty,0)$.
\end{explanation}
\end{example}

\begin{example}\label{example:1.2.3}
 Show that if $c_1$ and $c_2$ are constants then
\begin{equation} \label{eq:1.2.7}
y=(c_1+c_2x)e^{-x}+2x-4
\end{equation}
 is a solution of
\begin{equation} \label{eq:1.2.8}
y''+2y'+y=2x
\end{equation}
 on $(-\infty,\infty)$.

\begin{explanation} 
Differentiating \eqref{eq:1.2.7} twice yields
$$
y'=-(c_1+c_2x)e^{-x}+c_2e^{-x}+2
$$
 and
$$
y''=(c_1+c_2x)e^{-x}-2c_2e^{-x},
$$
so
\begin{eqnarray*}
y''+2y'+y&=&(c_1+c_2x)e^{-x}-2c_2e^{-x}\\
&&+2\left[-(c_1+c_2x)e^{-x}+c_2e^{-x}+2\right]\\
&&+(c_1+c_2x)e^{-x}+2x-4\\
&=&(1-2+1)(c_1+c_2x)e^{-x}+(-2+2)c_2e^{-x}\\ &&+4+2x-4=2x
\end{eqnarray*}
for all values of $x$.
Therefore $y$ is a solution of \eqref{eq:1.2.8} on $(-\infty,\infty)$.
\end{explanation}
\end{example}

\begin{example}\label{example:1.2.4}
Find all solutions of
\begin{equation} \label{eq:1.2.9}
y^{(n)}=e^{2x}.
\end{equation}

\begin{explanation} Integrating \eqref{eq:1.2.9} yields
$$
y^{(n-1)}=\frac{e^{2x}}{2}+k_1,
$$
 where $k_1$ is a constant. If $n\geq 2$,
integrating again yields
$$
y^{(n-2)}=\frac{e^{2x}}{4}+k_1x+k_2.
$$
If $n\geq 3$, repeatedly integrating yields
\begin{equation} \label{eq:1.2.10}
y=\frac{e^{2x}}{2^n}+k_1\frac{x^{n-1}}{(n-1)!}+k_2\frac{x^{n-2}}{(n-2)!}+\cdots+k_n,
\end{equation}
 where $k_1, k_2, \dots, k_n$ are constants.
This shows that every solution of \eqref{eq:1.2.9} has the form
\eqref{eq:1.2.10}
for some choice of the constants $k_1, k_2, \dots, k_n$.
On the other hand, differentiating \eqref{eq:1.2.10}  $n$ times shows
that if
$k_1, k_2, \dots, k_n$ are arbitrary constants, then the function $y$
in
\eqref{eq:1.2.10} satisfies \eqref{eq:1.2.9}.
\end{explanation}
\end{example}

Since the constants $k_1, k_2, \dots, k_n$  in \eqref{eq:1.2.10}
are arbitrary, so are the constants
$$\frac{k_1}{(n-1)!},\, \frac{k_2}{(n-2)!},\, \cdots, \, k_n.$$
 Therefore Example~\ref{example:1.2.4} actually shows that all
solutions of \eqref{eq:1.2.9}  can be written as
$$
y=\frac{e^{2x}}{2^n}+c_1+c_2x+\cdots+c_nx^{n-1},
$$
 where we  renamed the arbitrary constants in
\eqref{eq:1.2.10} to obtain a simpler formula. As a
general rule, arbitrary constants appearing in solutions  of differential
equations should be simplified if possible. You'll see examples
of this throughout the text.

\subsection*{Initial Value Problems}

In Example~\ref{example:1.2.4} we saw that the differential equation
$y^{(n)}=e^{2x}$ has an infinite \dfn{family of solutions} that depend upon
the $n$ arbitrary constants $c_1, c_2, \dots, c_n$. In the absence of
additional conditions, there's no reason to prefer one solution of a
differential equation over another. However, we'll often be
interested in finding a solution of a differential equation that
satisfies one or more specific conditions. The next example
illustrates this.

\begin{example}\label{example:1.2.5}
 Find a solution of
$$
y'=x^3
$$
 such that $y(1)=2$.

\begin{explanation}  At the beginning of this section we saw
that the  solutions of $y'=x^3$ are
$$
y=\frac{x^4}{4}+c.
$$
 To determine a value of $c$ such that $y(1)=2$,
we set $x=1$ and $y=2$ here to obtain
$$
2=y(1)=\frac{1}{4}+c,\quad\text{so}\quad c=\frac{\answer{7}}{\answer{4}}.
$$
 Therefore the required solution is
$$
y=\frac{x^4+\answer{7}}{\answer{4}}.
$$
The Desmos interactive below allows you to see several members of the family of solutions.  Note that imposing the  condition $y(1)=2$ is equivalent to requiring
 the graph of $y$ to pass through the point $(1,2)$.  Use the slider to find the value of $c$ that makes the solution satisfy $y(1)=2$.  Verify that the value that you have found graphically agrees with the computed value.
 
\desmos{zwrgfciyd3}{800}{600} 

\end{explanation}
\end{example}

 We can rewrite the problem considered in Example~\ref{example:1.2.5}
more briefly as
$$
y'=x^3,\quad y(1)=2.
$$

 We call this an \dfn{initial value problem}.
The requirement $y(1)=2$ is an \dfn{initial condition}.
 Initial value problems can also be
posed for higher order differential equations.  For example,
\begin{equation} \label{eq:1.2.11}
y'' - 2y'+3y=e^x, \quad y(0)=1, \quad y'(0)=2
\end{equation}
is  an initial value problem for a second order differential
equation where $y$ and $y'$ are required to have specified values at
 $x=0$. In general, an initial value
problem for an $n$-th order differential equation requires $y$ and its
first $n-1$ derivatives to have specified values at some point $x_0$.
These requirements are the \dfn{initial conditions}.

We'll denote an initial value problem for a differential equation by
writing the initial conditions after the equation, as in
\eqref{eq:1.2.11}. For example, we would write an initial value problem
for \eqref{eq:1.2.2} as
\begin{equation} \label{eq:1.2.12}
y^{(n)}=f(x,y,y', \dots,y^{(n-1)}),\, y(x_0)=k_0,\,
y'(x_0)=k_1,\, \dots,\, y^{(n-1)}=k_{n-1}.
\end{equation}
Consistent with our earlier definition of a solution of the
differential equation in \eqref{eq:1.2.12}, we say that $y$ is a \dfn{solution
of the initial value problem} \eqref{eq:1.2.12}  if $y$ is $n$ times
differentiable and
$$
y^{(n)}(x)=f(x,y(x),y'(x), \dots,y^{(n-1)}(x))
$$
for all $x$ in some open interval $(a,b)$  that contains $x_0$,
and  $y$ satisfies the initial conditions in \eqref{eq:1.2.12}. The
largest open interval that contains $x_0$ on which $y$ is defined and
satisfies the differential equation is  the \dfn{interval of
validity} of $y$.

\begin{example}\label{example:1.2.6}
In Example~\ref{example:1.2.5} we saw that
\begin{equation} \label{eq:1.2.13}
y=\frac{x^4+7}{4}
\end{equation}
is a solution of the initial value problem
$$
y'=x^3,\quad y(1)=2.
$$
Since the function in \eqref{eq:1.2.13} is defined for all $x$, the
interval of validity of this solution is $(-\infty,\infty)$.
\end{example}

\begin{example}\label{example:1.2.7}
In Example~\ref{example:1.2.2} we verified that
\begin{equation} \label{eq:1.2.14}
y=\frac{x^2}{3}+\frac{1}{x}
\end{equation}
is a solution of
$$
xy'+y=x^2
$$
on $(0,\infty)$ and on $(-\infty,0)$. By evaluating \eqref{eq:1.2.14} at
$x=\pm 1$, you can see that \eqref{eq:1.2.14} is a solution of the initial
value problems
\begin{equation} \label{eq:1.2.15}
xy'+y=x^2,\quad y(1)=\frac{4}{3}
\end{equation}
and
\begin{equation} \label{eq:1.2.16}
xy'+y=x^2,\quad y(-1)=-\frac{2}{3}.
\end{equation}
The interval of validity of \eqref{eq:1.2.14} as a solution of
\eqref{eq:1.2.15} is $(0,\infty)$, since this is the largest interval
that contains $x_0=1$ on which \eqref{eq:1.2.14} is defined. 

\desmos{xzhmpxofig}{800}{600} 

Similarly, the
interval of validity of \eqref{eq:1.2.14} as a solution of \eqref{eq:1.2.16}
is $(-\infty,0)$, since this is the largest interval that contains
$x_0=-1$ on which \eqref{eq:1.2.14} is defined.

\desmos{citkz0ppdp}{800}{600} 

\end{example}

\subsection*{Free Fall Under Constant Gravity}

The term \dfn{initial value problem}  originated in problems
of motion where the independent variable is $t$
(representing elapsed time), and the initial conditions
are the position and velocity of an object at the initial
(starting) time of an experiment.

\begin{example}\label{example:1.2.8}
An object falls under the influence of gravity near Earth's
surface, where it can be assumed that the magnitude of the
acceleration due to gravity is a constant $g$.
\begin{enumerate}
\item \label{part:ex1.2.8partA}
 Construct a mathematical model for the motion of the object in the
form of an initial value problem for a second order differential
equation, assuming that the altitude and velocity of the object at
time $t=0$ are known. Assume that gravity is the only force acting on
the object.
\item \label{part:ex1.2.8partB}
 Solve the initial value problem derived in Part \ref{part:ex1.2.8partA} to obtain the
altitude as a function of time.
\end{enumerate}


\begin{explanation} Part \ref{part:ex1.2.8partA}. Let $y(t)$ be the altitude of the object at time $t$.
 Since the acceleration of the object has
constant magnitude $g$ and is in the downward (negative) direction,
$y$ satisfies the second order equation
$$
y''=-g,
$$
where the prime now indicates differentiation with respect to $t$.
If $y_0$ and $v_0$ denote the altitude and velocity when
$t=0$, then $y$ is a solution of the initial value problem
\begin{equation} \label{eq:1.2.17}
y''=-g,\quad  y(0)=y_0,\quad y'(0)=v_0.
\end{equation}

Part \ref{part:ex1.2.8partB}.
Integrating \eqref{eq:1.2.17} twice yields
\begin{eqnarray*}
y'&=&-gt+c_1, \\
y&=&-\frac{gt^2}{2}+c_1t+c_2.
\end{eqnarray*}
Imposing the initial conditions $y(0)=y_0$ and $y'(0)=v_0$ in these
two equations shows that $c_1=v_0$ and $c_2=y_0$. Therefore the
solution of the initial value problem \eqref{eq:1.2.17} is
$$
y=- \frac{gt^2}{2}+v_0t+y_0.
$$
\end{explanation}
\end{example}


\section*{Text Source}
Trench, William F., "Elementary Differential Equations" (2013). Faculty Authored and Edited Books \& CDs. 8. (CC-BY-NC-SA)

\href{https://digitalcommons.trinity.edu/mono/8/}{https://digitalcommons.trinity.edu/mono/8/}
\end{document}