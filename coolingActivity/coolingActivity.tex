\documentclass{ximera}

%% You can put user macros here
%% However, you cannot make new environments

\listfiles

\graphicspath{{./}{firstExample/}{secondExample/}}

\usepackage{tikz}
\usepackage{tkz-euclide}
\usepackage{tikz-3dplot}
\usepackage{tikz-cd}
\usetikzlibrary{shapes.geometric}
\usetikzlibrary{arrows}
\usetikzlibrary{decorations.pathmorphing,patterns}
\usetkzobj{all}
\pgfplotsset{compat=1.13} % prevents compile error.

\renewcommand{\vec}[1]{\mathbf{#1}}
\newcommand{\RR}{\mathbb{R}}
\newcommand{\dfn}{\textit}
\newcommand{\dotp}{\cdot}
\newcommand{\id}{\text{id}}
\newcommand\norm[1]{\left\lVert#1\right\rVert}
 
\newtheorem{general}{Generalization}
\newtheorem{initprob}{Exploration Problem}

\tikzstyle geometryDiagrams=[ultra thick,color=blue!50!black]

\usepackage{mathtools}

\title{Cooling Activity}

\begin{document}

\begin{abstract}
An experiment related to Newton's cooling law.
\end{abstract}

\maketitle

The purpose of this exercise is to deepen your understanding of Newton?s Law of Heating (and Cooling) which is reviewed in Trench?s text in Section 4.2.  A hands-on activity will help to supplement and apply the background theory.  Students will record and predict the temperature of a potato baking in the oven and then subsequently left out to cool.  This experimental activity requires the use of an oven thermometer, preferable a digital one that has a wire to allow readings with a closed oven door.  If you have or can borrow a thermometer, you can conduct this experiment in your own kitchen.  Particularly hungry students will opt to bake a few potatoes and also be ready with some sour cream or blue cheese dressing.  Students without immediate access to an oven may try the same experiment with the potato submerged in boiling water.
Overview
In Section 4.2 of Trench?s text, Newton?s Law of Cooling is summarized a first order differential equation:
\[
T'==k(T-T_m)
\]

This equation stipulates that $T'$, the time rate of change of the temperature of some object, is linearly proportional to the difference between the temperature  $T$ of that object and the temperature of the medium or surroundings of the object $T_m$.  Trench notes the ideal case where  $T_m$ is perfectly constant, such as when an object is in a room or a hot oven that remains at a constant temperature.  A negative sign is placed before the ?temperature decay constant? of the medium $k$, which is itself some positive number that will stipulate the relative rate of the object?s temperature change.  Observe that this differential equation has a very similar appearance to a problem like radioactive decay.  In fact, heating (or cooling) of an object is essentially an exponential decay problem.  What decays is the difference between the object and its surroundings.  This difference decays at a rate proportional to its magnitude.

\section*{Predicting the heating process}

Consider what Newton?s Law of Heating would predict for a potato heated in an oven.  Examine the differential equation, and remember that $k$ and $T_m$ are constants.  At what point during the heating process will the rate of change of temperature be the smallest or the largest?  How will this rate change over time?  Which of the following plots of temperature vs. time most closely represents the behavior predicted by this law?

% Figure 1 here

\emph{Feedback for incorrect answer}: This answer is not correct.  Note that the final rate of change of temperature is too steep and the initial rate of change is too small.  The temperature should asymptotically approach the temperature of the medium because Newton?s Law of Heating shows that $T'$ will get very small as $T$ approaches $T_m$.  This plot suggests that the temperature of the object might actually exceed the temperature of the surroundings, which is impossible.


\end{document}