\documentclass{ximera}

%% You can put user macros here
%% However, you cannot make new environments

\listfiles

\graphicspath{{./}{firstExample/}{secondExample/}}

\usepackage{tikz}
\usepackage{tkz-euclide}
\usepackage{tikz-3dplot}
\usepackage{tikz-cd}
\usetikzlibrary{shapes.geometric}
\usetikzlibrary{arrows}
\usetikzlibrary{decorations.pathmorphing,patterns}
\usetkzobj{all}
\pgfplotsset{compat=1.13} % prevents compile error.

\renewcommand{\vec}[1]{\mathbf{#1}}
\newcommand{\RR}{\mathbb{R}}
\newcommand{\dfn}{\textit}
\newcommand{\dotp}{\cdot}
\newcommand{\id}{\text{id}}
\newcommand\norm[1]{\left\lVert#1\right\rVert}
 
\newtheorem{general}{Generalization}
\newtheorem{initprob}{Exploration Problem}

\tikzstyle geometryDiagrams=[ultra thick,color=blue!50!black]

\usepackage{mathtools}

\title{The Method of Frobenius I}%\label{Module 7-ADEF}


\begin{document}

\begin{abstract}

\end{abstract}

\maketitle

\section*{The Method of Frobenius I}

noindent
In this section we begin to study  series solutions of a
homogeneous linear second order differential equation with a regular
singular point at $x_0=0$, so it can be written as
\begin{equation} \label{eq:7.5.1}
x^2A(x)y''+xB(x)y'+C(x)y=0,
\end{equation}
where $A$, $B$, $C$ are polynomials and $A(0)\neq 0$.

We'll see that \eqref{eq:7.5.1} always has at least one solution of
the form
$$
y=x^r\sum_{n=0}^\infty a_nx^n
$$
where $a_0\ne0$ and $r$ is a suitably chosen number.
The method  we
will use to find solutions of this form and other forms that we'll
encounter in the next two sections is called
\href{http://www-history.mcs.st-and.ac.uk/Mathematicians/Frobenius.html}{the method of Frobenius},
and we'll call them
\href{http://www-history.mcs.st-and.ac.uk/Mathematicians/Frobenius.html}{Frobenius solutions}.

It can be shown that the power series $\sum_{n=0}^\infty a_nx^n$ in a
Frobenius solution of \eqref{eq:7.5.1} converges on some open interval
$(-\rho,\rho)$, where $0<\rho\leq\infty$. However, since $x^r$ may be
complex for negative $x$ or undefined if $x=0$, we'll consider
solutions defined for positive values of $x$. Easy modifications of
our results yield solutions defined for negative values of $x$.
%(Exercise~\ref{exer:7.5.54}).

We'll restrict our attention to the case where $A$, $B$, and $C$ are
polynomials of degree not greater than two, so \eqref{eq:7.5.1} becomes
\begin{equation} \label{eq:7.5.2}
x^2(\alpha_0+\alpha_1x+\alpha_2x^2)y''+x(\beta_0+\beta_1x+\beta_2x^2)y'
+(\gamma_0+\gamma_1x+\gamma_2x^2)y=0,
\end{equation}
where $\alpha_i$, $\beta_i$, and $\gamma_i$ are real constants and
$\alpha_0\neq 0$. Most  equations that arise in
applications can be written this way. Some examples are
\begin{eqnarray*}
\alpha x^2y''+\beta xy'+\gamma y&=&0 \quad\mbox{(Euler's equation)},
\\
x^2y''+xy'+(x^2-\nu^2)y&=&0 \quad\mbox{(Bessel's equation)},\\
\mbox{and}\\
xy''+(1-x)y'+\lambda y&=&0,
\quad\mbox{\href{http://www-history.mcs.st-and.ac.uk/Mathematicians/Laguerre.html}{(Laguerre's equation)}},
\end{eqnarray*}
where we would multiply the last equation through by $x$ to put it in
the form \eqref{eq:7.5.2}. However, the method of Frobenius can be
extended to the case where $A$, $B$, and $C$ are functions that can be
represented by power series in $x$ on some interval that contains zero,
and $A_0(0)\neq 0$.
%(Exercises~\ref{exer:7.5.57} and \ref{exer:7.5.58}).

The next two theorems will enable us to develop systematic methods for
finding Frobenius solutions of \eqref{eq:7.5.2}.

\begin{theorem}\label{thmtype:7.5.1}
Let
$$
Ly=
x^2(\alpha_0+\alpha_1x+\alpha_2x^2)y''+x(\beta_0+\beta_1x+\beta_2x^2)y'
+(\gamma_0+\gamma_1x+\gamma_2x^2)y,
$$
and define
\begin{eqnarray*}
p_0(r)&=&\alpha_0r(r-1)+\beta_0r+\gamma_0,\\
p_1(r)&=&\alpha_1r(r-1)+\beta_1r+\gamma_1,\\
p_2(r)&=&\alpha_2r(r-1)+\beta_2r+\gamma_2.\\
\end{eqnarray*}
Suppose the series
\begin{equation} \label{eq:7.5.3}
y=\sum_{n=0}^\infty a_nx^{n+r}
\end{equation}
converges on $(0,\rho)$.
 Then
\begin{equation} \label{eq:7.5.4}
Ly=\sum_{n=0}^\infty b_nx^{n+r}
\end{equation}
on $(0,\rho),$
where
\begin{equation} \label{eq:7.5.5}
\begin{array}{ccl}
b_0&=&p_0(r)a_0,\\
b_1&=&p_0(r+1)a_1+p_1(r)a_0,\\
b_n&=&p_0(n+r)a_n+p_1(n+r-1)a_{n-1}+p_2(n+r-2)a_{n-2},\quad n\geq2.
\end{array}
\end{equation}
\end{theorem}

\begin{proof}
We begin by showing that if $y$ is given by \eqref{eq:7.5.3} and
$\alpha$, $\beta$, and $\gamma$ are constants, then
\begin{equation} \label{eq:7.5.6}
\alpha x^2y''+\beta xy'+\gamma y=
\sum_{n=0}^\infty p(n+r)a_nx^{n+r},
\end{equation}
where
$$
p(r)=\alpha r(r-1)+\beta r +\gamma.
$$
Differentiating (3)  twice yields
\begin{equation} \label{eq:7.5.7}
y'=\sum_{n=0}^\infty (n+r)a_nx^{n+r-1}
\end{equation}
and
\begin{equation} \label{eq:7.5.8}
y''=\sum_{n=0}^\infty (n+r)(n+r-1)a_nx^{n+r-2}.
\end{equation}
Multiplying \eqref{eq:7.5.7} by $x$ and \eqref{eq:7.5.8} by $x^2$ yields
$$
xy'=\sum_{n=0}^\infty (n+r)a_nx^{n+r}
$$
and
$$
x^2y''=\sum_{n=0}^\infty (n+r)(n+r-1)a_nx^{n+r}.
$$
Therefore
\begin{eqnarray*}
\alpha x^2y''+\beta xy'+\gamma y
&=&\sum_{n=0}^\infty\left[\alpha(n+r)(n+r-1)+\beta(n+r)+\gamma\right]a_n
x^{n+r}\\
&=&\sum_{n=0}^\infty p(n+r)a_nx^{n+r},
\end{eqnarray*}
which proves  \eqref{eq:7.5.6}.

 Multiplying \eqref{eq:7.5.6} by $x$ yields
\begin{equation} \label{eq:7.5.9}
x(\alpha x^2y''+\beta xy'+\gamma
y)=\sum_{n=0}^\infty p(n+r) a_nx^{n+r+1}=
\sum_{n=1}^\infty p(n+r-1)a_{n-1}x^{n+r}.
\end{equation}
Multiplying \eqref{eq:7.5.6} by $x^2$ yields
\begin{equation} \label{eq:7.5.10}
x^2(\alpha x^2y''+\beta xy'+\gamma
y)=\sum_{n=0}^\infty p(n+r)a_nx^{n+r+2}=
\sum_{n=2}^\infty p(n+r-2)a_{n-2}x^{n+r}.
\end{equation}

To use these results, we rewrite
$$
Ly=
x^2(\alpha_0+\alpha_1x+\alpha_2x^2)y''+x(\beta_0+\beta_1x+\beta_2x^2)y'
+(\gamma_0+\gamma_1x+\gamma_2x^2)y
$$
 as
\begin{equation} \label{eq:7.5.11}
\begin{array}{ccl}
Ly&=&\left(\alpha_0x^2y''+\beta_0xy'
+\gamma_0y\right) +
x\left(\alpha_1x^2y''+\beta_1xy'+\gamma_1y\right)
\\&&+\
x^2\left(\alpha_2x^2y''+\beta_2xy'+\gamma_2y\right).
\end{array}
\end{equation}
From \eqref{eq:7.5.6} with $p=p_0$,
$$
\alpha_0x^2y''+\beta_0xy'+\gamma_0y=\sum_{n=0}^\infty
p_0(n+r)a_nx^{n+r}.
$$
From \eqref{eq:7.5.9} with $p=p_1$,
$$
x\left(\alpha_1x^2y''+\beta_1xy'+\gamma_1y\right)=\sum_{n=1}^\infty
p_1(n+r-1)a_{n-1}x^{n+r}.
$$
From \eqref{eq:7.5.10} with $p=p_2$,
$$
x^2\left(\alpha_2x^2y''+\beta_2xy'+\gamma_2y\right)=\sum_{n=2}^\infty
p_2(n+r-2)a_{n-2}x^{n+r}.
$$
Therefore we can rewrite \eqref{eq:7.5.11} as
\begin{eqnarray*}
Ly=\sum_{n=0}^\infty p_0(n+r)a_nx^{n+r}+
\sum_{n=1}^\infty p_1(n+r-1)a_{n-1}x^{n+r}\\+
\sum_{n=2}^\infty p_2(n+r-2)a_{n-2}x^{n+r},
\end{eqnarray*}
or
\begin{eqnarray*}
Ly&=& p_0(r)a_0x^r+\left[p_0(r+1)a_1+p_1(r)a_2\right]x^{r+1}\\
&& +\sum_{n=2}^\infty\left[p_0(n+r)a_n+p_1(n+r-1)a_{n-1}
+p_2(n+r-2)a_{n-2}\right]x^{n+r},
\end{eqnarray*}
which implies \eqref{eq:7.5.4} with $\{b_n\}$ defined as in
\eqref{eq:7.5.5}.
\end{proof}

\begin{theorem}\label{thmtype:7.5.2}
Let
$$
Ly=
x^2(\alpha_0+\alpha_1x+\alpha_2x^2)y''+x(\beta_0+\beta_1x+\beta_2x^2)y'
+(\gamma_0+\gamma_1x+\gamma_2x^2)y,
$$
where $\alpha_0\neq 0,$ and define
\begin{eqnarray*}
p_0(r)&=&\alpha_0r(r-1)+\beta_0r+\gamma_0,\\
p_1(r)&=&\alpha_1r(r-1)+\beta_1r+\gamma_1,\\
p_2(r)&=&\alpha_2r(r-1)+\beta_2r+\gamma_2.\\
\end{eqnarray*}
 Suppose $r$ is a real number such that $p_0(n+r)$ is nonzero
for all positive integers $n.$ Define
\begin{equation} \label{eq:7.5.12}
\begin{array}{ccl}
a_0(r)&=&1,\\
a_1(r)&=&-\frac{p_1(r)}{p_0(r+1)},\\
a_n(r)&=&-\frac{p_1(n+r-1)a_{n-1}(r)+p_2(n+r-2)a_{n-2}(r)}{p_0(n+r)},\quad n\geq2.
\end{array}
\end{equation}
Then the Frobenius series
\begin{equation} \label{eq:7.5.13}
y(x,r)=x^r\sum_{n=0}^\infty a_n(r)x^n
\end{equation}
converges and satisfies
\begin{equation} \label{eq:7.5.14}
Ly(x,r)=p_0(r)x^r
\end{equation}
on the interval $(0,\rho)$, where
 $\rho$ is the distance from the origin to the
nearest zero of $A(x)=\alpha_0+\alpha_1 x+\alpha_2 x^2$ in the complex
plane.
$($If
$A$ is constant, then $\rho=\infty$.$)$
\end{theorem}

 If $\{a_n(r)\}$ is determined by the recurrence relation
\eqref{eq:7.5.12} then substituting $a_n=a_n(r)$ into \eqref{eq:7.5.5}
yields
$b_0=p_0(r)$ and $b_n=0$ for $n\geq1$, so \eqref{eq:7.5.4} reduces to
\eqref{eq:7.5.14}. We omit the proof that the series \eqref{eq:7.5.13}
converges on  $(0,\rho)$.

If $\alpha_i=\beta_i=\gamma_i=0$ for $i=1$, $2,$ then $Ly=0$ reduces to
the Euler equation
$$
\alpha_0x^2y''+\beta_0xy'+\gamma_0y=0.
$$
Theorem~\ref{thmtype:7.4.3} shows that the solutions of this equation are
determined by the zeros of the indicial polynomial
$$
p_0(r)=\alpha_0r(r-1)+\beta_0r+\gamma_0.
$$
Since \eqref{eq:7.5.14} implies that this is also true for the solutions
of $Ly=0$, we'll also say that $p_0$ is the \dfn{indicial
polynomial} of \eqref{eq:7.5.2}, and that $p_0(r)=0$ is the \dfn{indicial equation} of $Ly=0$. We'll consider only cases where the
indicial equation has real roots $r_1$ and $r_2$, with $r_1\geq r_2$.

\begin{theorem}\label{thmtype:7.5.3}
Let $L$ and $\{a_n(r)\}$ be as in Theorem~$\ref{thmtype:7.5.2},$ and
suppose the
indicial equation $p_0(r)=0$ of $Ly=0$ has real roots $r_1$ and $r_2,$
where $r_1\geq r_2.$ Then
$$
y_1(x)=y(x,r_1)=x^{r_1}\sum_{n=0}^\infty a_n(r_1)x^n
$$
is a Frobenius solution of $Ly=0$. Moreover$,$ if $r_1-r_2$ isn't  an
integer then
$$
y_2(x)=y(x,r_2)=x^{r_2}\sum_{n=0}^\infty a_n(r_2)x^n
$$
is also a Frobenius solution of $Ly=0,$
and $\{y_1,y_2\}$ is a fundamental set of solutions.
\end{theorem}

\begin{proof}
Since $r_1$ and $r_2$ are roots of $p_0(r)=0$, the indicial polynomial
can be factored as
\begin{equation} \label{eq:7.5.15}
p_0(r)=\alpha_0(r-r_1)(r-r_2).
\end{equation}
Therefore
$$
p_0(n+r_1)=n\alpha_0(n+r_1-r_2),
$$
which is nonzero if $n>0$, since $r_1-r_2\geq0$. Therefore the
assumptions of Theorem~\ref{thmtype:7.5.2} hold with $r=r_1$, and
\eqref{eq:7.5.14} implies that $Ly_1=p_0(r_1)x^{r_1}=0$.

Now suppose $r_1-r_2$ isn't  an integer. From \eqref{eq:7.5.15},
$$
p_0(n+r_2)=n\alpha_0(n-r_1+r_2)\neq 0\quad\mbox{if}\quad n=1,2,\cdots.
$$
Hence, the assumptions of Theorem~\ref{thmtype:7.5.2} hold with $r=r_2$,
and \eqref{eq:7.5.14} implies that $Ly_2=p_0(r_2)x^{r_2}=0$. We leave the
proof that $\{y_1,y_2\}$ is a fundamental set of solutions as an
exercise. % (Exercise~\ref{exer:7.5.52}). \bbox
\end{proof}

It isn't  always possible to obtain explicit formulas for the
coefficients in Frobenius solutions. However, we can always set up the
recurrence relations and use them to compute as many coefficients as we
want. The next example illustrates this.

\begin{example}\label{example:7.5.1}
Find a fundamental set of
Frobenius solutions of
\begin{equation} \label{eq:7.5.16}
2x^2(1+x+x^2)y''+x(9+11x+11x^2)y'+(6+10x+7x^2)y=0.
\end{equation}
Compute just the first six coefficients $a_0,\dots, a_5$ in each
solution.

\begin{explanation}
For  the given equation, the polynomials defined in
Theorem~\ref{thmtype:7.5.2} are
$$
\begin{array}{ccccc}
p_0(r)&=&2r(r-1)+9r+6&=&(2r+3)(r+2),\\
p_1(r)&=&2r(r-1)+11r+10&=&(2r+5)(r+2),\\ 
p_2(r)&=&2r(r-1)+11r+7&=&(2r+7)(r+1).
\end{array}
$$
The zeros of the indicial polynomial $p_0$ are $r_1=-3/2$
and $r_2=-2$, so  $r_1-r_2=1/2$. Therefore
Theorem~\ref{thmtype:7.5.3} implies that
\begin{equation} \label{eq:7.5.17}
y_1=x^{-3/2}\sum_{n=0}^\infty a_n(-3/2)x^n\quad\mbox{and}\quad
y_2=x^{-2}\sum_{n=0}^\infty a_n(-2)x^n
\end{equation}
form a fundamental set of Frobenius solutions of \eqref{eq:7.5.16}.
To find the coefficients in these series, we use the recurrence
relation of Theorem~\ref{thmtype:7.5.2};   thus,
\begin{eqnarray*}
a_0(r)&=&1,\\
a_1(r)&=&-\frac{p_1(r)}{p_0(r+1)}
=-\frac{(2r+5)(r+2)}{(2r+5)(r+3)}
=-\frac{r+2}{r+3},\\
a_n(r)&=&-\frac{p_1(n+r-1)a_{n-1}+p_2(n+r-2)a_{n-2}}{p_0(n+r)}\\
&=&-\frac{(n+r+1)(2n+2r+3)a_{n-1}(r)
+(n+r-1)(2n+2r+3)a_{n-2}(r)}{(n+r+2)(2n+2r+3)}\\
&=&-\frac{(n+r+1)a_{n-1}(r)+(n+r-1)a_{n-2}(r)}{n+r+2},\quad n\geq2.
\end{eqnarray*}

Setting $r=-3/2$ in these equations yields
\begin{equation} \label{eq:7.5.18}
\begin{array}{ccl}
a_0(-3/2)&=&1,\\
a_1(-3/2)&=&-1/3,\\
a_n(-3/2)&=&-\frac{(2n-1)a_{n-1}(-3/2)+
(2n-5)a_{n-2}(-3/2)}{2n+1},\quad n\geq2,
\end{array}
\end{equation}
and setting $r=-2$ yields
\begin{equation} \label{eq:7.5.19}
\begin{array}{ccl}
a_0(-2)&=&1,\\
a_1(-2)&=&0,\\
a_n(-2)&=&-\frac{(n-1)a_{n-1}(-2)+(n-3)a_{n-2}(-2)}{n},\quad n\geq2.
\end{array}
\end{equation}
Calculating  with  \eqref{eq:7.5.18} and
\eqref{eq:7.5.19}  and substituting the results into
\eqref{eq:7.5.17}   yields the fundamental set of  Frobenius
solutions
\begin{eqnarray*}
y_1&=&x^{-3/2}\left(1-\frac{1}{3}x+\frac{2}{5}x^2-\frac{5}{21}x^3
+\frac{7}{135}x^4+\frac{76}{1155}x^5+\cdots\right),\\
y_2&=&x^{-2}\left(1+\frac{1}{2}x^2-\frac{1}{3}x^3+\frac{1}{8}x^4+\frac{1}{30}x^5
+\cdots\right).
\end{eqnarray*}
\end{explanation}
\end{example}

\subsection*{Special Cases With Two Term Recurrence Relations}

For $n\geq2$, the recurrence relation \eqref{eq:7.5.12} of
Theorem~\ref{thmtype:7.5.2} involves the three coefficients
$a_n(r)$, $a_{n-1}(r)$, and $a_{n-2}(r)$. We'll now consider some
special cases where \eqref{eq:7.5.12} reduces to a two term recurrence
relation; that is, a relation involving only $a_n(r)$ and $a_{n-1}(r)$
or only $a_n(r)$ and $a_{n-2}(r)$. This simplification often makes it
possible to obtain explicit formulas for the coefficents of Frobenius
solutions.

We first consider equations of the form
$$
x^2(\alpha_0+\alpha_1x)y''+x(\beta_0+\beta_1x)y'+(\gamma_0+\gamma_1x)y=0
$$
with $\alpha_0\neq 0$. For this equation,
$\alpha_2=\beta_2=\gamma_2=0$, so $p_2\equiv0$ and the recurrence
relations in Theorem~\ref{thmtype:7.5.2} simplify to
\begin{equation} \label{eq:7.5.20}
\begin{array}{ccl}
a_0(r)&=&1,\\
a_n(r)&=&-\frac{p_1(n+r-1)}{p_0(n+r)}a_{n-1}(r),\quad n\geq1.
\end{array}
\end{equation}

\begin{example}\label{example:7.5.2}
Find a fundamental set of Frobenius  solutions of
\begin{equation} \label{eq:7.5.21}
x^2(3+x)y''+5x(1+x)y'-(1-4x)y=0.
\end{equation}
Give explicit  formulas for the coefficients in the solutions.

\begin{explanation}
For this equation, the polynomials defined in
Theorem~\ref{thmtype:7.5.2} are
$$
\begin{array}{ccccc}
p_0(r)&=&3r(r-1)+5r-1&=&(3r-1)(r+1),\\
p_1(r)&=&r(r-1)+5r+4&=&(r+2)^2,\\
p_2(r)&=&0.
\end{array}
$$
The zeros of the indicial polynomial $p_0$ are $r_1=1/3$ and $r_2=-1$,
so $r_1-r_2=4/3$. Therefore Theorem~\ref{thmtype:7.5.3} implies that
$$
y_1=x^{1/3}\sum_{n=0}^\infty a_n(1/3)x^n\quad\mbox{and}\quad
y_2=x^{-1}\sum_{n=0}^\infty a_n(-1)x^n
$$
form a fundamental set of Frobenius solutions of \eqref{eq:7.5.21}. To
find the coefficients in these series, we use the recurrence relationss
\eqref{eq:7.5.20};   thus,
\begin{equation} \label{eq:7.5.22}
\begin{array}{ccl}
a_0(r)&=&1,\\
a_n(r)&=&-\frac{p_1(n+r-1)}{p_0(n+r)}a_{n-1}(r)\\
&=&-\frac{(n+r+1)^2}{(3n+3r-1)(n+r+1)}a_{n-1}(r)\\
&=&-\frac{n+r+1}{3n+3r-1}a_{n-1}(r),\quad n\geq1.
\end{array}
\end{equation}

Setting $r=1/3$ in \eqref{eq:7.5.22} yields
\begin{eqnarray*}
a_0(1/3)&=&1,\\
a_n(1/3)&=&-\frac{3n+4}{9n} a_{n-1}(1/3),\quad n\geq1.
\end{eqnarray*}
By using the product notation introduced in Section~7.2 and
proceeding as we did in the examples in that section yields
$$
a_n(1/3)=\frac{(-1)^n\prod_{j=1}^n(3j+4)}{9^nn!},\quad n\geq0.
$$
Therefore
$$
y_1=x^{1/3}\sum_{n=0}^\infty\frac{(-1)^n\prod_{j=1}^n(3j+4)}{9^nn!}x^n
$$
is a Frobenius solution of \eqref{eq:7.5.21}.

Setting $r=-1$ in \eqref{eq:7.5.22} yields
\begin{eqnarray*}
a_0(-1)&=&1,\\
a_n(-1)&=&-\frac{n}{3n-4}a_{n-1}(-1),\quad n\geq1,
\end{eqnarray*}
so
$$
a_n(-1)=\frac{(-1)^nn!}{\prod_{j=1}^n(3j-4)}.
$$
Therefore
$$
y_2=x^{-1}\sum_{n=0}^\infty\frac{(-1)^nn!}{\prod_{j=1}^n(3j-4)}x^n
$$
is a Frobenius solution of \eqref{eq:7.5.21}, and $\{y_1,y_2\}$ is a
fundamental set of solutions.
\end{explanation}
\end{example}

We now consider equations of the form
\begin{equation} \label{eq:7.5.23}
x^2(\alpha_0+\alpha_2x^2)y''+x(\beta_0+\beta_2x^2)y'+
(\gamma_0+\gamma_2x^2)y=0
\end{equation}
with $\alpha_0\neq 0$. For this equation,
$\alpha_1=\beta_1=\gamma_1=0$, so $p_1\equiv0$ and the recurrence
relations in Theorem~\ref{thmtype:7.5.2} simplify to
\begin{eqnarray*}
a_0(r)&=&1,\\
a_1(r)&=&0,\\
a_n(r)&=&-\frac{p_2(n+r-2)}{p_0(n+r)}a_{n-2}(r),\quad n\geq2.
\end{eqnarray*}
Since $a_1(r)=0$, the last equation implies that $a_n(r)=0$ if $n$ is
odd, so the Frobenius solutions are of the form
$$
y(x,r)=x^r\sum_{m=0}^\infty a_{2m}(r)x^{2m},
$$
where
\begin{equation} \label{eq:7.5.24}
\begin{array}{ccl}
a_0(r)&=&1,\\
a_{2m}(r)&=&-\frac{p_2(2m+r-2)}{p_0(2m+r)}a_{2m-2}(r),\quad m\geq1.
\end{array}
\end{equation}

\begin{example}\label{example:7.5.3}
Find a fundamental set of   Frobenius solutions of
\begin{equation} \label{eq:7.5.25}
x^2(2-x^2)y''-x(3+4x^2)y'+(2-2x^2)y=0.
\end{equation}
Give explicit  formulas for the coefficients in the solutions.

\begin{explanation}
For  this equation,  the polynomials defined in
Theorem~\ref{thmtype:7.5.2} are
$$
\begin{array}{ccccc}
p_0(r)&=&2r(r-1)-3r+2&=&(r-2)(2r-1),\\
p_1(r)&=&0\\
p_2(r)&=&-\left[r(r-1)+4r+2\right]&=&-(r+1)(r+2).
\end{array}
$$
The zeros of the indicial polynomial $p_0$ are $r_1=2$
and $r_2=1/2$, so  $r_1-r_2=3/2$. Therefore
Theorem~\ref{thmtype:7.5.3} implies that
$$
y_1=x^2\sum_{m=0}^\infty a_{2m}(1/3)x^{2m}\quad\mbox{and}\quad
y_2=x^{1/2}\sum_{m=0}^\infty a_{2m}(1/2)x^{2m}
$$
form a fundamental set of  Frobenius solutions of \eqref{eq:7.5.25}.
To find the coefficients in these series, we use the recurrence
relation \eqref{eq:7.5.24};   thus,
\begin{equation} \label{eq:7.5.26}
\begin{array}{ccl}
a_0(r)&=&1,\\
a_{2m}(r)&=&-\frac{p_2(2m+r-2)}{p_0(2m+r)}a_{2m-2}(r)\\
&=&\frac{(2m+r)(2m+r-1)}{(2m+r-2)(4m+2r-1)}a_{2m-2}(r),\quad m\geq1.
\end{array}
\end{equation}

Setting $r=2$ in \eqref{eq:7.5.26} yields
\begin{eqnarray*}
a_0(2)&=&1,\\
a_{2m}(2)&=&\frac{(m+1)(2m+1)}{m(4m+3)}a_{2m-2}(2),\quad m\geq1,
\end{eqnarray*}
so
$$
a_{2m}(2)=(m+1)\prod_{j=1}^m\frac{2j+1}{4j+3}.
$$
Therefore

$$
 y_1=x^2\sum_{m=0}^\infty
(m+1)\left(\prod_{j=1}^m\frac{2j+1}{4j+3}\right)x^{2m}
$$
is a Frobenius solution of \eqref{eq:7.5.25}.


Setting $r=1/2$ in \eqref{eq:7.5.26} yields
\begin{eqnarray*}
a_0(1/2)&=&1,\\
a_{2m}(1/2)&=&\frac{(4m-1)(4m+1)}{8m(4m-3)}a_{2m-2}(1/2),\quad
m\geq1,
\end{eqnarray*}
so
$$
a_{2m}(1/2)=\frac{1}{8^mm!}\prod_{j=1}^m\frac{(4j-1)(4j+1)}{4j-3}.
$$
Therefore
$$
y_2=x^{1/2}\sum_{m=0}^\infty
\frac{1}{8^mm!}\left(\prod_{j=1}^m\frac{(4j-1)(4j+1)}{4j-3}\right)x^{2m}
$$
is a Frobenius solution of \eqref{eq:7.5.25} and $\{y_1,y_2\}$ is a
fundamental set of solutions.
\end{explanation}
\end{example}

\begin{remark}
Thus far, we considered only the case where the indicial equation
has real roots that don't differ by an integer, which allows us to
apply Theorem~\ref{thmtype:7.5.3}. However, for equations of the form
\eqref{eq:7.5.23}, the sequence $\{a_{2m}(r)\}$ in \eqref{eq:7.5.24} is
defined for $r=r_2$ if $r_1-r_2$ isn't  an \dfn{even} integer. It can be
shown %(Exercise~\ref{exer:7.5.56}) 
that in this case
$$
y_1=x^{r_1}\sum_{m=0}^\infty a_{2m}(r_1)x^{2m}\quad \mbox{and}\quad y_2=x^{r_2}\sum_{m=0}^\infty a_{2m}(r_2)x^{2m}
$$
form a fundamental set  Frobenius solutions of \eqref{eq:7.5.23}.
\end{remark}

\subsection*{A Note on Technology}
As we said at the end of Section~7.2, if you're interested in
actually using series to compute numerical approximations to solutions
of a differential equation, then whether or not there's a simple
closed form for the coefficents is essentially irrelevant;   recursive
computation is usually more efficient. Since it's also laborious, we
encourage you to write short programs to implement recurrence
relations on a calculator or computer, even in exercises where this is
not specifically required.

In practical use of the method of Frobenius when $x_0=0$ is a regular
singular point, we're interested in how well the functions
$$
y_N(x,r_i)=x^{r_i}\sum_{n=0}^N a_n(r_i)x^n,\quad i=1,2,
$$
approximate solutions to a given equation when $r_i$ is a zero of the
indicial polynomial. In dealing with the corresponding problem for the
case where $x_0=0$ is an ordinary point, we used numerical integration
to solve the differential equation subject to initial conditions
$y(0)=a_0,\quad y'(0)=a_1$, and compared the result with values of the
Taylor polynomial
$$
T_N(x)=\sum_{n=0}^Na_nx^n.
$$
We can't do that here, since in general we can't prescribe
arbitrary initial values for solutions of a differential equation at a
singular point. Therefore, motivated by Theorem~\ref{thmtype:7.5.2}
(specifically, \eqref{eq:7.5.14}), we suggest the following  procedure.

\begin{procedure}
Let $L$ and
$Y_n(x; r_i)$ be defined by
$$
Ly=
x^2(\alpha_0+\alpha_1x+\alpha_2x^2)y''+x(\beta_0+\beta_1x+\beta_2x^2)y'
+(\gamma_0+\gamma_1x+\gamma_2x^2)y
$$
and
$$
y_N(x;r_i)=x^{r_i}\sum_{n=0}^N a_n(r_i)x^n,
$$
where the coefficients $\{a_n(r_i)\}_{n=0}^N$ are computed as in
$\eqref{eq:7.5.12}$, Theorem~$\ref{thmtype:7.5.2}$. Compute the error
\begin{equation} \label{eq:7.5.27}
E_N(x;r_i)=x^{-r_i}Ly_N(x;r_i)/\alpha_0
\end{equation}
for various values of $N$ and various values of $x$ in the interval
$(0,\rho)$, with $\rho$ as defined in Theorem~$\ref{thmtype:7.5.2}$.
\end{procedure}

The multiplier $x^{-r_i}/\alpha_0$ on the right of \eqref{eq:7.5.27}
eliminates the effects of small or large values of $x^{r_i}$ near
$x=0$, and of multiplication by an arbitrary constant. 
%In some
%exercises you will be asked to estimate the maximum value of
%$E_N(x; r_i)$ on an interval $(0,\delta]$ by computing $E_N(x_m;r_i)$
%at the $M$ points $x_m=m\delta/M,\; m=1$, $2$, \dots, $M$, and finding the
%maximum of the absolute values:
%\begin{equation} \label{eq:7.5.28}
%\sigma_N(\delta)=\max\{|E_N(x_m;r_i)|,\;   m=1,2,\dots,M\}.
%\end{equation}
%(For simplicity, this notation ignores the dependence of
%the right side of the equation on $i$ and $M$.)


To implement this procedure, you'll have to write a computer program to
calculate $\{a_n(r_i)\}$ from the applicable recurrence relation, and
to evaluate $E_N(x;r_i)$.

%The next exercise set contains five exercises specifically
%identified by \Lex that ask you to implement the verification
%procedure. These particular exercises were chosen arbitrarily   you can
%just as well formulate such laboratory problems for any of the
%equations in any  of the Exercises~\ref{exer:7.5.1}--\ref{exer:7.5.10},
%\ref{exer:7.5.14}-\ref{exer:7.5.25}, and \ref{exer:7.5.28}--\ref{exer:7.5.51}



\section*{Text Source}
Trench, William F., "Elementary Differential Equations" (2013). Faculty Authored and Edited Books \& CDs. 8. (CC-BY-NC-SA)

\href{https://digitalcommons.trinity.edu/mono/8/}{https://digitalcommons.trinity.edu/mono/8/}


\end{document}