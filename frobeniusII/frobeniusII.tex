\documentclass{ximera}

%% You can put user macros here
%% However, you cannot make new environments

\listfiles

\graphicspath{{./}{firstExample/}{secondExample/}}

\usepackage{tikz}
\usepackage{tkz-euclide}
\usepackage{tikz-3dplot}
\usepackage{tikz-cd}
\usetikzlibrary{shapes.geometric}
\usetikzlibrary{arrows}
\usetikzlibrary{decorations.pathmorphing,patterns}
\usetkzobj{all}
\pgfplotsset{compat=1.13} % prevents compile error.

\renewcommand{\vec}[1]{\mathbf{#1}}
\newcommand{\RR}{\mathbb{R}}
\newcommand{\dfn}{\textit}
\newcommand{\dotp}{\cdot}
\newcommand{\id}{\text{id}}
\newcommand\norm[1]{\left\lVert#1\right\rVert}
 
\newtheorem{general}{Generalization}
\newtheorem{initprob}{Exploration Problem}

\tikzstyle geometryDiagrams=[ultra thick,color=blue!50!black]

\usepackage{mathtools}

\title{7.6 The Method of Frobenius II}%\label{Module 7-ADEF}


\begin{document}

\begin{abstract}
We continue our study of the method of Frobenius for finding series solutions of linear second order differential equations, extending to the case where the indicial equation has a repeated real root.
\end{abstract}

\maketitle

\section*{The Method of Frobenius II}

In this section we discuss a method for finding two linearly
independent Frobenius solutions of a homogeneous linear second order
equation near a regular singular point in the case where the indicial
equation has a repeated real root. As in the preceding section, we
consider equations that can be written as
\begin{equation} \label{eq:7.6.1}
x^2(\alpha_0+\alpha_1x+\alpha_2x^2)y''+x(\beta_0+\beta_1x+\beta_2x^2)y'
+(\gamma_0+\gamma_1x+\gamma_2x^2)y=0
\end{equation}
where $\alpha_0\ne0$. We assume that the indicial equation $p_0(r)=0$
has a repeated real root $r_1$. In this case Theorem~\ref{thmtype:7.5.3}
implies that \eqref{eq:7.6.1} has one solution of the form
$$
y_1=x^{r_1}\sum_{n=0}^\infty a_nx^n,
$$
but does not provide a second solution $y_2$ such that $\{y_1,y_2\}$
is a fundamental set of solutions. The following extension of
Theorem~\ref{thmtype:7.5.2} provides a way to find a second solution.

\begin{theorem}\label{thmtype:7.6.1}
Let
\begin{equation} \label{eq:7.6.2}
Ly=
x^2(\alpha_0+\alpha_1x+\alpha_2x^2)y''+x(\beta_0+\beta_1x+\beta_2x^2)y'
+(\gamma_0+\gamma_1x+\gamma_2x^2)y,
\end{equation}
where $\alpha_0\ne0$ and define
\begin{eqnarray*}
p_0(r)&=&\alpha_0r(r-1)+\beta_0r+\gamma_0,\\
p_1(r)&=&\alpha_1r(r-1)+\beta_1r+\gamma_1,\\
p_2(r)&=&\alpha_2r(r-1)+\beta_2r+\gamma_2.\\
\end{eqnarray*}
Suppose $r$ is
a real number such that $p_0(n+r)$ is nonzero
for all positive integers $n$, and define
$$
\begin{array}{ccl}
a_0(r)&=&1,\\
a_1(r)&=&-\frac{p_1(r)}{p_0(r+1)},\\
a_n(r)&=&-\frac{p_1(n+r-1)a_{n-1}(r)+p_2(n+r-2)a_{n-2}(r)}{p_0(n+r)},\quad n\geq 2.
\end{array}
$$
Then the Frobenius series
\begin{equation} \label{eq:7.6.3}
y(x,r)=x^r\sum_{n=0}^\infty a_n(r)x^n
\end{equation}
satisfies
\begin{equation} \label{eq:7.6.4}
Ly(x,r)=p_0(r)x^r.
\end{equation}
Moreover$,$
\begin{equation} \label{eq:7.6.5}
\frac{\partial y}{\partial r}(x,r)=y(x,r)\ln x+x^r\sum_{n=1}^\infty
a_n'(r) x^n,
\end{equation}
and
\begin{equation} \label{eq:7.6.6}
L\left(\frac{\partial y}{\partial
r}(x,r)\right)=p'_0(r)x^r+x^rp_0(r)\ln x.
\end{equation}
\end{theorem}

\begin{proof}
Theorem~\ref{thmtype:7.5.2} implies \eqref{eq:7.6.4}. Differentiating
formally
with respect to $r$ in \eqref{eq:7.6.3} yields
\begin{eqnarray*}
\frac{\partial y}{\partial r} (x,r)&=&\frac{\partial}{\partial r}(x^r)\sum_{n=0}^\infty
a_n(r)x^n +x^r\sum_{n=1}^\infty  a_n'(r)x^n\\
&=&x^r\ln x\sum_{n=0}^\infty
a_n(r)x^n +x^r\sum_{n=1}^\infty  a_n'(r)x^n\\
&=&y(x,r) \ln x +
x^r\sum_{n=1}^\infty  a_n'(r)x^n,
\end{eqnarray*}
which proves \eqref{eq:7.6.5}.

To prove that  $\partial y(x,r)/\partial r$ satisfies \eqref{eq:7.6.6}, we
view $y$ in \eqref{eq:7.6.2} as a function $y=y(x,r)$  of two
variables, where the prime indicates partial differentiation with
respect to $x$;   thus,
$$
y'=y'(x,r)=\frac{\partial y}{\partial x}(x,r)\quad\mbox{and}\quad
y''=y''(x,r)=\frac{\partial^2 y}{\partial x^2}(x,r).
$$
With this notation we can use \eqref{eq:7.6.2} to rewrite \eqref{eq:7.6.4} as
\begin{equation} \label{eq:7.6.7}
x^2q_0(x)\frac{\partial^2 y}{\partial x^2}(x,r)+xq_1(x)\frac{\partial y}{\partial x}(x,r)+q_2(x)y(x,r)=p_0(r)x^r,
\end{equation}
where
\begin{eqnarray*}
q_0(x)&=&\alpha_0+\alpha_1x+\alpha_2x^2,\\
q_1(x)&=&\beta_0+\beta_1x+\beta_2x^2,\\
q_2(x)&=&\gamma_0+\gamma_1x+\gamma_2x^2.\\
\end{eqnarray*}
Differentiating both sides of \eqref{eq:7.6.7} with respect to $r$
yields
$$
x^2q_0(x)\frac{\partial^3y}{\partial r\partial x^2}(x,r)+
xq_1(x)\frac{\partial^2y}{\partial r\partial x}(x,r)+q_2(x)\frac{\partial
y}{\partial r}(x,r)=p'_0(r)x^r+p_0(r) x^r \ln x.
$$
By changing the order of differentiation in the first two terms on the left
we can rewrite this  as
$$
x^2q_0(x)\frac{\partial^3 y}{\partial x^2\partial r}(x,r)
+xq_1(x)\frac{\partial^2 y}{\partial x\partial r}(x,r)+q_2(x)\frac{\partial
y}{\partial r}(x,r)=p'_0(r)x^r+p_0(r) x^r \ln x,
$$
or
$$
x^2q_0(x)\frac{\partial^2}{\partial x^2}
\left(\frac{\partial y}{\partial r}(x,r)\right)
+xq_1(x)\frac{\partial}{\partial r}\left(\frac{\partial y}{\partial
x}(x,r)\right) +q_2(x)\frac{\partial y}{\partial r}(x,r)=
p'_0(r)x^r+p_0(r) x^r \ln x,
$$
which is equivalent to \eqref{eq:7.6.6}.
\end{proof}

\begin{theorem}\label{thmtype:7.6.2}
Let $L$ be as in Theorem~\ref{thmtype:7.6.1} and suppose the
indicial equation $p_0(r)=0$ has a repeated real root $r_1$.
Then
$$
y_1(x)=y(x,r_1)=x^{r_1}\sum_{n=0}^\infty a_n(r_1)x^n
$$
and
\begin{equation} \label{eq:7.6.8}
y_2(x)=\frac{\partial y}{\partial r}(x,r_1)=y_1(x)\ln
x+x^{r_1}\sum_{n=1}^\infty a_n'(r_1)x^n
\end{equation}
form a fundamental set of solutions of $Ly=0.$
\end{theorem}

\begin{proof}
Since $r_1$  is a repeated root of $p_0(r)=0$, the indicial polynomial
can be factored as
$$
p_0(r)=\alpha_0(r-r_1)^2,
$$
so
$$
p_0(n+r_1)=\alpha_0n^2,
$$
which is nonzero if $n>0$. Therefore the assumptions of
Theorem~\ref{thmtype:7.6.1} hold with $r=r_1$, and \eqref{eq:7.6.4} implies
that $Ly_1=p_0(r_1)x^{r_1}=0$. Since
$$
p_0'(r)=2\alpha(r-r_1)
$$
it follows that $p_0'(r_1)=0$, so \eqref{eq:7.6.6} implies that
$$
Ly_2=p_0'(r_1)x^{r_1}+x^{r_1}p_0(r_1)\ln x=0.
$$
This proves that $y_1$ and $y_2$ are both solutions of $Ly=0$. We
leave the proof that $\{y_1,y_2\}$ is a fundamental set as an exercise.
%(Exercise~\ref{exer:7.6.53}).
\end{proof}

\begin{example}\label{example:7.6.1}
Find  a fundamental set of solutions of
\begin{equation} \label{eq:7.6.9}
x^2(1-2x+x^2)y''-x(3+x)y'+(4+x)y=0.
\end{equation}
Compute just the terms involving $x^{n+r_1}$, where  $0\leq n\leq 4$
and $r_1$ is the root of the indicial equation.


\begin{explanation}
For the given equation,  the polynomials defined in
Theorem~\ref{thmtype:7.6.1} are
$$
\begin{array}{lllll}
p_0(r)&=&r(r-1)-3r+4&=&(r-2)^2,\\
p_1(r)&=&-2r(r-1)-r+1&=&-(r-1)(2r+1),\\
p_2(r)&=&r(r-1).
\end{array}
$$
Since $r_1=2$ is a repeated root of the indicial polynomial $p_0$,
Theorem~\ref{thmtype:7.6.2} implies that
\begin{equation} \label{eq:7.6.10}
y_1=x^2\sum_{n=0}^\infty  a_n(2)x^n\quad\mbox{ and }\quad
y_2=y_1\ln x+x^2\sum_{n=1}^\infty a_n'(2)x^n
\end{equation}
form a fundamental set of  Frobenius solutions of \eqref{eq:7.6.9}.
To find the coefficients in these series, we use the recurrence
formulas from Theorem~\ref{thmtype:7.6.1}:
\begin{equation} \label{eq:7.6.11}
\begin{array}{ccl}
a_0(r)&=&1,\\
a_1(r)&=&-\frac{p_1(r)}{p_0(r+1)}
=-\frac{(r-1)(2r+1)}{(r-1)^2}
=\frac{2r+1}{r-1},\\
a_n(r)&=&-\frac{p_1(n+r-1)a_{n-1}(r)+p_2(n+r-2)a_{n-2}(r)}{p_0(n+r)}\\
&=&\frac{(n+r-2)\left[(2n+2r-1)a_{n-1}(r)
-(n+r-3)a_{n-2}(r)\right]}{(n+r-2)^2}\\
&=&\frac{(2n+2r-1)}{(n+r-2)}a_{n-1}(r)-
\frac{(n+r-3)}{(n+r-2)}a_{n-2}(r),n\geq 2.
\end{array}
\end{equation}
Differentiating yields
\begin{equation} \label{eq:7.6.12}
\begin{array}{ccl}
a'_1(r)&=&-\frac{3}{(r-1)^2},\\
a'_n(r)&=&\frac{2n+2r-1}{n+r-2}a'_{n-1}(r)-\frac{n+r-3}{n+r-2}a'_{n-2}(r)\\
&&-\frac{3}{(n+r-2)^2}a_{n-1}(r)-\frac{1}{(n+r-2)^2}a_{n-2}(r),\quad
n\geq 2.
\end{array}
\end{equation}

Setting $r=2$ in \eqref{eq:7.6.11} and \eqref{eq:7.6.12} yields
\begin{equation} \label{eq:7.6.13}
\begin{array}{ccl}
a_0(2)&=&1,\\
a_1(2)&=&5,\\
a_n(2)&=&\frac{(2n+3)}{n}
a_{n-1}(2)-\frac{(n-1)}{n}a_{n-2}(2),\quad n\geq 2
\end{array}
\end{equation}
and
\begin{equation} \label{eq:7.6.14}
\begin{array}{ccl}
a_1'(2)&=&-3,\\
a'_n(2)&=&\frac{2n+3}{n}a'_{n-1}(2)-\frac{n-1}{n}a'_{n-2}(2)
-\frac{3}{n^2}a_{n-1}(2)-\frac{1}{n^2}a_{n-2}(2),\quad n\geq 2.
\end{array}
\end{equation}
Computing recursively with  \eqref{eq:7.6.13} and \eqref{eq:7.6.14}
yields
$$
a_0(2)=1,\,a_1(2)=5,\,a_2(2)=17,\,a_3(2)=\frac{143}{3},\,a_4(2)=\frac{355}{3},
$$
and
$$
a_1'(2)=-3,\,a_2'(2)=-\frac{29}{2},\,a_3'(2)=-\frac{859}{18},
\,a_4'(2)=-\frac{4693}{36}.
$$
Substituting these coefficients into \eqref{eq:7.6.10} yields
$$
y_1=x^2\left(1+5x+17x^2+\frac{143}{3}x^3
+\frac{355}{3}x^4+\cdots\right)
$$
and
$$
y_2=y_1 \ln x
-x^3\left(3+\frac{29}{2}x+\frac{859}{18}x^2+\frac{4693}{36}x^3
+\cdots\right).
$$
\end{explanation}
\end{example}

Since the recurrence formula \eqref{eq:7.6.11} involves three terms, it's
not possible to obtain a simple explicit formula for the coefficients
in the Frobenius solutions of \eqref{eq:7.6.9}. However, as we saw in the
preceding sections, the recurrence formula for $\{a_n(r)\}$ involves
only two terms if either $\alpha_1=\beta_1=\gamma_1=0$ or
$\alpha_2=\beta_2=\gamma_2=0$ in \eqref{eq:7.6.1}. In this case, it's
often
possible to find explicit formulas for the coefficients. The next two
examples illustrate this.



\begin{example}\label{example:7.6.2}
Find a fundamental set of Frobenius  solutions of
\begin{equation} \label{eq:7.6.15}
2x^2(2+x)y''+5x^2y'+(1+x)y=0.
\end{equation}
Give explicit formulas for the coefficients in the solutions.

\begin{explanation}
For  the given equation, the polynomials defined in
Theorem~\ref{thmtype:7.6.1} are
$$
\begin{array}{cclcl}
p_0(r)&=&4r(r-1)+1&=&(2r-1)^2,\\
p_1(r)&=&2r(r-1)+5r+1&=&(r+1)(2r+1),\\
p_2(r)&=&0.
\end{array}
$$
Since $r_1=1/2$ is a repeated zero of the  indicial polynomial $p_0$,
Theorem~\ref{thmtype:7.6.2} implies that
\begin{equation} \label{eq:7.6.16}
y_1=x^{1/2}\sum_{n=0}^\infty  a_n(1/2)x^n
\end{equation}
and
\begin{equation} \label{eq:7.6.17}
y_2=y_1\ln x+x^{1/2}\sum_{n=1}^\infty a_n'(1/2)x^n
\end{equation}
form a fundamental set of Frobenius solutions of \eqref{eq:7.6.15}.
Since $p_2\equiv0$, the recurrence formulas in Theorem~\ref{thmtype:7.6.1}
 reduce to
$$
\begin{array}{ccl}
a_0(r)&=&1,\\
a_n(r)&=&-\frac{p_1(n+r-1)}{p_0(n+r)}a_{n-1}(r),\\
&=&-\frac{(n+r)(2n+2r-1)}{(2n+2r-1)^2}a_{n-1}(r),\\
&=&-\frac{n+r}{2n+2r-1}a_{n-1}(r),\quad n\geq 0.
\end{array}
$$
We leave it to you to show that
\begin{equation} \label{eq:7.6.18}
a_n(r)=(-1)^n\prod_{j=1}^n\frac{j+r}{2j+2r-1},\quad n\geq 0.
\end{equation}
Setting $r=1/2$  yields
\begin{equation} \label{eq:7.6.19}
\begin{array}{ccl}
a_n(1/2)&=&(-1)^n\prod_{j=1}^n\frac{j+1/2}{2j}=
(-1)^n\prod_{j=1}^n\frac{2j+1}{4j},\\
&=&\frac{(-1)^n\prod_{j=1}^n(2j+1)}{4^nn!},\quad n\geq 0.
\end{array}
\end{equation}
Substituting this into \eqref{eq:7.6.16} yields
$$
y_1=x^{1/2}\sum_{n=0}^\infty\frac{(-1)^n\prod_{j=1}^n(2j+1)}{4^nn!}x^n.
$$

To obtain $y_2$ in \eqref{eq:7.6.17}, we must compute $a_n'(1/2)$
for $n=1, 2,\dots$. We'll do this by logarithmic
differentiation.  From \eqref{eq:7.6.18},
$$
|a_n(r)|=\prod_{j=1}^n\frac{|j+r|}{|2j+2r-1|},\quad n\geq 1.
$$
Therefore
$$
\ln |a_n(r)|=\sum^n_{j=1} \left(\ln |j+r|-\ln|2j+2r-1|\right).
$$
Differentiating with respect to $r$ yields
$$
\frac{a'_n(r)}{a_n(r)}=\sum^n_{j=1} \left(\frac{1}{j+r}-\frac{2}{2j+2r-1}\right).
$$
Therefore
$$
a'_n(r)=a_n(r) \sum^n_{j=1} \left(\frac{1}{j+r}-\frac{2}{2j+2r-1}\right).
$$
Setting $r=1/2$ here and recalling \eqref{eq:7.6.19} yields
\begin{equation} \label{eq:7.6.20}
a'_n(1/2)=\frac{(-1)^n\prod_{j=1}^n(2j+1)}{4^nn!}\left(\sum_{j=1}^n\frac{1}{j+1/2}-\sum_{j=1}^n\frac{1}{j}\right).
\end{equation}
Since
$$
\frac{1}{j+1/2}-\frac{1}{j}=\frac{j-j-1/2}{j(j+1/2)}=-\frac{1}{j(2j+1)},
$$
 \eqref{eq:7.6.20} can be rewritten as
$$
a'_n(1/2)=-\frac{(-1)^n\prod_{j=1}^n(2j+1)}{4^nn!}
\sum_{j=1}^n\frac{1}{j(2j+1)}.
$$
Therefore, from \eqref{eq:7.6.17},
$$
y_2=y_1\ln
x-x^{1/2}\sum_{n=1}^\infty\frac{(-1)^n\prod_{j=1}^n(2j+1)}{4^nn!}
\left(\sum_{j=1}^n\frac{1}{j(2j+1)}\right)x^n.
$$
\end{explanation}
\end{example}

\begin{example}\label{example:7.6.3}
Find a fundamental set of Frobenius  solutions of
\begin{equation} \label{eq:7.6.21}
x^2(2-x^2)y''-2x(1+2x^2)y'+(2-2x^2)y=0.
\end{equation}
Give explicit formulas for the coefficients in the solutions.

\begin{explanation}
For \eqref{eq:7.6.21}, the polynomials defined in
Theorem~\ref{thmtype:7.6.1} are
$$
\begin{array}{ccccc}
p_0(r)&=&2r(r-1)-2r+2&=&2(r-1)^2,\\
p_1(r)&=&0,\\
p_2(r)&=&-r(r-1)-4r-2&=&-(r+1)(r+2).
\end{array}
$$
As in \href{https://ximera.osu.edu/ode/main/frobeniusI/frobeniusI}{Trench 7.5}, since $p_1\equiv0$,  the recurrence
formulas
of Theorem~\ref{thmtype:7.6.1} imply that $a_n(r)=0$ if $n$ is odd, and
$$
\begin{array}{ccl}
a_0(r)&=&1,\\
a_{2m}(r)&=&-\frac{p_2(2m+r-2)}{p_0(2m+r)}a_{2m-2}(r)\\
&=&\frac{(2m+r-1)(2m+r)}{2(2m+r-1)^2}a_{2m-2}(r)\\
&=&\frac{2m+r}{2(2m+r-1)}a_{2m-2}(r),\quad m\geq 1.
\end{array}
$$
Since $r_1=1$ is a repeated root of the indicial polynomial $p_0$,
Theorem~\ref{thmtype:7.6.2} implies that
\begin{equation} \label{eq:7.6.22}
y_1=x\sum_{m=0}^\infty  a_{2m}(1)x^{2m}
\end{equation}
and
\begin{equation} \label{eq:7.6.23}
y_2=y_1\ln x+x\sum_{m=1}^\infty a'_{2m}(1)x^{2m}
\end{equation}
form a fundamental set of Frobenius solutions of \eqref{eq:7.6.21}.
We leave it to you to show  that
\begin{equation} \label{eq:7.6.24}
a_{2m}(r)=\frac{1}{2^m}\prod_{j=1}^m\frac{2j+r}{2j+r-1}.
\end{equation}
Setting $r=1$ yields
\begin{equation} \label{eq:7.6.25}
a_{2m}(1)=\frac{1}{2^m}\prod_{j=1}^m\frac{2j+1}{2j}
=\frac{\prod_{j=1}^m(2j+1)}{4^mm!},
\end{equation}
and substituting this into \eqref{eq:7.6.22} yields
$$
y_1=x\sum_{m=0}^\infty\frac{\prod_{j=1}^m(2j+1)}{4^mm!}x^{2m}.
$$

To obtain $y_2$ in \eqref{eq:7.6.23}, we must compute $a_{2m}'(1)$
for $m=1, 2, \dots$. Again we use logarithmic differentiation. From
\eqref{eq:7.6.24},
$$
|a_{2m}(r)|=\frac{1}{2^m}\prod_{j=1}^m\frac{|2j+r|}{|2j+r-1|}.
$$
Taking logarithms yields
$$
\ln |a_{2m}(r)|=-m\ln2+ \sum^m_{j=1} \left(\ln
|2j+r|-\ln|2j+r-1|\right).
$$
Differentiating with respect to $r$ yields
$$
\frac{a'_{2m}(r)}{a_{2m}(r)}=\sum^m_{j=1} \left(\frac{1}{2j+r}-\frac{1}{2j+r-1}\right).
$$
Therefore
$$
a'_{2m}(r)=a_{2m}(r) \sum^m_{j=1} \left(\frac{1}{2j+r}-\frac{1}{2j+r-1}\right).
$$
Setting $r=1$  and recalling \eqref{eq:7.6.25} yields
\begin{equation} \label{eq:7.6.26}
a'_{2m}(1)=\frac{\prod_{j=1}^m(2j+1)}{4^mm!}
\sum_{j=1}^m\left(\frac{1}{2j+1}-\frac{1}{2j}\right).
\end{equation}
Since
$$
\frac{1}{2j+1}-\frac{1}{2j}=-\frac{1}{2j(2j+1)},
$$
\eqref{eq:7.6.26} can be rewritten as
$$
a_{2m}'(1)=-\frac{\prod_{j=1}^m(2j+1)}{2\cdot4^mm!}
\sum_{j=1}^m\frac{1}{j(2j+1)}.
$$
Substituting this into \eqref{eq:7.6.23} yields
$$
y_2=y_1\ln
x-\frac{x}{2}\sum_{m=1}^\infty\frac{\prod_{j=1}^m(2j+1)}{4^mm!}
\left(\sum_{j=1}^m\frac{1}{j(2j+1)}\right)x^{2m}.
$$
\end{explanation}
\end{example}

If the solution $y_1=y(x,r_1)$ of $Ly=0$ reduces to a finite sum, then
there's a
difficulty in using logarithmic differentiation to obtain the coefficients
$\{a_n'(r_1)\}$ in the second solution. The next example illustrates
this difficulty and shows how to overcome it.


\begin{example}\label{example:7.6.4}
Find a fundamental set of Frobenius solutions of
\begin{equation} \label{eq:7.6.27}
x^2y''-x(5-x)y'+(9-4x)y=0.
\end{equation}
Give explicit formulas for the coefficients in the solutions.

\begin{explanation}
For \eqref{eq:7.6.27} the polynomials defined in Theorem~\ref{thmtype:7.6.1}
are
$$
\begin{array}{cclcc}
p_0(r)&=&r(r-1)-5r+9&=&(r-3)^2,\\
p_1(r)&=&r-4,\\
p_2(r)&=&0.
\end{array}
$$
Since $r_1=3$ is a repeated zero of the  indicial polynomial $p_0$,
Theorem~\ref{thmtype:7.6.2} implies that
\begin{equation} \label{eq:7.6.28}
y_1=x^3\sum_{n=0}^\infty  a_n(3)x^n
\end{equation}
and
\begin{equation} \label{eq:7.6.29}
y_2=y_1\ln x+x^3\sum_{n=1}^\infty a_n'(3)x^n
\end{equation}
are linearly independent Frobenius solutions of \eqref{eq:7.6.27}.
To find the coefficients in \eqref{eq:7.6.28} we use the recurrence
formulas
$$
\begin{array}{ccl}
a_0(r)&=&1,\\
a_n(r)&=&-\frac{p_1(n+r-1)}{p_0(n+r)}a_{n-1}(r)\\
&=&-\frac{n+r-5}{(n+r-3)^2}a_{n-1}(r),\quad n\geq 1.
\end{array}
$$
We leave it to you to show that
\begin{equation} \label{eq:7.6.30}
a_n(r)=(-1)^n\prod_{j=1}^n\frac{j+r-5}{(j+r-3)^2}.
\end{equation}
Setting $r=3$ here yields
$$
a_n(3)=(-1)^n\prod_{j=1}^n\frac{j-2}{j^2},
$$
so  $a_1(3)=1$ and $a_n(3)=0$ if $n\ge2$. Substituting
these coefficients into \eqref{eq:7.6.28} yields
$$
y_1=x^3(1+x).
$$

To obtain $y_2$ in \eqref{eq:7.6.29} we must compute $a_n'(3)$
for $n=1, 2, \dots$.
 Let's first try  logarithmic differentiation.
From \eqref{eq:7.6.30},
$$
|a_n(r)|=\prod_{j=1}^n\frac{|j+r-5|}{|j+r-3|^2},\quad n\geq 1,
$$
so
$$
\ln |a_n(r)|=\sum^n_{j=1} \left(\ln |j+r-5|-2\ln|j+r-3|\right).
$$
Differentiating  with respect to $r$ yields
$$
\frac{a'_n(r)}{a_n(r)}=\sum^n_{j=1} \left(\frac{1}{j+r-5}-\frac{2}{j+r-3}\right).
$$
Therefore
\begin{equation} \label{eq:7.6.31}
a'_n(r)=a_n(r) \sum^n_{j=1} \left(\frac{1}{j+r-5}-\frac{2}{j+r-3}\right).
\end{equation}
However, we can't simply set $r=3$ here if $n\ge2$,  since the bracketed
expression in the sum corresponding to  $j=2$ contains the term $1/(r-3)$.
In fact, since $a_n(3)=0$ for $n\ge2$, the  formula \eqref{eq:7.6.31} for
$a_n'(r)$ is actually an indeterminate form at $r=3$.

We  overcome this difficulty as follows. From \eqref{eq:7.6.30} with
$n=1$,
$$
a_1(r)=-\frac{r-4}{(r-2)^2}.
$$
Therefore
$$
a_1'(r)=\frac{r-6}{(r-2)^3},
$$
so
\begin{equation} \label{eq:7.6.32}
a_1'(3)=-3.
\end{equation}
From \eqref{eq:7.6.30} with $n\geq 2$,
$$
a_n(r)=(-1)^n (r-4)(r-3)\,\frac{\prod_{j=3}^n(j+r-5)}{\prod_{j=1}^n(j+r-3)^2}
=(r-3)c_n(r),
$$
where
$$
c_n(r)=(-1)^n(r-4)\,
\frac{\prod_{j=3}^n(j+r-5)}{\prod_{j=1}^n(j+r-3)^2},\quad  n\geq 2.
$$
Therefore
$$
a_n'(r)=c_n(r)+(r-3)c_n'(r),\quad n\geq 2,
$$
which implies that  $a_n'(3)=c_n(3)$ if $n\geq 3$. We leave it to
you to verify that
$$
a_n'(3)=c_n(3)=\frac{(-1)^{n+1}}{n(n-1)n!},\quad n\geq 2.
$$
Substituting this and \eqref{eq:7.6.32} into \eqref{eq:7.6.29}
yields
$$
y_2=x^3(1+x)\ln x-3x^4-x^3\sum_{n=2}^\infty \frac{(-1)^n}{n(n-1)n!}x^n.
$$
\end{explanation}
\end{example}

\section*{Text Source}
Trench, William F., "Elementary Differential Equations" (2013). Faculty Authored and Edited Books \& CDs. 8. (CC-BY-NC-SA)

\href{https://digitalcommons.trinity.edu/mono/8/}{https://digitalcommons.trinity.edu/mono/8/}


\end{document}