\documentclass{ximera}

%% You can put user macros here
%% However, you cannot make new environments

\listfiles

\graphicspath{{./}{firstExample/}{secondExample/}}

\usepackage{tikz}
\usepackage{tkz-euclide}
\usepackage{tikz-3dplot}
\usepackage{tikz-cd}
\usetikzlibrary{shapes.geometric}
\usetikzlibrary{arrows}
\usetikzlibrary{decorations.pathmorphing,patterns}
\usetkzobj{all}
\pgfplotsset{compat=1.13} % prevents compile error.

\renewcommand{\vec}[1]{\mathbf{#1}}
\newcommand{\RR}{\mathbb{R}}
\newcommand{\dfn}{\textit}
\newcommand{\dotp}{\cdot}
\newcommand{\id}{\text{id}}
\newcommand\norm[1]{\left\lVert#1\right\rVert}
 
\newtheorem{general}{Generalization}
\newtheorem{initprob}{Exploration Problem}

\tikzstyle geometryDiagrams=[ultra thick,color=blue!50!black]

\usepackage{mathtools}

\title{Homogeneous Linear Equations}%\label{Module 7-ADEF}


\begin{document}

\begin{abstract}
We develop a technique for solving homogeneous linear differential equations.
\end{abstract}

\maketitle

\section*{Homogeneous Linear Equations}

A second order differential equation is said to be \textit{linear} if
it can be written as
\begin{equation}\label{eq:5.1.1}
y''+p(x)y'+q(x)y=f(x).
\end{equation}
We call the function $f$ on the right a \textit{forcing function},
since in physical applications it's often related to a force acting
on some system modeled by the differential equation. We say that
\eqref{eq:5.1.1} is \textit{homogeneous} if $f\equiv0$ or \textit{nonhomogeneous} if $f\not\equiv0$. Since these definitions are like
the corresponding definitions in Module \ref{Module 1-B} for the linear
first order equation
\begin{equation}\label{eq:5.1.2}
y'+p(x)y=f(x),
\end{equation}
it's natural to expect similarities between methods of solving
\eqref{eq:5.1.1} and \eqref{eq:5.1.2}. However, solving \eqref{eq:5.1.1} is more
difficult than solving \eqref{eq:5.1.2}. For example, while
Theorem~\ref{thmtype:2.1.1} gives a formula for the general solution of
\eqref{eq:5.1.2} in the case where $f\equiv0$ and
Theorem~\ref{thmtype:2.1.2} gives
a formula for the case where $f\not\equiv0$, there are no formulas for
the general solution of \eqref{eq:5.1.1} in either case. Therefore we must
be content to solve linear second order equations of special forms.

In Module \ref{Module 1-B} %Section~2.1 
we considered the homogeneous equation
$y'+p(x)y=0$ first, and then used a nontrivial solution of this
equation to find the general solution of the nonhomogeneous equation
$y'+p(x)y=f(x)$. Although the progression from the homogeneous to the
nonhomogeneous case isn't  that simple for the linear second order
equation, it's still necessary to solve the homogeneous equation
\begin{equation}\label{eq:5.1.3}
y''+p(x)y'+q(x)y=0
\end{equation}
in order to solve the nonhomogeneous equation \eqref{eq:5.1.1}. This
section is devoted to \eqref{eq:5.1.3}.

The next theorem gives sufficient conditions for existence and
uniqueness of solutions of initial value problems for \eqref{eq:5.1.3}. We
omit the proof.

\begin{theorem} \label{thmtype:5.1.1}
Suppose $p$ and $q$ are continuous on an open interval $(a,b),$
let $x_0$ be any point in $(a,b),$ and let $k_0$ and $k_1$ be
arbitrary real numbers. Then the initial value problem
$$
y''+p(x)y'+q(x)y=0,\ y(x_0)=k_0,\ y'(x_0)=k_1
$$
 has a unique solution  on $(a,b)$.
\end{theorem}

Since $y\equiv0$ is obviously a solution of \eqref{eq:5.1.3} we call it
the \textit{trivial} solution. Any other solution is \textit{nontrivial}.
Under the assumptions of Theorem~\ref{thmtype:5.1.1}, the only solution of
the initial value problem
$$
y''+p(x)y'+q(x)y=0,\ y(x_0)=0,\ y'(x_0)=0
$$
on  $(a,b)$ is the trivial solution.
%(Exercise~\ref{exer:5.1.24}).

The next three examples illustrate concepts that we'll develop later
in this section. You shouldn't be concerned with how to \textit{find}
the given solutions of the equations in these examples. This will be
explained in later sections.

\begin{example}\label{example:5.1.1}  
The coefficients of $y'$ and $y$ in
\begin{equation}\label{eq:5.1.4}
y''-y=0
\end{equation}
are the constant functions $p\equiv0$ and $q\equiv-1$, which are
continuous on $(-\infty,\infty)$. Therefore Theorem~\ref{thmtype:5.1.1}
implies that every initial value problem for \eqref{eq:5.1.4}  has a
unique solution on $(-\infty,\infty)$.

\begin{enumerate}
\item \label{item:5.1.1a} % (a)
Verify that $y_1=e^x$ and $y_2=e^{-x}$ are solutions of
\eqref{eq:5.1.4} on $(-\infty,\infty)$.
\item \label{item:5.1.1b}% (b)
Verify that if $c_1$ and $c_2$ are arbitrary constants,
$y=c_1e^x+c_2e^{-x}$ is a solution of \eqref{eq:5.1.4} on
$(-\infty,\infty)$.
\item \label{item:5.1.1c}% (c)
Solve the initial value problem
\begin{equation}\label{eq:5.1.5}
y''-y=0,\quad y(0)=1,\quad y'(0)=3.22
\end{equation}
\end{enumerate}
\begin{explanation}

\ref{item:5.1.1a} If $y_1=e^x$ then $y_1'=e^x$ and $y_1''=e^x=y_1$,
so  $y_1''-y_1=0$.
If $y_2=e^{-x}$, then $y_2'=-e^{-x}$ and $y_2''=e^{-x}=y_2$,
so  $y_2''-y_2=0$.

\ref{item:5.1.1b} If
\begin{equation}\label{eq:5.1.6}
y=c_1e^x+c_2e^{-x}
\end{equation}
 then
\begin{equation}\label{eq:5.1.7}
y'=c_1e^x-c_2e^{-x}
\end{equation}
and
$$
y''=c_1e^x+c_2e^{-x},
$$
 so
\begin{eqnarray*}
y''-y&=&(c_1e^x+c_2e^{-x})-(c_1e^x+c_2e^{-x})\\
&=&c_1(e^x-e^x)+c_2(e^{-x}-e^{-x})=0
\end{eqnarray*}
for all $x$. Therefore $y=c_1e^x+c_2e^{-x}$ is a solution of
\eqref{eq:5.1.4} on $(-\infty,\infty)$.

\ref{item:5.1.1c}
We can solve \eqref{eq:5.1.5} by choosing $c_1$ and $c_2$ in \eqref{eq:5.1.6}
so that $y(0)=1$ and $y'(0)=3$. Setting $x=0$ in \eqref{eq:5.1.6} and
\eqref{eq:5.1.7} shows that this is equivalent to
\begin{eqnarray*}
c_1+c_2&=&1\\
c_1-c_2&=&3.
\end{eqnarray*}
Solving these equations yields  $c_1=2$  and $c_2=-1$.
Therefore $y=2e^x-e^{-x}$ is the unique solution of
\eqref{eq:5.1.5} on $(-\infty,\infty)$.
\end{explanation}
\end{example}

\begin{example}\label{example:5.1.2}  %\newcommand{\exampletrig}
Let $\omega$ be a positive constant. The coefficients of $y'$
and $y$  in
\begin{equation}\label{eq:5.1.8}
y''+\omega^2y=0
\end{equation}
are the constant functions $p\equiv0$ and $q\equiv\omega^2$,
which are continuous on $(-\infty,\infty)$. Therefore
Theorem~\ref{thmtype:5.1.1}
implies that every initial value problem for \eqref{eq:5.1.8}  has a
unique solution on $(-\infty,\infty)$.

\begin{enumerate}
    \item \label{item:5.1.2a}  %(a)
Verify that $y_1=\cos\omega x$ and $y_2=\sin\omega x$ are
solutions of \eqref{eq:5.1.8} on $(-\infty,\infty)$.
\item \label{item:5.1.2b}  % (b)
Verify that if $c_1$ and $c_2$ are arbitrary constants then
$y=c_1\cos\omega x+c_2\sin\omega x$ is a solution of \eqref{eq:5.1.8}
on $(-\infty,\infty)$.
\item \label{item:5.1.2c} % (c)
Solve the initial value problem
\begin{equation}\label{eq:5.1.9}
y''+\omega^2y=0,\quad y(0)=1,\quad y'(0)=3.
\end{equation}
\end{enumerate}
\begin{explanation}
\ref{item:5.1.2a} If $y_1=\cos\omega x$ then $y_1'=-\omega\sin\omega x$
and
$y_1''=-\omega^2\cos\omega x=-\omega^2y_1$, so  $y_1''+\omega^2y_1=0$.
If $y_2=\sin\omega x$ then, $y_2'=\omega\cos\omega x$ and
$y_2''=-\omega^2\sin\omega x=-\omega^2y_2$, so  $y_2''+\omega^2y_2=0$.

\ref{item:5.1.2b} If
\begin{equation}\label{eq:5.1.10}
y=c_1\cos\omega x+c_2\sin\omega x
\end{equation}
 then
\begin{equation}\label{eq:5.1.11}
y'=\omega(-c_1\sin\omega x+c_2\cos\omega x)
\end{equation}
and
$$
y''=-\omega^2(c_1\cos\omega x+c_2\sin\omega x),
$$
so
\begin{eqnarray*}
y''+\omega^2y&=& -\omega^2(c_1\cos\omega x+c_2\sin\omega x)
+\omega^2(c_1\cos\omega x+c_2\sin\omega x)\\
&=&c_1\omega^2(-\cos\omega x+\cos\omega x)+
c_2\omega^2(-\sin\omega x+\sin\omega x)=0
\end{eqnarray*}
for all $x$. Therefore $y=c_1\cos\omega x+c_2\sin\omega x$ is a
solution of \eqref{eq:5.1.8} on $(-\infty,\infty)$.

\ref{item:5.1.2c}
To solve
\eqref{eq:5.1.9}, we must choosing $c_1$ and $c_2$ in \eqref{eq:5.1.10}
so that $y(0)=1$ and $y'(0)=3$. Setting $x=0$ in \eqref{eq:5.1.10}
and \eqref{eq:5.1.11} shows that $c_1=1$ and $c_2=3/\omega$.
Therefore
$$
y=\cos\omega x+\frac{3}{\omega}\sin\omega x
$$
 is the unique solution of \eqref{eq:5.1.9} on
$(-\infty,\infty)$.
\end{explanation}
\end{example}

Theorem~\ref{thmtype:5.1.1} implies that if $k_0$ and $k_1$ are
arbitrary real numbers then the initial value problem
\begin{equation}\label{eq:5.1.12}
P_0(x)y''+P_1(x)y'+P_2(x)y=0,\quad y(x_0)=k_0,\quad y'(x_0)=k_1
\end{equation}
has a unique solution on an interval $(a,b)$ that contains $x_0$,
provided that $P_0$, $P_1$, and $P_2$ are continuous and $P_0$
has no zeros on $(a,b)$. To see this, we rewrite the differential
equation in  \eqref{eq:5.1.12} as
$$
y''+\frac{P_1(x)}{P_0(x)}y'+\frac{P_2(x)}{P_0(x)}y=0
$$
and apply Theorem~\ref{thmtype:5.1.1} with $p=P_1/P_0$ and $q=P_2/P_0$.

\begin{example}\label{example:5.1.3}  
The equation
\begin{equation}\label{eq:5.1.13}
x^2y''+xy'-4y=0
\end{equation}
has the form of the differential equation in \eqref{eq:5.1.12}, with
$P_0(x)=x^2$, $P_1(x)=x$, and $P_2(x)=-4$, which are are all
continuous on $(-\infty,\infty)$. However, since $P(0)=0$ we must
consider solutions of \eqref{eq:5.1.13} on $(-\infty,0)$ and $(0,\infty)$.
Since $P_0$ has no zeros on these intervals, Theorem~\ref{thmtype:5.1.1}
implies that the initial value problem
$$
x^2y''+xy'-4y=0,\quad y(x_0)=k_0,\quad y'(x_0)=k_1
$$
has a unique solution on $(0,\infty)$ if $x_0>0$, or on $(-\infty,0)$
if $x_0<0$.

\begin{enumerate}
\item\label{item:5.1.3a}% (a)
Verify that $y_1=x^2$ is a solution of \eqref{eq:5.1.13} on
$(-\infty,\infty)$ and $y_2=1/x^2$ is a solution of \eqref{eq:5.1.13}
on $(-\infty,0)$  and $(0,\infty)$.
\item\label{item:5.1.3b}% (b)
Verify that if $c_1$ and $c_2$ are any constants then
$y=c_1x^2+c_2/x^2$ is a solution of \eqref{eq:5.1.13} on $(-\infty,0)$
and $(0,\infty)$.
\item\label{item:5.1.3c}% (c)
Solve the initial value problem
\begin{equation}\label{eq:5.1.14}
x^2y''+xy'-4y=0,\quad y(1)=2,\quad y'(1)=0.
\end{equation}
\item\label{item:5.1.3d}% (d)
Solve the initial value problem
\begin{equation}\label{eq:5.1.15}
x^2y''+xy'-4y=0,\quad y(-1)=2,\quad y'(-1)=0.
\end{equation}
\end{enumerate}
\begin{explanation}
\ref{item:5.1.3a} If $y_1=x^2$ then $y_1'=2x$ and $y_1''=2$, so
$$
x^2y_1''+xy_1'-4y_1=x^2(2)+x(2x)-4x^2=0
$$
for $x$ in $(-\infty,\infty)$.
If $y_2=1/x^2$, then $y_2'=-2/x^3$ and $y_2''=6/x^4$, so
$$
x^2y_2''+xy_2'-4y_2=x^2\left(\frac{6}{x^4}\right)-x\left(\frac{2}{x^3}\right)-{\frac{4}{x^2}}=0
$$
for $x$ in $(-\infty,0)$ or $(0,\infty)$.

\ref{item:5.1.3b} If
\begin{equation}\label{eq:5.1.16}
y=c_1x^2+\frac{c_2}{x^2}
\end{equation}
 then
\begin{equation}\label{eq:5.1.17}
y'=2c_1x-\frac{2c_2}{x^3}
\end{equation}
and
$$
y''=2c_1+\frac{6c_2}{x^4},
$$
so
\begin{eqnarray*}
x^2y''+xy'-4y&=&x^2\left(2c_1+\frac{6c_2}{x^4}\right)
+x\left(2c_1x-\frac{2c_2}{x^3}\right)
-4\left(c_1x^2+\frac{c_2}{x^2}\right)\\
&=&c_1(2x^2+2x^2-4x^2)
+c_2\left(\frac{6}{x^2}-\frac{2}{x^2}-\frac{4}{x^2}\right)
\\
&=&c_1\cdot0+c_2\cdot0=0
\end{eqnarray*}
for $x$ in $(-\infty,0)$ or $(0,\infty)$.

\ref{item:5.1.3c}
To solve
\eqref{eq:5.1.14}, we choose $c_1$ and $c_2$ in \eqref{eq:5.1.16}
so that $y(1)=2$ and $y'(1)=0$. Setting $x=1$ in \eqref{eq:5.1.16}
and \eqref{eq:5.1.17} shows that this is equivalent to
\begin{eqnarray*}
c_1+c_2&=&2\\
2c_1-2c_2&=&0.
\end{eqnarray*}
Solving these equations yields  $c_1=1$  and $c_2=1$.
Therefore $y=x^2+1/x^2$ is the unique  solution of \eqref{eq:5.1.14}
on $(0,\infty)$.

\ref{item:5.1.3d}
We can solve
\eqref{eq:5.1.15} by choosing $c_1$ and $c_2$ in \eqref{eq:5.1.16}
so that $y(-1)=2$ and $y'(-1)=0$. Setting $x=-1$ in \eqref{eq:5.1.16}
and \eqref{eq:5.1.17} shows that this is equivalent to
\begin{eqnarray*}
c_1+c_2&=&2\\
-2c_1+2c_2&=&0.
\end{eqnarray*}
Solving these equations yields  $c_1=1$  and $c_2=1$.
Therefore $y=x^2+1/x^2$ is the unique solution of \eqref{eq:5.1.15}
on $(-\infty,0)$.

\end{explanation}
\end{example}



Although the \textit{formulas} for the solutions of \eqref{eq:5.1.14} and
\eqref{eq:5.1.15} are both $y=x^2+1/x^2$, you should not conclude that
these two initial value problems have the same solution. Remember that
a solution of an initial value problem is defined \textit{on an interval
that contains the initial point};  therefore, the solution of
\eqref{eq:5.1.14} is $y=x^2+1/x^2$ \textit{on the interval} $(0,\infty)$,
which contains the initial point $x_0=1$, while the solution of
\eqref{eq:5.1.15} is $y=x^2+1/x^2$ \textit{on the interval} $(-\infty,0)$,
which contains the initial point $x_0=-1$.

\subsection*{The General Solution of a Homogeneous Linear Second Order
Equation}

If $y_1$ and $y_2$ are defined on an interval
$(a,b)$ and $c_1$ and $c_2$ are constants, then
$$
y=c_1y_1+c_2y_2
$$
is a \textit{linear combination of $y_1$ and $y_2$}. For
example, $y=2\cos x+7 \sin x$ is a linear combination of $y_1=
\cos x$ and $y_2=\sin x$, with $c_1=2$ and $c_2=7$.

The next theorem states a fact that we've already verified in
Examples~\ref{example:5.1.1}, \ref{example:5.1.2}, and \ref{example:5.1.3}.

\begin{theorem}\label{thmtype:5.1.2}
If $y_1$ and $y_2$ are solutions of the homogeneous equation
\begin{equation}\label{eq:5.1.18}
y''+p(x)y'+q(x)y=0
\end{equation}
on $(a,b),$ then any linear combination
\begin{equation}\label{eq:5.1.19}
y=c_1y_1+c_2y_2
\end{equation}
of $y_1$ and $y_2$ is also a solution of $\eqref{eq:5.1.18}$ on $(a,b).$
\end{theorem}

\begin{proof}
 If
$$
y=c_1y_1+c_2y_2
$$
 then
$$
y'=c_1y_1'+c_2y_2'\quad\mbox{and}\quad y''=c_1y_1''+c_2y_2''.
$$
Therefore
\begin{eqnarray*}
y''+p(x)y'+q(x)y&=&(c_1y_1''+c_2y_2'')+p(x)(c_1y_1'+c_2y_2')
+q(x)(c_1y_1+c_2y_2)\\[2\jot]
&=&c_1\left(y_1''+p(x)y_1'+q(x)y_1\right)
+c_2\left(y_2''+p(x)y_2'+q(x)y_2\right)\\[2\jot]
&=&c_1\cdot0+c_2\cdot0=0,
\end{eqnarray*}
since $y_1$ and $y_2$ are solutions of \eqref{eq:5.1.18}.  
\end{proof}
We say that  $\{y_1,y_2\}$ is a \textit{fundamental set of
solutions of $\eqref{eq:5.1.18}$ on}  $(a,b)$ if every solution
of \eqref{eq:5.1.18} on  $(a,b)$ can be written as a linear
combination of $y_1$ and $y_2$ as in \eqref{eq:5.1.19}.
In this case we say that \eqref{eq:5.1.19} is
\textit{general solution of $\eqref{eq:5.1.18}$ on}  $(a,b)$.

\subsection*{Linear Independence}

We need a way to determine whether a given set $\{y_1,y_2\}$
of solutions of \eqref{eq:5.1.18}  is a fundamental set.
The next definition will enable us to state necessary and
sufficient conditions for this.

We say that two functions $y_1$ and $y_2$ defined on an interval
$(a,b)$ are \textit{linearly independent on} $(a,b)$ if neither is a
constant multiple of the other on $(a,b)$. (In
particular, this means
that neither can be the trivial solution of \eqref{eq:5.1.18}, since, for
example, if $y_1\equiv0$  we could write $y_1=0y_2$.) We'll also
say that the set $\{y_1,y_2\}$ \textit{is linearly independent on}
$(a,b)$.

 \begin{theorem}\label{thmtype:5.1.3}
Suppose $p$ and $q$ are continuous on $(a,b).$ Then a set
$\{y_1,y_2\}$ of solutions of
\begin{equation}\label{eq:5.1.20}
y''+p(x)y'+q(x)y=0
\end{equation}
on $(a,b)$ is a fundamental set if and only if $\{y_1,y_2\}$ is
linearly independent on $(a,b).$
\end{theorem}

We'll present the proof of Theorem~\ref{thmtype:5.1.3} in steps worth
regarding as theorems in their own right. However, let's first
interpret Theorem~\ref{thmtype:5.1.3} in terms of
Examples~\ref{example:5.1.1}, \ref{example:5.1.2}, and \ref{example:5.1.3}.

\begin{example}\label{example:5.1.4}   
\begin{enumerate}
\item\label{item:5.1.4a}% (a)
Since $e^x/e^{-x}=e^{2x}$ is nonconstant, Theorem~\ref{thmtype:5.1.3}
implies that $y=c_1e^x+c_2e^{-x}$ is the general solution of $y''-y=0$
on $(-\infty,\infty)$.
\item\label{item:5.1.4b}% (b)
Since $\cos\omega x/\sin\omega x=\cot\omega x$ is nonconstant,
Theorem~\ref{thmtype:5.1.3} implies that
 $y=c_1\cos\omega x+c_2\sin\omega x$ is the general solution
of $y''+\omega^2y=0$ on $(-\infty,\infty)$.
\item\label{item:5.1.4c} % (c)
Since $x^2/x^{-2}=x^4$ is nonconstant,
Theorem~\ref{thmtype:5.1.3} implies that
 $y=c_1x^2+c_2/x^2$ is the general
solution of $x^2y''+xy'-4y=0$ on $(-\infty,0)$ and $(0,\infty)$.
\end{enumerate}
\end{example}


\subsection*{The Wronskian and Abel's Formula}

To motivate a result that we need in order to prove
Theorem~\ref{thmtype:5.1.3},
 let's see what is required to prove that  $\{y_1,y_2\}$
is a  fundamental set of solutions of \eqref{eq:5.1.20} on  $(a,b)$.
Let $x_0$ be an arbitrary point in  $(a,b)$, and
suppose  $y$ is an arbitrary  solution of
\eqref{eq:5.1.20} on  $(a,b)$.  Then $y$ is the unique solution of the
initial value problem
\begin{equation}\label{eq:5.1.21}
y''+p(x)y'+q(x)y=0,\quad y(x_0)=k_0,\quad y'(x_0)=k_1;
\end{equation}
that is, $k_0$ and $k_1$ are the numbers obtained by evaluating $y$
and $y'$ at $x_0$. Moreover, $k_0$ and $k_1$ can be any real numbers,
since Theorem~\ref{thmtype:5.1.1} implies that \eqref{eq:5.1.21} has a
solution no matter how $k_0$ and $k_1$ are chosen.
Therefore  $\{y_1,y_2\}$ is a fundamental set of solutions of
\eqref{eq:5.1.20} on $(a,b)$ if and only if it's possible to write the
solution of an arbitrary initial value problem \eqref{eq:5.1.21}
as $y=c_1y_1+c_2y_2$. This is equivalent to requiring that the
system
\begin{equation}\label{eq:5.1.22}
\begin{array}{rcl}
c_1y_1(x_0)+c_2y_2(x_0)&=&k_0\\
c_1y_1'(x_0)+c_2y_2'(x_0)&=&k_1
\end{array}
\end{equation}
has  a solution $(c_1,c_2)$ for every choice of $(k_0,k_1)$.
Let's try to solve \eqref{eq:5.1.22}.

Multiplying the first equation in \eqref{eq:5.1.22} by $y_2'(x_0)$
and the second by $y_2(x_0)$ yields
\begin{eqnarray*}
c_1y_1(x_0)y_2'(x_0)+c_2y_2(x_0)y_2'(x_0)&=& y_2'(x_0)k_0\\
c_1y_1'(x_0)y_2(x_0)+c_2y_2'(x_0)y_2(x_0)&=& y_2(x_0)k_1,
\end{eqnarray*}
and subtracting the second equation here from the first yields
\begin{equation}\label{eq:5.1.23}
\left(y_1(x_0)y_2'(x_0)-y_1'(x_0)y_2(x_0)\right)c_1=
y_2'(x_0)k_0-y_2(x_0)k_1.
\end{equation}
Multiplying the first equation in \eqref{eq:5.1.22} by $y_1'(x_0)$
and the second by $y_1(x_0)$ yields
\begin{eqnarray*}
c_1y_1(x_0)y_1'(x_0)+c_2y_2(x_0)y_1'(x_0)&=& y_1'(x_0)k_0\\
c_1y_1'(x_0)y_1(x_0)+c_2y_2'(x_0)y_1(x_0)&=& y_1(x_0)k_1,
\end{eqnarray*}
and subtracting the first equation here from the second yields
\begin{equation}\label{eq:5.1.24}
\left(y_1(x_0)y_2'(x_0)-y_1'(x_0)y_2(x_0)\right)c_2=
y_1(x_0)k_1-y_1'(x_0)k_0.
\end{equation}
If
$$
y_1(x_0)y_2'(x_0)-y_1'(x_0)y_2(x_0)=0,
$$
 it's impossible to satisfy \eqref{eq:5.1.23} and \eqref{eq:5.1.24}
(and therefore \eqref{eq:5.1.22})
unless $k_0$ and $k_1$ happen to satisfy
\begin{eqnarray*}
y_1(x_0)k_1-y_1'(x_0)k_0&=&0\\
y_2'(x_0)k_0-y_2(x_0)k_1&=&0.
\end{eqnarray*}
On the other hand, if
\begin{equation}\label{eq:5.1.25}
y_1(x_0)y_2'(x_0)-y_1'(x_0)y_2(x_0)\ne0
\end{equation}
 we can  divide \eqref{eq:5.1.23} and \eqref{eq:5.1.24} through by the
quantity on the left  to obtain
\begin{equation}\label{eq:5.1.26}
\begin{array}{rcl}
c_1&=&\frac{y_2'(x_0)k_0-y_2(x_0)k_1}
{y_1(x_0)y_2'(x_0)-y_1'(x_0)y_2(x_0)}\\[4\jot]
c_2&=&\frac{y_1(x_0)k_1-y_1'(x_0)k_0}
{y_1(x_0)y_2'(x_0)-y_1'(x_0)y_2(x_0)},
\end{array}
\end{equation}
no matter how $k_0$ and $k_1$ are chosen. This motivates us to
consider conditions on $y_1$ and $y_2$ that imply \eqref{eq:5.1.25}.

\begin{theorem}\label{thmtype:5.1.4}
Suppose $p$ and $q$ are continuous on $(a,b),$ let $y_1$ and
$y_2$ be solutions of
\begin{equation}\label{eq:5.1.27}
y''+p(x)y'+q(x)y=0
\end{equation}
on $(a,b)$, and define
\begin{equation}\label{eq:5.1.28}
W=y_1y_2'-y_1'y_2.
\end{equation}
Let  $x_0$ be any point in $(a,b).$ Then
\begin{equation} \label{eq:5.1.29}
W(x)=W(x_0) e^{-\int^x_{x_0}p(t)\,
dt}, \quad a<x<b.
\end{equation}
Therefore  either $W$ has no zeros in  $(a,b)$ or $W\equiv0$
on  $(a,b).$
\end{theorem}



\begin{proof}
Differentiating \eqref{eq:5.1.28} yields
\begin{equation}\label{eq:5.1.30}
W'=y'_1y'_2+y_1y''_2-y'_1y'_2-y''_1y_2=
y_1y''_2-y''_1y_2.
\end{equation}
 Since $y_1$ and $y_2$ both satisfy \eqref{eq:5.1.27},
$$
y''_1 =-py'_1-qy_1\quad\mbox{and}\quad
y''_2 =-py'_2-qy_2.
$$
Substituting these into \eqref{eq:5.1.30} yields
\begin{eqnarray*}
W'&=& -y_1\bigl(py'_2+qy_2\bigr)
+y_2\bigl(py'_1+qy_1\bigr) \\
&=&  -p(y_1y'_2-y_2y'_1)-q(y_1y_2-y_2y_1)\\
&=& -p(y_1y'_2-y_2y'_1)=-pW.
\end{eqnarray*}
Therefore $W'+p(x)W=0$;
that is, $W$ is the solution of the initial value problem
$$
y'+p(x)y=0,\quad y(x_0)=W(x_0).
$$
We leave it to you to verify by separation of variables that this
implies \eqref{eq:5.1.29}. If $W(x_0)\ne0$, \eqref{eq:5.1.29} implies
that
$W$ has no zeros in $(a,b)$, since an exponential is never zero. On
the other hand, if $W(x_0)=0$,  \eqref{eq:5.1.29} implies that $W(x)=0$
for all $x$ in $(a,b)$. 
\end{proof}

The function $W$ defined in \eqref{eq:5.1.28} is  the
\href{http://www-history.mcs.st-and.ac.uk/Mathematicians/Wronski.html}
{Wronskian of $\{y_1,y_2\}$}.
 Formula \eqref{eq:5.1.29} is
 \href{http://www-history.mcs.st-and.ac.uk/Mathematicians/Abel.html}
{Abel's formula}.

The Wronskian of  $\{y_1,y_2\}$ is usually written as the determinant
$$
W=\left| \begin{array}{cc}
y_1 & y_2 \\
y'_1 & y'_2
\end{array} \right|.
$$
 The expressions in \eqref{eq:5.1.26} for $c_1$ and $c_2$ can
be written in terms of determinants as
$$
c_1=\frac{1}{W(x_0)}
\left| \begin{array}{cc}
k_0 & y_2(x_0) \\
k_1 & y'_2(x_0)
\end{array} \right|
\quad\mbox{and}\quad
c_2=\frac{1}{W(x_0)}
\left| \begin{array}{cc}
y_1(x_0) & k_0 \\
y'_1(x_0) &k_1
\end{array} \right|.
$$
If you've taken linear algebra you may recognize this as
\href{http://www-history.mcs.st-and.ac.uk/Mathematicians/Cramer.html}
{\color{blue}\it Cramer's rule}.

\begin{example}\label{example:5.1.5}
Verify Abel's formula for the following differential equations and the
corresponding solutions, from  Examples~\ref{example:5.1.1},
\ref{example:5.1.2}, and \ref{example:5.1.3}:
\begin{enumerate}
\item\label{item:5.1.5a} % (a)
$y''-y=0;\quad  y_1=e^x,\;  y_2=e^{-x}$
\item\label{item:5.1.5b}% (b)
$y''+\omega^2y=0;\quad \quad y_1=\cos\omega x,\;  y_2=\sin\omega x$
\item\label{item:5.1.5c}% (c)
$x^2y''+xy'-4y=0;\quad  y_1=x^2,\;  y_2=1/x^2$
\end{enumerate}
\begin{explanation}
\ref{item:5.1.5a} Since $p\equiv0$, we can verify Abel's formula
by showing that $W$ is constant, which is true, since
$$
W(x)=\left| \begin{array}{rr}
e^x & e^{-x} \\
e^x & -e^{-x}
\end{array} \right|=e^x(-e^{-x})-e^xe^{-x}=-2
$$
for all $x$.

\ref{item:5.1.5b} Again, since $p\equiv0$, we can verify Abel's
formula
by showing that $W$ is constant, which is true, since
\begin{eqnarray*}
W(x)&=&\left| \begin{array}{cc}
\cos\omega x & \sin\omega x \\
-\omega\sin\omega x &\omega\cos\omega x
\end{array} \right|\\
&=&\cos\omega x (\omega\cos\omega x)-(-\omega\sin\omega x)\sin\omega
x\\ &=&\omega(\cos^2\omega x+\sin^2\omega x)=\omega
\end{eqnarray*}
for all $x$.

\ref{item:5.1.5c}
Computing the Wronskian of $y_1=x^2$ and
$y_2=1/x^2$ directly yields
\begin{equation}\label{eq:5.1.31}
W=\left| \begin{array}{cc}
x^2 & 1/x^2 \\
2x & -2/x^3
\end{array} \right|=x^2\left(-\frac{2}{x^3}\right)-2x\left(\frac{1}{x^2}\right)
=-\frac{4}{x}.
\end{equation}
To verify Abel's formula we
rewrite the differential  equation as
$$
y''+\frac{1}{x}y'-\frac{4}{x^2}y=0
$$
to see  that $p(x)=1/x$. If $x_0$ and $x$ are either both in
$(-\infty,0)$ or both in $(0,\infty)$ then
$$
\int_{x_0}^x p(t)\,dt=\int_{x_0}^x \frac{dt}{t}=\ln\left(\frac{x}{x_0}
\right),
$$
so Abel's formula becomes
\begin{eqnarray*}
W(x)&=&W(x_0)e^{-\ln(x/x_0)}=W(x_0)\frac{x_0}{x}\\
&=&-\left(\frac{4}{x_0}\right)\left(\frac{x_0}{x}\right)\quad\mbox{from
\eqref{eq:5.1.31}}\\
&=&-\frac{4}{x},
\end{eqnarray*}
which is consistent with \eqref{eq:5.1.31}.
\end{explanation}
\end{example}



The next theorem will enable us to complete the proof of
Theorem~\ref{thmtype:5.1.3}.

\begin{theorem}\label{thmtype:5.1.5}
 Suppose  $p$ and $q$ are continuous on an open interval $(a,b),$
 let $y_1$ and $y_2$ be solutions of
\begin{equation}\label{eq:5.1.32}
y''+p(x)y'+q(x)y=0
\end{equation}
on  $(a,b),$ and let $W=y_1y_2'-y_1'y_2.$
Then $y_1$ and $y_2$ are linearly independent on  $(a,b)$ if and only
if $W$ has no zeros on  $(a,b).$
\end{theorem}

\begin{proof}
We first show that if $W(x_0)=0$ for some $x_0$ in  $(a,b)$, then
$y_1$ and $y_2$ are linearly dependent on  $(a,b)$.  Let $I$ be a
subinterval
of  $(a,b)$ on which $y_1$ has no zeros. (If there's no such
subinterval, $y_1\equiv0$ on  $(a,b)$, so $y_1$ and $y_2$
are linearly independent, and we're finished with this part of the
proof.) Then $y_2/y_1$ is defined on $I$, and
\begin{equation}\label{eq:5.1.33}
\left(\frac{y_2}{y_1}\right)'=\frac{y_1y_2'-y_1'y_2}{y_1^2}=\frac{W}{y_1^2}.
\end{equation}
However, if $W(x_0)=0$,
 Theorem~\ref{thmtype:5.1.4} implies that $W\equiv0$ on  $(a,b)$.
Therefore \eqref{eq:5.1.33} implies that  $(y_2/y_1)'\equiv0$, so
$y_2/y_1=c$ (constant) on $I$. This shows that $y_2(x)=cy_1(x)$
for all $x$ in $I$. However, we want to show that $y_2=cy_1(x)$
for all $x$ in  $(a,b)$.
Let $Y=y_2-cy_1$. Then $Y$ is a solution of \eqref{eq:5.1.32}
on  $(a,b)$ such that $Y\equiv0$ on $I$, and therefore $Y'\equiv0$ on
$I$. Consequently, if $x_0$  is chosen arbitrarily in  $I$ then
$Y$ is a solution of the initial value problem
$$
y''+p(x)y'+q(x)y=0,\quad y(x_0)=0,\quad y'(x_0)=0,
$$
which implies that $Y\equiv0$ on  $(a,b)$,
by the paragraph following
Theorem~\ref{thmtype:5.1.1}. %(See also Exercise~\ref{exer:5.1.24}).
 Hence, $y_2-cy_1\equiv0$
on  $(a,b)$, which implies that $y_1$ and $y_2$ are not linearly
independent on  $(a,b)$.

Now suppose   $W$ has no zeros on  $(a,b)$.  Then $y_1$
can't be identically zero on  $(a,b)$ (why not?), and therefore there
is a subinterval $I$ of  $(a,b)$ on which $y_1$ has no zeros. Since
\eqref{eq:5.1.33} implies that $y_2/y_1$ is nonconstant on $I$,
$y_2$ isn't  a constant multiple of $y_1$ on  $(a,b)$.
A similar argument shows that $y_1$ isn't  a constant multiple of
$y_2$ on  $(a,b)$, since
$$
\left(\frac{y_1}{y_2}\right)'=\frac{y_1'y_2-y_1y_2'}{y_2^2}=-\frac{W}{y_2^2}
$$
on any subinterval of  $(a,b)$ where $y_2$ has no zeros.
\end{proof}
We can now complete the proof of Theorem~\ref{thmtype:5.1.3}.
\begin{proof}[Proof of Theorem~\ref{thmtype:5.1.3}:]
From Theorem~\ref{thmtype:5.1.5}, two solutions $y_1$ and $y_2$ of
\eqref{eq:5.1.32} are linearly
independent on  $(a,b)$ if and only if $W$ has no zeros on  $(a,b)$.
From Theorem~\ref{thmtype:5.1.4} and the motivating comments preceding
it,  $\{y_1,y_2\}$ is  a fundamental set of solutions of
\eqref{eq:5.1.32}  if and only if $W$ has no zeros on  $(a,b)$.
Therefore  $\{y_1,y_2\}$ is   a fundamental set
for \eqref{eq:5.1.32} on  $(a,b)$ if and only if  $\{y_1,y_2\}$
is linearly independent on $(a,b)$.  
\end{proof}
The next theorem summarizes the relationships among the
concepts discussed in this section.

\begin{theorem}\label{thmtype:5.1.6}
 Suppose  $p$ and $q$ are continuous on an open interval $(a,b)$
and  let $y_1$ and $y_2$ be solutions of
\begin{equation}\label{eq:5.1.34}
y''+p(x)y'+q(x)y=0
\end{equation}
on  $(a,b).$  Then the following statements are equivalent; that
is, they are either all true or all false.
\begin{enumerate}
\item % (a)
The general solution of $\eqref{eq:5.1.34}$ on  $(a,b)$ is
$y=c_1y_1+c_2y_2$.
\item % (b)
 $\{y_1,y_2\}$ is a fundamental set of solutions of $\eqref{eq:5.1.34}$
on  $(a,b).$
\item % (c)
 $\{y_1,y_2\}$ is linearly independent on  $(a,b)$.
\item % (d)
The Wronskian of   $\{y_1,y_2\}$ is nonzero at some point in  $(a,b)$.
\item % (d)
The Wronskian of   $\{y_1,y_2\}$ is nonzero at all points in  $(a,b)$.
\end{enumerate}
\end{theorem}

We can apply this theorem to an equation written as
$$
P_0(x)y''+P_1(x)y'+P_2(x)y=0
$$
on an interval $(a,b)$ where $P_0$, $P_1$, and $P_2$
are continuous and $P_0$ has no zeros.


\begin{theorem}\label{thmtype:5.1.7}
Suppose  $c$ is in $(a,b)$ and $\alpha$ and $\beta$  are real numbers,
not both zero.
Under the assumptions of Theorem~\ref{thmtype:5.1.7}, suppose
$y_{1}$ and $y_{2}$ are  solutions  of \eqref{eq:5.1.34}   such that
\begin{equation} \label{eq:5.1.35}
\alpha y_{1}(c)+\beta y_{1}'(c)=0\quad\text{and}\quad
\alpha y_{2}(c)+\beta y_{2}'(c)=0.
\end{equation}
Then $\{y_{1},y_{2}\}$ isn't  linearly independent on $(a,b)$.
\end{theorem}


\begin{proof} Since $\alpha$ and $\beta$ are not both zero,   \eqref{eq:5.1.35}
implies that
$$
\left|\begin{array}{ccccccc}
y_{1}(c)&y_{1}'(c)\\y_{2}(c)& y_{2}'(c)
\end{array}\right|=0, \quad\text{so}\quad
\left|\begin{array}{cccccc}
y_{1}(c)&y_{2}(c)\\ y_{1}'(c)&y_{2}'(c)
\end{array}\right|=0
$$
and Theorem~\ref{thmtype:5.1.6} implies the stated conclusion.
\end{proof}

\section*{Text Source}
Trench, William F., "Elementary Differential Equations" (2013). Faculty Authored and Edited Books \& CDs. 8. (CC-BY-NC-SA)

\href{https://digitalcommons.trinity.edu/mono/8/}{https://digitalcommons.trinity.edu/mono/8/}


\end{document}