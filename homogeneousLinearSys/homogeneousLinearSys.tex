\documentclass{ximera}

%% You can put user macros here
%% However, you cannot make new environments

\listfiles

\graphicspath{{./}{firstExample/}{secondExample/}}

\usepackage{tikz}
\usepackage{tkz-euclide}
\usepackage{tikz-3dplot}
\usepackage{tikz-cd}
\usetikzlibrary{shapes.geometric}
\usetikzlibrary{arrows}
\usetikzlibrary{decorations.pathmorphing,patterns}
\usetkzobj{all}
\pgfplotsset{compat=1.13} % prevents compile error.

\renewcommand{\vec}[1]{\mathbf{#1}}
\newcommand{\RR}{\mathbb{R}}
\newcommand{\dfn}{\textit}
\newcommand{\dotp}{\cdot}
\newcommand{\id}{\text{id}}
\newcommand\norm[1]{\left\lVert#1\right\rVert}
 
\newtheorem{general}{Generalization}
\newtheorem{initprob}{Exploration Problem}

\tikzstyle geometryDiagrams=[ultra thick,color=blue!50!black]

\usepackage{mathtools}

\title{Basic Theory of Homogeneous Linear System}%\label{Module 7-ADEF}


\begin{document}

\begin{abstract}

\end{abstract}

\maketitle

\section*{Basic Theory of Homogeneous Linear System}

In this section we consider homogeneous linear systems ${\bf y}'=
A(t){\bf y}$, where $A=A(t)$ is a continuous $n\times n$ matrix
function on an interval $(a,b)$. The theory of linear homogeneous
systems has much in common with the theory of linear homogeneous
scalar equations, which we considered in
Sections~2.1, 5.1, and 9.1.

Whenever we refer to solutions of ${\bf y}'=A(t){\bf y}$ we'll mean
solutions on $(a,b)$. Since ${\bf y}\equiv{\bf 0}$ is obviously a
solution of ${\bf y}'=A(t){\bf y}$, we call it the \dfn{trivial}
solution. Any other solution is \dfn{nontrivial}.

If ${\bf y}_1, {\bf y}_2, \dots, {\bf y}_n$ are vector functions
defined on an interval $(a,b)$ and $c_1, c_2, \dots, c_n$ are
constants, then
\begin{equation} \label{eq:10.3.1}
{\bf y}=c_1{\bf y}_1+c_2{\bf y}_2+\cdots+c_n{\bf y}_n
\end{equation}
is a \dfn{linear combination of} ${\bf y}_1, {\bf y}_2, \ldots,{\bf
y}_n$. It's easy show that if ${\bf
y}_1, {\bf y}_2, \dots,{\bf y}_n$ are solutions of ${\bf y}'=A(t){\bf
y}$ on $(a,b)$, then so is any linear combination of
 ${\bf y}_1, {\bf y}_2, \dots, {\bf y}_n$ %(Exercise~\ref{exer:10.3.1}). 
 We say that
$\{{\bf y}_1,{\bf y}_2,\dots,{\bf y}_n\}$ is a \dfn{fundamental set of
solutions of ${\bf y}'=A(t){\bf y}$ on} $(a,b)$ on if every solution of
${\bf y}'=A(t){\bf y}$ on $(a,b)$ can be written as a linear combination of
${\bf y}_1, {\bf y}_2, \dots, {\bf y}_n$, as in \eqref{eq:10.3.1}.
In this
case we say that \eqref{eq:10.3.1} is the \dfn{general solution of ${\bf
y}'=A(t){\bf y}$ on} $(a,b)$.



It can be shown that if $A$ is continuous on $(a,b)$ then ${\bf
y}'=A(t){\bf y}$ has infinitely many fundamental sets of solutions on
$(a,b)$ %(Exercises~\ref{exer:10.3.15} and \ref{exer:10.3.16}). 
The next
definition will help to characterize fundamental sets of solutions of
${\bf y}'=A(t){\bf y}$.


We say that a set $\{{\bf y}_1,{\bf y}_2,\dots,{\bf y}_n\}$ of
$n$-vector functions is \dfn{linearly independent} on $(a,b)$ if the
only constants $c_1, c_2, \dots, c_n$ such that
\begin{equation} \label{eq:10.3.2}
 c_1{\bf y}_1(t)+c_2{\bf y}_2(t)+\cdots+c_n{\bf y}_n(t)=0,\quad
a<t<b,
\end{equation}
are $c_1=c_2=\cdots=c_n=0$. If \eqref{eq:10.3.2} holds for some set of
constants $c_1, c_2, \dots, c_n$ that are not all zero, then $\{{\bf
y}_1,{\bf y}_2,\dots,{\bf y}_n\}$ is \dfn{linearly dependent} on
$(a,b)$

The next theorem is analogous to
Theorems~\ref{thmtype:5.1.3} and
\ref{thmtype:9.1.2}.

\begin{theorem}\label{thmtype:10.3.1}
Suppose the $n\times n$ matrix $A=A(t)$ is continuous on $(a,b)$.
Then a set
$\{{\bf y}_1,{\bf y}_2,\dots,{\bf y}_n\}$ of $n$ solutions of ${\bf
y}'=A(t){\bf y}$ on $(a,b)$ is a fundamental set if and only if it's
linearly independent on $(a,b)$.
\end{theorem}

\begin{example}\label{example:10.3.1}
Show that the vector functions
$$
{\bf y}_1=\begin{bmatrix}e^t\\0\\e^{-t}\end{bmatrix},\quad
{\bf
y}_2=\begin{bmatrix}0\\e^{3t}\\1\end{bmatrix},
\quad\mbox{and}\quad{\bf
y}_3=\begin{bmatrix}e^{2t}\\e^{3t}\\0\end{bmatrix}
$$
are linearly independent on every interval  $(a,b)$.


\begin{explanation}
Suppose
$$
c_1\begin{bmatrix}e^t\\0\\e^{-t}\end{bmatrix}+
c_2\begin{bmatrix}0\\e^{3t}\\1\end{bmatrix}+c_3
\begin{bmatrix}e^{2t}\\e^{3t}\\0\end{bmatrix}=
\begin{bmatrix}0\\0\\0\end{bmatrix},\quad a<t<b.
$$
We must show that $c_1=c_2=c_3=0$. Rewriting this equation in matrix
form yields
$$
\begin{bmatrix}e^t&0&e^{2t}\\0&e^{3t}&e^{3t}\\e^{-t}&1&0
\end{bmatrix}
\begin{bmatrix}c_1\\c_2\\c_3\end{bmatrix}=
\begin{bmatrix}0\\0\\0\end{bmatrix},\quad a<t<b.
$$
Expanding the determinant of this system in cofactors of the entries
of the first row yields
\begin{eqnarray*}
\begin{vmatrix}e^t&0&e^{2t}\\0&e^{3t}&e^{3t}\\e^{-t}&1&0
\end{vmatrix}&=&e^t
\begin{vmatrix}e^{3t}&e^{3t}\\1&0\end{vmatrix}-0
\begin{vmatrix}0&e^{3t}\\e^{-t}&0\end{vmatrix}
+e^{2t}\begin{vmatrix}0&e^{3t}\\e^{-t}&1\end{vmatrix}\\
&=&e^t(-e^{3t})+e^{2t}(-e^{2t})=-2e^{4t}.
\end{eqnarray*}
Since this determinant is never zero,
$c_1=c_2=c_3=0$.
\end{explanation}
\end{example}


We can use the method in
 Example~\ref{example:10.3.1}  to test
$n$ solutions $\{{\bf y}_1,{\bf y}_2,\dots,{\bf y}_n\}$ of any
$n\times n$ system
${\bf y}'=A(t){\bf y}$  for linear independence on an interval $(a,b)$
on which $A$ is continuous.  To explain this (and for other purposes
later), it's useful to write a linear combination of
${\bf y}_1, {\bf y}_2, \dots, {\bf y}_n$ in a different way. We first
write the vector functions in terms of their components as
$$
{\bf y}_1=\begin{bmatrix} y_{11}\\y_{21}\\ \vdots\\
y_{n1}\end{bmatrix},\quad
{\bf y}_2=\begin{bmatrix} y_{12}\\y_{22}\\ \vdots\\
y_{n2}\end{bmatrix},\dots,\quad
{\bf y}_n=\begin{bmatrix} y_{1n}\\y_{2n}\\ \vdots\\
y_{nn}\end{bmatrix}.
$$
If
$$
{\bf y}=c_1{\bf y}_1+c_2{\bf y}_2+\cdots+c_n{\bf y}_n
$$
then

\begin{eqnarray*}
{\bf y}&=&
c_1\begin{bmatrix} y_{11}\\y_{21}\\ \vdots\\
y_{n1}\end{bmatrix}+
c_2\begin{bmatrix} y_{12}\\y_{22}\\ \vdots\\
y_{n2}\end{bmatrix}+\cdots
+c_n\begin{bmatrix} y_{1n}\\y_{2n}\\ \vdots\\
y_{nn}\end{bmatrix}\\
&=&\begin{bmatrix}
y_{11}&y_{12}&\cdots&y_{1n} \\
y_{21}&y_{22}&\cdots&y_{2n}\\
\vdots&\vdots&\ddots&\vdots \\
y_{n1}&y_{n2}&\cdots&y_{nn} \\
\end{bmatrix}\begin{bmatrix}c_1\\c_2\\\vdots\\c_n\end{bmatrix}.
\end{eqnarray*}
This shows that
\begin{equation} \label{eq:10.3.3}
c_1{\bf y}_1+c_2{\bf y}_2+\cdots+c_n{\bf y}_n=Y{\bf c},
\end{equation}
where
$$
{\bf c}=\begin{bmatrix}c_1\\c_2\\\vdots\\c_n\end{bmatrix}
$$

and
\begin{equation} \label{eq:10.3.4}
Y=[{\bf y}_1\; {\bf y}_2\; \cdots\; {\bf y}_n]=
\begin{bmatrix}
y_{11}&y_{12}&\cdots&y_{1n} \\
y_{21}&y_{22}&\cdots&y_{2n}\\
\vdots&\vdots&\ddots&\vdots \\
y_{n1}&y_{n2}&\cdots&y_{nn} \\
\end{bmatrix};
\end{equation}
 that is, the columns of $Y$
are the vector functions ${\bf y}_1,{\bf y}_2,\dots,{\bf y}_n$.

For reference below, note that
\begin{eqnarray*}
Y'&=&[{\bf y}_1'\; {\bf y}_2'\; \cdots\; {\bf y}_n']\\
&=&[A{\bf y}_1\; A{\bf y}_2\; \cdots\; A{\bf y}_n]\\
&=&A[{\bf y}_1\; {\bf y}_2\; \cdots\; {\bf y}_n]=AY;
\end{eqnarray*}
that is, $Y$ satisfies the matrix differential equation

$$
Y'=AY.
$$

The determinant of $Y$,
\begin{equation} \label{eq:10.3.5}
W=\begin{vmatrix}
y_{11}&y_{12}&\cdots&y_{1n} \\
y_{21}&y_{22}&\cdots&y_{2n}\\
\vdots&\vdots&\ddots&\vdots \\
y_{n1}&y_{n2}&\cdots&y_{nn} \\
\end{vmatrix}
\end{equation}
is called the
\href{http://www-history.mcs.st-and.ac.uk/Mathematicians/Wronski.html}{Wronskian} of $\{{\bf y}_1,{\bf y}_2,\dots,{\bf
y}_n\}$. It can be shown 
%(Exercises~\ref{exer:10.3.2} and \ref{exer:10.3.3})
that this definition is analogous to definitions of the Wronskian of
scalar functions given in Sections~5.1 and 9.1.
The next theorem is analogous to
Theorems~\ref{thmtype:5.1.4} and
\ref{thmtype:9.1.3}. %The proof is sketched in
%Exercise~\ref{exer:10.3.4} for
%$n=2$ and in Exercise~\ref{exer:10.3.5} for general~$n$.

\begin{theorem}[Abel's Formula] \label{thmtype:10.3.2}
Suppose the $n\times n$ matrix $A=A(t)$ is continuous on $(a,b),$ let
${\bf y}_1, {\bf y}_2, \dots, {\bf y}_n$ be solutions of ${\bf
y}'=A(t){\bf y}$ on $(a,b),$ and let $t_0$ be in $(a,b)$. Then the
Wronskian of $\{{\bf y}_1,{\bf y}_2,\dots,{\bf y}_n\}$ is given by
\begin{equation} \label{eq:10.3.6}
W(t)=W(t_0)\exp\left(
\int^t_{t_0}\big[a_{11}(s)+a_{22}(s)+\cdots+a_{nn}(s)]\,
ds\right), \quad  a < t < b.
\end{equation}
Therefore$,$  either $W$ has no zeros in  $(a,b)$ or $W\equiv0$
on  $(a,b).$
\end{theorem}

\begin{remark}
The sum of the diagonal entries of a square matrix $A$ is called the
\dfn{trace} of $A$, denoted by tr$(A)$. Thus, for an $n\times n$
matrix $A$,
$$
\mbox{tr}(A)=a_{11}+a_{22}+\cdots+a_{nn},
$$
and  \eqref{eq:10.3.6} can be written as
$$
W(t)=W(t_0)\exp\left(
\int^t_{t_0}\mbox{tr}(A(s))\,
ds\right), \quad a < t < b.
$$
\end{remark}

The next theorem is analogous to
Theorems~\ref{thmtype:5.1.6} and
\ref{thmtype:9.1.4}.

\begin{theorem}\label{thmtype:10.3.3}
Suppose the  $n\times n$ matrix $A=A(t)$ is continuous
 on $(a,b)$ and let
${\bf y}_1, {\bf y}_2, \dots,{\bf y}_n$
be  solutions of ${\bf y}'=A(t){\bf y}$ on  $(a,b)$.
 Then the following statements are equivalent; that is, they are
either all true or all false:
\begin{enumerate}
\item\label{item:10.3.3a} % (a)
The general solution of ${\bf y}'=A(t){\bf y}$ on  $(a,b)$ is
${\bf y}=c_1{\bf y}_1+c_2{\bf y}_2+\cdots+c_n{\bf y}_n$,
where $c_1$, $c_2$, \dots, $c_n$  are arbitrary constants.
\item\label{item:10.3.3b} % (b)
  $\{{\bf y}_1,{\bf y}_2,\dots,{\bf y}_n\}$ is a fundamental
set of solutions of ${\bf y}'=A(t){\bf y}$  on $(a,b)$.
\item\label{item:10.3.3c} % (c)
 $\{{\bf y}_1,{\bf y}_2,\dots,{\bf y}_n\}$ is linearly
independent on $(a,b)$.
\item\label{item:10.3.3d} % (d)
The Wronskian of  $\{{\bf y}_1,{\bf y}_2,\dots,{\bf y}_n\}$ is nonzero
at some point in $(a,b)$.
\item\label{item:10.3.3e} % (e)
The Wronskian of  $\{{\bf y}_1,{\bf y}_2,\dots,{\bf y}_n\}$ is nonzero
at all points in $(a,b)$.
\end{enumerate}
\end{theorem}


We say that $Y$ in \eqref{eq:10.3.4} is a \dfn{fundamental matrix} for
${\bf y}'=A(t){\bf y}$ if any (and therefore all) of the statements
\ref{item:10.3.3a}-\ref{item:10.3.3e} of Theorem~\ref{thmtype:10.3.2} are true for the
columns of $Y$. In this case, \eqref{eq:10.3.3} implies that the general
solution of ${\bf y}'=A(t){\bf y}$ can be written as ${\bf y}=Y{\bf
c}$, where ${\bf c}$ is an arbitrary constant $n$-vector.

\begin{example}\label{example:10.3.2}
The vector functions
$$
{\bf y}_1=\begin{bmatrix} -e^{2t}\\2e^{2t}\end{bmatrix}\quad\mbox{and}\quad
{\bf y}_2=\begin{bmatrix}-e^{-t}\\e^{-t}\end{bmatrix}
$$
are solutions of the constant coefficient system
\begin{equation} \label{eq:10.3.7}
{\bf y}'=\begin{bmatrix}-4&-3\\ 6&5\end{bmatrix} {\bf y}
\end{equation}
on $(-\infty,\infty)$.  (Verify.)
\begin{enumerate}
\item\label{item:10.3.2a} % (a)
Compute the Wronskian of $\{{\bf y}_1,{\bf y}_2\}$
directly from the definition \eqref{eq:10.3.5}
\item\label{item:10.3.2b} % (b)
Verify Abel's formula \eqref{eq:10.3.6} for the Wronskian of
$\{{\bf y}_1,{\bf y}_2\}$.
\item\label{item:10.3.2c} % (c)
Find the general solution of \eqref{eq:10.3.7}.
\item\label{item:10.3.2d} % (d)
 Solve the initial value problem
\begin{equation} \label{eq:10.3.8}
{\bf y}'=\begin{bmatrix}-4&-3\\6&5\end{bmatrix} {\bf y}, \quad {\bf y}(0)=
\begin{bmatrix} 4 \\-5\end{bmatrix}.
\end{equation}
\end{enumerate}


\begin{explanation}
\ref{item:10.3.2a}
From \eqref{eq:10.3.5}
\begin{equation} \label{eq:10.3.9}
W(t)=\begin{vmatrix}-e^{2t}&-e^{-t}\\2e^{2t}&
e^{-t}\end{vmatrix}=
\answer{e^t}
\end{equation}

\ref{item:10.3.2b}  Here
$$
A=\begin{bmatrix}-4&-3\\ 6&5\end{bmatrix},
$$
so
tr$(A)=-4+5=1$. If $t_0$ is an arbitrary real number then
\eqref{eq:10.3.6}  implies that
$$
W(t)=W(t_0)\exp{\left(\int_{t_0}^t1\,ds\right)}=
\begin{bmatrix}
-e^{2t_0}&-e^{-t_0}\\2e^{2t_0}&e^{-t_0}\end{bmatrix}e^{(t-t_0)}
=e^{t_0}e^{t-t_0}=e^t,
$$
which is consistent with \eqref{eq:10.3.9}.

\ref{item:10.3.2c} Since $W(t)\ne0$,
Theorem~\ref{thmtype:10.3.3} implies that $\{{\bf y}_1,{\bf y}_2\}$ is
a fundamental set of solutions  of \eqref{eq:10.3.7} and
$$
Y=\begin{bmatrix}-e^{2t}&-e^{-t}\\2e^{2t}&
e^{-t}\end{bmatrix}
$$
is a fundamental matrix for \eqref{eq:10.3.7}.
 Therefore the general
solution of \eqref{eq:10.3.7} is
\begin{equation} \label{eq:10.3.10}
{\bf y}=c_1{\bf y}_1+c_2{\bf y}_2=
c_1\begin{bmatrix} -e^{2t}\\2e^{2t}\end{bmatrix}+c_2\begin{bmatrix}-e^{-t}\\e^{-t}\end{bmatrix}
=\begin{bmatrix}-e^{2t}&-e^{-t}\\2e^{2t}&
e^{-t}\end{bmatrix}
\begin{bmatrix}c_1\\c_2\end{bmatrix}.
\end{equation}

\ref{item:10.3.2d}
Setting $t=0$ in \eqref{eq:10.3.10} and imposing the initial condition
in \eqref{eq:10.3.8} yields
$$
c_1\begin{bmatrix}-1 \\2\end{bmatrix}+c_2
\begin{bmatrix}-1 \\1\end{bmatrix}=
\begin{bmatrix} 4 \\-5\end{bmatrix}.
$$
Thus,
\begin{eqnarray*}
-c_1-c_2&=&4 \\
2c_1+c_2&=&-5.
\end{eqnarray*}
The solution of this system is $c_1=-1$, $c_2=-3$.  Substituting these
values into  \eqref{eq:10.3.10} yields
$$
{\bf y}=-\begin{bmatrix}-e^{2t} \\ 2e^{2t}\end{bmatrix}-3
\begin{bmatrix}-e^{-t} \\ e^{-t}\end{bmatrix}=
\begin{bmatrix} e^{2t}+3e^{-t} \\-2e^{2t}-3e^{-t}
\end{bmatrix}
$$
as the solution of  \eqref{eq:10.3.8}.
\end{explanation}
\end{example}


\section*{Text Source}
Trench, William F., "Elementary Differential Equations" (2013). Faculty Authored and Edited Books \& CDs. 8. (CC-BY-NC-SA)

\href{https://digitalcommons.trinity.edu/mono/8/}{https://digitalcommons.trinity.edu/mono/8/}


\end{document}