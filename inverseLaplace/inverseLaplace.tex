\documentclass{ximera}

%% You can put user macros here
%% However, you cannot make new environments

\listfiles

\graphicspath{{./}{firstExample/}{secondExample/}}

\usepackage{tikz}
\usepackage{tkz-euclide}
\usepackage{tikz-3dplot}
\usepackage{tikz-cd}
\usetikzlibrary{shapes.geometric}
\usetikzlibrary{arrows}
\usetikzlibrary{decorations.pathmorphing,patterns}
\usetkzobj{all}
\pgfplotsset{compat=1.13} % prevents compile error.

\renewcommand{\vec}[1]{\mathbf{#1}}
\newcommand{\RR}{\mathbb{R}}
\newcommand{\dfn}{\textit}
\newcommand{\dotp}{\cdot}
\newcommand{\id}{\text{id}}
\newcommand\norm[1]{\left\lVert#1\right\rVert}
 
\newtheorem{general}{Generalization}
\newtheorem{initprob}{Exploration Problem}

\tikzstyle geometryDiagrams=[ultra thick,color=blue!50!black]

\usepackage{mathtools}

\title{The Inverse Laplace Transform}%\label{Module 7-ADEF}


\begin{document}

\begin{abstract}

\end{abstract}

\maketitle

\section*{The Inverse Laplace Transform}

\subsection*{Definition of the Inverse Laplace Transform}

In \href{https://ximera.osu.edu/ode/main/introToLaplace/introToLaplace}{Trench 8.1} we defined the Laplace transform
of
$f$ by
$$
F(s)={\cal L}(f)=\int_0^\infty e^{-st}f(t)\,dt.
$$
We'll also say that $f$ is an \textit{inverse Laplace Transform} of
$F$, and write
$$
f={\cal L}^{-1}(F).
$$
To solve differential equations with the Laplace transform, we
must be able to obtain $f$ from its transform $F$. There's a formula
for doing this, but we can't use it because it requires the theory of
functions of a complex variable. Fortunately, we can use the table of
Laplace transforms to find inverse transforms that we'll need.

\begin{example}\label{example:8.2.1}
Use the table of Laplace transforms to find
\begin{enumerate}
    
    \item\label{item:8.2.1a} ${\cal L}^{-1}\left(\frac{1}{s^2-1}\right)$
    \item\label{item:8.2.1b} ${\cal L}^{-1}\left(\frac{s}{s^2+9}\right)$. 
\end{enumerate}
\begin{explanation}
\ref{item:8.2.1a}
Setting $b=1$ in  the transform pair
$$
\sinh bt\leftrightarrow \frac{b}{s^2-b^2}
$$
shows that
$$
{\cal L}^{-1}\left(\frac{1}{s^2-1}\right)=\sinh t.
$$

\ref{item:8.2.1b}
Setting $\omega=3$ in  the transform pair
$$
\cos\omega t\leftrightarrow\frac{s}{s^2+\omega^2}
$$
shows that
$$
{\cal L}^{-1}\left(\frac{s}{s^2+9}\right)=\cos3t.
$$
\end{explanation}
\end{example}

The next theorem enables us to find inverse transforms of linear
combinations of transforms in the table. We omit the proof.

\begin{theorem}[Linearity Property]\label{thmtype:8.2.1}
If $F_1, F_2, \dots, F_n$ are Laplace transforms and
$c_1, c_2, \dots, c_n$ are constants$,$ then
$$
{\cal L}^{-1}(c_1F_1+c_2F_2+\cdots+c_nF_n)=c_1{\cal L}^{-1}(F_1)+c_2{\cal
L}^{-1}(F_2)+\cdots+c_n{\cal L}^{-1}F_n.
$$
\end{theorem}

\begin{example}\label{example:8.2.2} Find
$$
{\cal L}^{-1}\left(\frac{8}{s+5}+\frac{7}{s^2+3}\right).
$$


\begin{explanation}
From the table of Laplace transforms in \href{https://ximera.osu.edu/ode/main/laplaceTable/laplaceTable}{Trench 8.8},
$$
e^{at}\leftrightarrow \frac{1}{s-a}\quad\mbox{and}\quad
\sin\omega t\leftrightarrow \frac{\omega}{s^2+\omega^2}.
$$
Theorem~\ref{thmtype:8.2.1}
with $a=-5$ and   $\omega=\sqrt3$
yields
\begin{eqnarray*}
{\cal L}^{-1}\left(\frac{8}{s+5}+\frac{7}{s^2+3}\right)&=&
8{\cal L}^{-1}\left(\frac{1}{s+5}\right)+7{\cal L}^{-1}\left(\frac{1}{s^2+3}\right)\\
&=&
8{\cal L}^{-1}\left(\frac{1}{s+5}\right)+\frac{7}{\sqrt3}{\cal
L}^{-1}\left(\frac{\sqrt3}{s^2+3}\right)\\&=&8e^{-5t}+\frac{7}{\sqrt3}\sin\sqrt3t.
\end{eqnarray*}
\end{explanation}
\end{example}

\begin{example}\label{example:8.2.3} Find
$$
{\cal L}^{-1}\left(\frac{3s+8}{s^2+2s+5}\right).
$$

\begin{explanation}
Completing the square in the denominator yields
$$
\frac{3s+8}{s^2+2s+5}=\frac{3s+8}{(s+1)^2+4}.
$$
Because of the form of the denominator, we consider
the transform pairs
$$
 e^{-t}\cos 2t\leftrightarrow\frac{s+1}{(s+1)^2+4}
\quad\mbox{and}\quad
 e^{-t}\sin 2t\leftrightarrow\frac{2}{(s+1)^2+4},
$$
and write
\begin{eqnarray*}
{\cal L}^{-1}\left(\frac{3s+8}{(s+1)^2+4}\right)&=&
{\cal L}^{-1}\left(\frac{3s+3}{(s+1)^2+4}\right)+
{\cal L}^{-1}\left(\frac{5}{(s+1)^2+4}\right)\\&=&
3{\cal L}^{-1}\left(\frac{s+1}{(s+1)^2+4}\right)+
\frac{5}{2}{\cal L}^{-1}\left(\frac{2}{(s+1)^2+4}\right)\\&=&
e^{-t}(3\cos 2t+\frac{5}{2}\sin 2t).
\end{eqnarray*}
\end{explanation}
\end{example}

\begin{remark}
We'll often write inverse Laplace transforms of specific
functions without explicitly stating how they are obtained. In such
cases you should refer to the table of Laplace transforms in \href{https://ximera.osu.edu/ode/main/laplaceTable/laplaceTable}{Trench 8.8}.
\end{remark}

\subsection*{Inverse Laplace Transforms of Rational Functions}

Using the Laplace transform to
 solve differential equations
often requires  finding the inverse
transform of a rational function
$$
F(s)=\frac{P(s)}{Q(s)},
$$
where $P$ and $Q$ are polynomials in $s$ with no common factors. Since
it can be shown that $\lim_{s\rightarrow\infty}F(s)=0$ if $F$ is a Laplace
transform, we need only consider the case where
$\mbox{degree}(P)<\mbox{degree}(Q)$. To obtain ${\cal L}^{-1}(F)$, we
find the partial fraction expansion of $F$, obtain inverse transforms
of the individual terms in the expansion from the table of Laplace
transforms, and use the linearity property of the inverse transform.
The next two examples illustrate this.

\begin{example}\label{example:8.2.4} Find the inverse Laplace transform
of
\begin{equation}\label{eq:8.2.1}
F(s)=\frac{3s+2}{s^2-3s+2}.
\end{equation}
\begin{explanation}
({\bf Method 1:})
 Factoring the denominator in  \eqref{eq:8.2.1} yields
\begin{equation}\label{eq:8.2.2}
F(s)=\frac{3s+2}{(s-1)(s-2)}.
\end{equation}
The form for the partial fraction expansion is
\begin{equation}\label{eq:8.2.3}
\frac{3s+2}{(s-1)(s-2)}=\frac{A}{s-1}+\frac{B}{s-2}.
\end{equation}
Multiplying  this by $(s-1)(s-2)$ yields
$$
3s+2=(s-2)A+(s-1)B.
$$
Setting $s=2$ yields $B=8$ and setting $s=1$ yields $A=-5$. Therefore
$$
F(s)=-\frac{5}{s-1}+\frac{8}{s-2}
$$
and
$$
{\cal L}^{-1}(F)=-5{\cal L}^{-1}\left(\frac{1}{s-1}\right)
+8{\cal L}^{-1}\left(\frac{1}{s-2}\right)=-5e^t+8e^{2t}.
$$
\end{explanation}
\begin{explanation}
({\bf Method 2:})
We don't really have to multiply  \eqref{eq:8.2.3} by $(s-1)(s-2)$
to compute $A$ and $B$. We can obtain $A$ by simply ignoring the factor
$s-1$ in the denominator of  \eqref{eq:8.2.2} and setting $s=1$ elsewhere;
thus,
\begin{equation}\label{eq:8.2.4}
A=\left.\frac{3s+2}{s-2}\right|_{s=1}=\frac{3\cdot1+2}{ 1-2}=-5.
\end{equation}
Similarly, we can obtain $B$ by ignoring the factor
$s-2$ in the denominator of  \eqref{eq:8.2.2} and setting $s=2$ elsewhere;
thus,
\begin{equation}\label{eq:8.2.5}
B=\left.\frac{3s+2}{s-1}\right|_{s=2}=\frac{3\cdot2+2}{2-1}=8.
\end{equation}
To justify this, we observe that multiplying  \eqref{eq:8.2.3} by $s-1$
yields
$$
\frac{3s+2}{s-2}=A+(s-1)\frac{B}{s-2},
$$
and setting $s=1$ leads to
\eqref{eq:8.2.4}. Similarly, multiplying  \eqref{eq:8.2.3} by $s-2$ yields
$$
\frac{3s+2}{s-1}=(s-2)\frac{A}{s-2}+B
$$
and setting $s=2$ leads to
\eqref{eq:8.2.5}.  (It isn't necessary to
 write the last two equations. We wrote them only to justify
the shortcut procedure indicated in  \eqref{eq:8.2.4} and
\eqref{eq:8.2.5}.)
\end{explanation}
\end{example}

The shortcut employed in the second solution of
Example~\ref{example:8.2.4} is
\href{http://www-history.mcs.st-and.ac.uk/Mathematicians/Heaviside.html}{Heaviside's method}.
The next theorem states this method formally. 
% For a proof
% and an extension of this theorem, see Exercise~\ref{exer:8.2.10}.

\begin{theorem}\label{thmtype:8.2.2}
Suppose
\begin{equation}\label{eq:8.2.6}
F(s)=\frac{P(s)}{(s-s_1)(s-s_2)\cdots(s-s_n)},
\end{equation}
where $s_1, s_2, \dots, s_n$ are distinct  and $P$ is a polynomial of
degree less than $n$.  Then
$$
F(s)=\frac{A_1}{s-s_1}+\frac{A_2}{s-s_2}+\cdots+\frac{A_n}{s-s_n},$$
where $A_i$ can be computed from \eqref{eq:8.2.6}
by ignoring the factor $s-s_i$ and setting $s=s_i$ elsewhere.
\end{theorem}

\begin{example}\label{example:8.2.5} Find the inverse Laplace transform
of
\begin{equation}\label{eq:8.2.7}
F(s)=\frac{6+(s+1)(s^2-5s+11)}{s(s-1)(s-2)(s+1)}.
\end{equation}
\begin{explanation}
The partial fraction expansion of  \eqref{eq:8.2.7} is of the form
\begin{equation}\label{eq:8.2.8}
F(s)=\frac{A}{s}+\frac{B}{s-1}+\frac{C}{s-2}+\frac{D}{s+1}.
\end{equation}
To find $A$, we ignore the factor $s$ in the denominator of
\eqref{eq:8.2.7} and set $s=0$ elsewhere. This yields
$$
A=\frac{6+(1)(11)}{(-1)(-2)(1)}=\frac{17}{2}.
$$
Similarly, the other coefficients are given by
$$
B=\frac{6+(2)(7)}{(1)(-1)(2)}=-10,
$$
$$
C=\frac{6+3(5)}{2(1)(3)}=\frac{7}{2},
$$
and
$$
D=\frac{6}{(-1)(-2)(-3)}=-1.
$$
Therefore
$$
F(s)=\frac{17}{2}\,\frac{1}{s}-\frac{10}{s-1}+\frac{7}{2}\,\frac{1}{s-2}-\frac{1}{s+1}
$$
and
\begin{eqnarray*}
{\cal L}^{-1}(F)&=&
 \frac{17}{2}{\cal L}^{-1}\left(\frac{1}{s}\right)-10{\cal L}^{-1}\left(\frac{1}{s-1}\right)+\frac{7}{2}{\cal L}^{-1}\left(\frac{1}{s-2}\right)-{\cal L}^{-1}\left(\frac{1}{s+1}\right)\\
&=& \frac{17}{2}-10e^t+\frac{7}{2}e^{2t}-e^{-t}.
\end{eqnarray*}
\end{explanation}
\end{example}

\begin{remark}
We didn't ``multiply out'' the numerator in
\eqref{eq:8.2.7} before
computing the coefficients in \eqref{eq:8.2.8}, since
 it wouldn't simplify the computations.
\end{remark}

\begin{example}\label{example:8.2.6} Find the inverse Laplace transform
of
\begin{equation}\label{eq:8.2.9}
F(s)=\frac{8-(s+2)(4s+10)}{(s+1)(s+2)^2}.
\end{equation}
\begin{explanation}
The  form for the partial fraction expansion  is
\begin{equation}\label{eq:8.2.10}
F(s)=\frac{A}{s+1}+\frac{B}{s+2}+\frac{C}{(s+2)^2}.
\end{equation}
Because of the repeated factor $(s+2)^2$ in \eqref{eq:8.2.9}, Heaviside's
method doesn't work. Instead, we find a common denominator in
\eqref{eq:8.2.10}. This yields
\begin{equation}\label{eq:8.2.11}
F(s)=\frac{A(s+2)^2+B(s+1)(s+2)+C(s+1)}{(s+1)(s+2)^2}.
\end{equation}
If  \eqref{eq:8.2.9} and  \eqref{eq:8.2.11} are to be equivalent, then
\begin{equation}\label{eq:8.2.12}
A(s+2)^2+B(s+1)(s+2)+C(s+1)=8-(s+2)(4s+10).
\end{equation}
The two sides of this equation are polynomials of degree two. From a
theorem of algebra, they will be equal for all $s$ if they are equal
for any three distinct values of $s$. We may determine $A$, $B$ and
$C$ by choosing convenient values of $s$.

The left side of \eqref{eq:8.2.12} suggests that we take $s=-2$ to obtain
$C=-8$, and $s=-1$ to obtain $A=2$. We can now choose any third value
of $s$ to determine $B$. Taking $s=0$ yields $4A+2B+C=-12$. Since
$A=2$ and $C=-8$ this implies that $B=-6$. Therefore
$$
F(s)=\frac{2}{s+1}-\frac{6}{s+2}-\frac{8}{(s+2)^2}
$$
and
\begin{eqnarray*}
{\cal L}^{-1}(F)&=&
2{\cal L}^{-1}\left(\frac{1}{s+1}\right)-6{\cal L}^{-1}\left(\frac{1}{s+2}\right)-8{\cal L}^{-1}\left(\frac{1}{(s+2)^2}\right)\\
&=&2e^{-t}-6e^{-2t}-8te^{-2t}.
\end{eqnarray*}
\end{explanation}
\end{example}

\begin{example}\label{example:8.2.7} Find the inverse Laplace transform
of
$$
F(s)=\frac{s^2-5s+7}{(s+2)^3}.
$$

\begin{explanation}
The form for the partial fraction expansion  is
$$
F(s)=\frac{A}{s+2}+\frac{B}{(s+2)^2}+\frac{C}{(s+2)^3}.
$$
The easiest way to obtain $A$, $B$, and $C$ is to expand the numerator
in powers of $s+2$. This yields
$$
s^2-5s+7=[(s+2)-2]^2-5[(s+2)-2]+7=(s+2)^2-9(s+2)+21.
$$
Therefore
\begin{eqnarray*}
F(s)&=&\frac{(s+2)^2-9(s+2)+21}{(s+2)^3}\\[10pt]
&=&\frac{1}{s+2}-\frac{9}{(s+2)^2}+\frac{21}{(s+2)^3}
\end{eqnarray*}
and
\begin{eqnarray*}
{\cal L}^{-1}(F)&=&
{\cal L}^{-1}\left(\frac{1}{s+2}\right)-9{\cal
L}^{-1}\left(\frac{1}{(s+2)^2}\right)+\frac{21}{2}{\cal
L}^{-1}\left(\frac{2}{(s+2)^3}\right)\\&=&
e^{-2t}\left(1-9t+\frac{21}{2}t^2\right).
\end{eqnarray*}
\end{explanation}
\end{example}

\begin{example}\label{example:8.2.8} %\exampleinverse
 Find the inverse Laplace transform of
\begin{equation}\label{eq:8.2.13}
F(s)=\frac{1-s(5+3s)}{s\left[(s+1)^2+1\right]}.
\end{equation}
\begin{explanation}
One  form for the partial fraction expansion of $F$ is
\begin{equation}\label{eq:8.2.14}
F(s)=\frac{A}{s}+\frac{Bs+C}{(s+1)^2+1}.
\end{equation}
However, we see from the table of Laplace transforms that the inverse
transform of the second fraction on the right of \eqref{eq:8.2.14} will be
a linear combination of the inverse transforms
$$
e^{-t}\cos t\quad\mbox{and}\quad e^{-t}\sin t
$$
of
$$
\frac{s+1}{(s+1)^2+1}\quad\mbox{and}\quad \frac{1}{(s+1)^2+1}
$$
respectively. Therefore, instead of \eqref{eq:8.2.14} we write
\begin{equation}\label{eq:8.2.15}
F(s)=\frac{A}{s}+\frac{B(s+1)+C}{(s+1)^2+1}.
\end{equation}
Finding a common denominator yields
\begin{equation}\label{eq:8.2.16}
F(s)=\frac{A\left[(s+1)^2+1\right]+B(s+1)s+Cs}{s\left[(s+1)^2+1\right]}.
\end{equation}
If  \eqref{eq:8.2.13} and  \eqref{eq:8.2.16} are to be equivalent, then
$$
A\left[(s+1)^2+1\right]+B(s+1)s+Cs=1-s(5+3s).
$$
This is true for all $s$ if it's true for three distinct values of
$s$.  Choosing $s=0$, $-1$, and $1$ yields the system
$$
\begin{array}{rcr}
2A&=&1\\ A-C&=&3\\5A+2B+C&=&-7
\end{array}
$$
Solving this system yields
$$
A=\frac{1}{2},\quad B=-\frac{7}{2},\quad C=-\frac{5}{2}.
$$
Hence, from  \eqref{eq:8.2.15},
$$
F(s)=\frac{1}{2s}-\frac{7}{2}\,\frac{s+1}{(s+1)^2+1}-
\frac{5}{2}\,\frac{1}{(s+1)^2+1}.
$$
Therefore
\begin{eqnarray*}
{\cal L}^{-1}(F)&=&
\frac{1}{2}{\cal L}^{-1}\left(\frac{1}{s}\right)-{\frac{7}{2}}{\cal
L}^{-1}\left(s+\frac{1
}{(s+1)^2+1}\right)-\frac{5}{2} {\cal L}^{-1}\left(\frac{1}{(s+1)^2+1}\right)\\
&=& {\frac{1}{2}}-{\frac{7}{2}}e^{-t}\cos t-{\frac{5}{2}}e^{-t}\sin t.
\end{eqnarray*}
\end{explanation}
\end{example}

\begin{example}\label{example:8.2.9}
Find the inverse Laplace transform of
\begin{equation}\label{eq:8.2.17}
F(s)=\frac{8+3s}{(s^2+1)(s^2+4)}.
\end{equation}
\begin{explanation}
The form for the partial fraction expansion is
$$
F(s)=\frac{A+Bs}{s^2+1}+\frac{C+Ds}{s^2+4}.
$$
The coefficients $A$, $B$, $C$ and $D$ can be obtained by finding a common
denominator and equating the resulting numerator to the numerator in
\eqref{eq:8.2.17}. However, since there's no first power of $s$ in the
denominator of \eqref{eq:8.2.17}, there's an easier way: the expansion of
$$
F_1(s)=\frac{1}{(s^2+1)(s^2+4)}
$$
can be obtained quickly by using Heaviside's method to expand
$$
\frac{1}{(x+1)(x+4)}=\frac{1}{3}\left(\frac{1}{x+1}-\frac{1}{x+4}\right)
$$
and then setting $x=s^2$ to obtain
$$
\frac{1}{(s^2+1)(s^2+4)}=\frac{1}{3}\left(\frac{1}{s^2+1}-\frac{1}{s^2+4}\right).
$$
Multiplying this by $8+3s$ yields
$$
F(s)=\frac{8+3s}{(s^2+1)(s^2+4)}=\frac{1}{3}\left(\frac{8+3s}{s^2+1}-\frac{8+3s}{s^2+4}\right).
$$
Therefore
$$
{\cal L}^{-1}(F)=\frac{8}{3}\sin t+\cos t-\frac{4}{3}\sin 2t-\cos 2t.
$$
\end{explanation}
\end{example}

Some software packages that do symbolic algebra can find partial
fraction expansions very easily. We recommend that you use such a
package if one is available to you, but only after you've  done
enough partial fraction expansions on your own to  master the
technique.



\section*{Text Source}
Trench, William F., "Elementary Differential Equations" (2013). Faculty Authored and Edited Books \& CDs. 8. (CC-BY-NC-SA)

\href{https://digitalcommons.trinity.edu/mono/8/}{https://digitalcommons.trinity.edu/mono/8/}


\end{document}