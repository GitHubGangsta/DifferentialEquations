\documentclass{ximera}

%% You can put user macros here
%% However, you cannot make new environments

\listfiles

\graphicspath{{./}{firstExample/}{secondExample/}}

\usepackage{tikz}
\usepackage{tkz-euclide}
\usepackage{tikz-3dplot}
\usepackage{tikz-cd}
\usetikzlibrary{shapes.geometric}
\usetikzlibrary{arrows}
\usetikzlibrary{decorations.pathmorphing,patterns}
\usetkzobj{all}
\pgfplotsset{compat=1.13} % prevents compile error.

\renewcommand{\vec}[1]{\mathbf{#1}}
\newcommand{\RR}{\mathbb{R}}
\newcommand{\dfn}{\textit}
\newcommand{\dotp}{\cdot}
\newcommand{\id}{\text{id}}
\newcommand\norm[1]{\left\lVert#1\right\rVert}
 
\newtheorem{general}{Generalization}
\newtheorem{initprob}{Exploration Problem}

\tikzstyle geometryDiagrams=[ultra thick,color=blue!50!black]

\usepackage{mathtools}

\title{Linear Systems of Differential Equations}%\label{Module 7-ADEF}


\begin{document}

\begin{abstract}

\end{abstract}

\maketitle

\section*{Linear Systems of Differential Equations}

A  first order system of  differential equations that can be written in
the form
\begin{equation} \label{eq:10.2.1}
\begin{array}{ccl}
y'_1&=&a_{11}(t)y_1+a_{12}(t)y_2+\cdots+a_{1n}(t)y_n+f_1(t)\\
y'_2&=&a_{21}(t)y_1+a_{22}(t)y_2+\cdots+a_{2n}(t)y_n+f_2(t)\\
&\vdots\\
y'_n&
=&a_{n1}(t)y_1+a_{n2}(t)y_2+\cdots+a_{nn}(t)y_n+f_n(t)\end{array}
\end{equation}
is called a \dfn{linear system}.

The linear system \eqref{eq:10.2.1}  can be written in matrix form as
$$
\begin{bmatrix}y_1'\\y_2'\\\vdots\\y_n'\end{bmatrix}=\begin{bmatrix}a_{11}(t)&a_{12}(t)&\dots &a_{1n}(t)\\
a_{21}(t)&a_{22}(t)&\dots &a_{2n}(t)\\
\vdots &\vdots &\ddots &\vdots\\
a_{n1}(t)&a_{n2}(t)&\dots &a_{nn}(t)\end{bmatrix}\begin{bmatrix}y_1\\y_2\\\vdots\\y_n\end{bmatrix}+\begin{bmatrix}f_1(t)\\f_2(t)\\\vdots\\f_n(t)\end{bmatrix},
$$

or more briefly as
\begin{equation} \label{eq:10.2.2}
\vec{y}'=A(t)\vec{y}+\vec{f}(t),
\end{equation}
where
$$
\vec{y}=\begin{bmatrix}y_1\\y_2\\\vdots\\y_n\end{bmatrix},\quad
A(t)=\begin{bmatrix}a_{11}(t)&a_{12}(t)&\dots &a_{1n}(t)\\
a_{21}(t)&a_{22}(t)&\dots &a_{2n}(t)\\
\vdots &\vdots &\ddots &\vdots\\
a_{n1}(t)&a_{n2}(t)&\dots &a_{nn}(t)\end{bmatrix},\quad\mbox{and}\quad\vec{f}(t)=\begin{bmatrix}f_1(t)\\f_2(t)\\\vdots\\f_n(t)\end{bmatrix}
$$
We call $A$  the \dfn{coefficient matrix} of \eqref{eq:10.2.2} and
${\bf f}$  the \dfn{forcing function}.   We'll say that $A$
and $\vec{f}$ are \dfn{continuous} if their entries are continuous.
If ${\bf f}={\bf 0}$, then \eqref{eq:10.2.2} is \dfn{homogeneous};
otherwise, \eqref{eq:10.2.2} is \dfn{nonhomogeneous}.

An  initial value problem  for \eqref{eq:10.2.2} consists of
finding a solution of \eqref{eq:10.2.2} that equals a given constant
vector
$$
{\bf k} =\begin{bmatrix}k_1\\k_2\\\vdots\\k_n\end{bmatrix}.
$$
at some initial point $t_0$. We write this initial value problem as
$$
{\bf y}'=A(t){\bf y}+{\bf f}(t), \quad  {\bf y}(t_0)={\bf k}.
$$

The next theorem gives sufficient conditions for the existence
of solutions of initial value problems for \eqref{eq:10.2.2}. We omit the
proof.

\begin{theorem}\label{thmtype:10.2.1}
Suppose the coefficient matrix $A$ and the forcing function ${\bf
f}$ are continuous on $(a,b)$, let $t_0$ be in $(a,b)$, and let ${\bf
k}$ be an arbitrary constant $n$-vector. Then the initial value
problem
$$
{\bf y}'=A(t){\bf y}+{\bf f}(t), \quad  {\bf y}(t_0)={\bf k}
$$
 has a unique solution on $(a,b)$.
\end{theorem}

\begin{example}\label{example:10.2.1}

\begin{enumerate}
\item\label{item:10.2.1a} % (a)
Write the system
\begin{equation} \label{eq:10.2.3}
\begin{array}{rcl}
y_1'&=&y_1+2y_2+2e^{4t} \\
y_2'&=&2y_1+y_2+e^{4t}
\end{array}
\end{equation}
in matrix form and conclude from Theorem~\ref{thmtype:10.2.1} that every
initial value problem for \eqref{eq:10.2.3} has a unique solution on
$(-\infty,\infty)$.
\item\label{item:10.2.1b} % (b)
Verify that
\begin{equation} \label{eq:10.2.4}
{\bf y}=
\frac{1}{5}\begin{bmatrix}8\\7\end{bmatrix}e^{4t}+c_1\begin{bmatrix}1\\1\end{bmatrix}e^{3t}+c_2\begin{bmatrix}1\\-1\end{bmatrix}e^{-t}
\end{equation}
is a solution of \eqref{eq:10.2.3} for all values of the constants $c_1$
and $c_2$.
\item\label{item:10.2.1c} % (c)
Find the  solution of the initial value problem
\begin{equation} \label{eq:10.2.5}
{\bf y}'=\begin{bmatrix}1&2\\2&1\end{bmatrix}{\bf y}+\begin{bmatrix}2\\1\end{bmatrix}e^{4t},\quad  {\bf
y}(0)=\frac{1}{5}\begin{bmatrix}3\\22\end{bmatrix}.
\end{equation}
\end{enumerate}

\begin{explanation}
\ref{item:10.2.1a}
The system  \eqref{eq:10.2.3} can be written in matrix form as
$$
{\bf y}'=\begin{bmatrix}1&2\\2&1\end{bmatrix}{\bf y}+\begin{bmatrix}2\\1\end{bmatrix}e^{4t}.
$$
An initial value problem for \eqref{eq:10.2.3} can be written as
$$
{\bf y}'=\begin{bmatrix}1&2\\2&1\end{bmatrix}{\bf y}+\begin{bmatrix}2\\1\end{bmatrix}e^{4t}, \quad
y(t_0)=\begin{bmatrix}k_1\\k_2\end{bmatrix}.
$$
Since the coefficient matrix and the forcing function are both
continuous on $(-\infty,\infty)$, Theorem~\ref{thmtype:10.2.1} implies that
this problem has a unique solution on $(-\infty,\infty)$.

\ref{item:10.2.1b}
If ${\bf y}$ is given by \eqref{eq:10.2.4}, then
\begin{eqnarray*}
A{\bf y}+{\bf f}&=&
\frac{1}{5}\begin{bmatrix}1&2\\2&1\end{bmatrix}\begin{bmatrix}8\\7\end{bmatrix}e^{4t}+
c_1\begin{bmatrix}1&2\\2&1\end{bmatrix}\begin{bmatrix}1\\1\end{bmatrix}e^{3t}\\
&&+c_2\begin{bmatrix}1&2\\2&1\end{bmatrix}\begin{bmatrix}1\\-1\end{bmatrix}e^{-t}
+\begin{bmatrix}2\\1\end{bmatrix}e^{4t}\\
&=&\frac{1}{5}\begin{bmatrix}22\\23\end{bmatrix}e^{4t}+c_1\begin{bmatrix}3\\3\end{bmatrix}e^{3t}+c_2\begin{bmatrix}-1\\1\end{bmatrix}e^{-t}
+\begin{bmatrix}2\\1\end{bmatrix}e^{4t}\\
&=&\frac{1}{5}\begin{bmatrix}32\\28\end{bmatrix}e^{4t}+3c_1\begin{bmatrix}1\\1\end{bmatrix}e^{3t}-c_2\begin{bmatrix}1\\-1\end{bmatrix}e^{-t}
={\bf y}'.
\end{eqnarray*}

\ref{item:10.2.1c}
We must choose $c_1$ and $c_2$ in \eqref{eq:10.2.4} so that
$$
\frac{1}{5}\begin{bmatrix}8\\7\end{bmatrix}+c_1\begin{bmatrix}1\\1\end{bmatrix}+c_2\begin{bmatrix}1\\-1\end{bmatrix}=\frac{1}{5}\begin{bmatrix}3\\22\end{bmatrix},
$$
which is equivalent to
$$
\begin{bmatrix}1&1\\1&-1\end{bmatrix}\begin{bmatrix}c_1\\c_2\end{bmatrix}=\begin{bmatrix}-1\\3\end{bmatrix}.
$$
Solving this system yields $c_1=1$, $c_2=-2$, so
$$
{\bf y}=\frac{1}{5}\begin{bmatrix}8\\7\end{bmatrix}e^{4t}+\begin{bmatrix}1\\1\end{bmatrix}e^{3t}-2\begin{bmatrix}1\\-1\end{bmatrix}e^{-t}
$$
is the solution of  \eqref{eq:10.2.5}.
\end{explanation}
\end{example}


\begin{remark}The theory of $n\times n$ linear systems of differential
equations is analogous to the theory of the scalar $n$-th order
equation
\begin{equation} \label{eq:10.2.6}
P_0(t)y^{(n)}+P_1(t)y^{(n-1)}+\cdots+P_n(t)y=F(t),
\end{equation}
as
developed in Sections~9.1. For example, by rewriting
\eqref{eq:10.2.6} as an equivalent linear system it can be shown that
Theorem~\ref{thmtype:10.2.1} implies Theorem~\ref{thmtype:9.1.1}.
%(Exercise~\ref{exer:10.2.12}).
\end{remark}




\section*{Text Source}
Trench, William F., "Elementary Differential Equations" (2013). Faculty Authored and Edited Books \& CDs. 8. (CC-BY-NC-SA)

\href{https://digitalcommons.trinity.edu/mono/8/}{https://digitalcommons.trinity.edu/mono/8/}


\end{document}