\documentclass{ximera}
 
%% You can put user macros here
%% However, you cannot make new environments

\listfiles

\graphicspath{{./}{firstExample/}{secondExample/}}

\usepackage{tikz}
\usepackage{tkz-euclide}
\usepackage{tikz-3dplot}
\usepackage{tikz-cd}
\usetikzlibrary{shapes.geometric}
\usetikzlibrary{arrows}
\usetikzlibrary{decorations.pathmorphing,patterns}
\usetkzobj{all}
\pgfplotsset{compat=1.13} % prevents compile error.

\renewcommand{\vec}[1]{\mathbf{#1}}
\newcommand{\RR}{\mathbb{R}}
\newcommand{\dfn}{\textit}
\newcommand{\dotp}{\cdot}
\newcommand{\id}{\text{id}}
\newcommand\norm[1]{\left\lVert#1\right\rVert}
 
\newtheorem{general}{Generalization}
\newtheorem{initprob}{Exploration Problem}

\tikzstyle geometryDiagrams=[ultra thick,color=blue!50!black]

\usepackage{mathtools}
 
\title{Linear First-Order Differential Equations}
 
 
\begin{document}
 
\begin{abstract}
We develop a technique for solving first-order linear differential equations.
\end{abstract}
 
\maketitle
 
\section*{Linear First-Order Differential Equations}
\subsection*{Identifying Linear First-Order Differential Equations}
 
We say that a first-order equation is \textit{linear} if it can be expressed in the form:
\begin{equation}
\label{eq:linear-first-order-de}
y'+p(x)y=f(x)
\end{equation}
\begin{example}\label{ex:linearex1}
$$y'+2y=xe^{-2x}$$
This equation is linear. Comparing with~\eqref{eq:linear-first-order-de}, we see that $p(x)=2$ and $f(x)=xe^{-2x}$.
\end{example}
 
\begin{example}\label{ex:linearex2}
$$x^2y'+3y=x^2$$
This equation is not in the form of~\eqref{eq:linear-first-order-de}. However, we can divide the equation by $x^2$, obtaining:
$$
y'+\frac{1}{x^2}y=1,
$$
which is in the form of~\eqref{eq:linear-first-order-de}. We conclude that the equation is linear, and we can see directly that $p(x)=\frac{1}{x^2}$ and $f(x)=1$.
\end{example}
Notice that the form of a linear equation is quite rigid. A rule of thumb is that the only terms that depend on $y$ that can appear in a first-order linear equation are of the form:
\[
a(x)y\quad\text{or}\quad b(x)y'.
\]
We call these terms \textit{linear}. Any other kind of first-order term is \textit{nonlinear}. So, for example:
\begin{itemize}
\item $\displaystyle xy'+3y^2=2x$ is not linear, because of the nonlinear term $y^2$.
\item $\displaystyle yy' = 3$ is not linear, because of the nonlinear term $yy'$.
\item $\displaystyle y'+xe^y=12$ is not linear, because of the nonlinear term $e^y$.
\end{itemize}
 
In the following sections we develop a general method for solving linear first-order equations.
 
\subsection*{A Special Kind of Linear Equation}
We now want to devise a method to find the general solution of a linear first order differential equation. With the goal of developing intuition about the method, we start with an equation of very special form.
 
Let's consider the following DE:
\begin{equation}
\label{eq:linear-de-example1}
x^2y'+2xy=\cos x.
\end{equation}
This equation is not in the form~\ref{eq:linear-first-order-de}, but it is actually easier to solve it using the form above. Notice the following ``coincidence'':
\[
\frac{d}{dx}[x^2]=2x.
\]
In words, we can express this remarkable coincidence as:
 
\begin{center}
\emph{The derivative of the function multiplying $y'$ is equal to the function multiplying $y$}.
\end{center}
 
(Repeat this to yourself until you fully understand what it means.)
 
This coincidence allows us to rewrite the left-hand side of the equation in a very convenient form. Using the product rule we have:
\[
\frac{d}{dx}\left[x^2y(x)\right]=x^2y'(x)+2xy(x).
\]
Using this, \ref{eq:linear-de-example1} becomes:
\[
\frac{d}{dx}\left[x^2y(x)\right]=\cos x
\]
The nice thing is that this equation can be directly integrated:
\[
\int \frac{d}{dx}\left[x^2y(x)\right]\, dx = \int \cos x\,dx,
\]
so that:
\[
x^2y(x)=\sin x + C,
\]
where $C$ is an arbitrary constant. We can write the solution in explicit form by dividing by $x^2$:
\[
y(x)=\frac{\sin x + C}{x^2}
\]
Notice that this solution will only be defined on intervals that do not contain the value $x=0$. Typically, the right interval is determined by an initial condition. For example, the solution of the differential equation that satisfies the initial condition $y(\pi)=2$ is:
\[
y(x)=\frac{\sin x + 2\pi^2}{x^2}\text{ for $x$ in the interval $(0,\infty)$}.
\]
We choose the interval $(0,\infty)$ because this interval contains the value $x=\pi$, where the initial condition is specified.
 
It is not too difficult to identify the general case in which this method can be applied. The differential equation must be in the special form:
\begin{equation}
\label{eq:linear-equation-special}
m(x)y'+m'(x)y=g(x)
\end{equation}
where $m(x)$ is a function of the variable $x$ only. Then, the equation can be rewritten as:
\[
\frac{d}{dx}\left[m(x)y(x)\right]=g(x),
\]
which can be directly integrated. Equation~\ref{eq:linear-equation-special} seems to be of a very special kind. In the next section we will see that there is a method to transform any equation of the form~\ref{eq:linear-first-order-de} into an equivalent equation of the form~\ref{eq:linear-equation-special}.
 
\subsection*{General Method --- An Example}
 
We now want to move in the direction of finding a general method to solve equations like~\ref{eq:linear-first-order-de}. Let's consider the DE:
\[
y'+\frac{4}{x}y=1
\]
We would like to make this equation have the same pattern as~\ref{eq:linear-equation-special}. We can achieve this by multiplying the equation by a function $m(x)$ to be determined:
\begin{equation}
\label{eq:integrating-factor-example}
m(x)y'+\frac{4}{x}m(x)y=m(x)
\end{equation}
Comparing~\ref{eq:linear-equation-special} and~\ref{eq:integrating-factor-example}, we see that we need:
\[
m'(x)=\frac{4}{x}m(x)
\]
This is a first order separable equation, so we write it as:
\[
\frac{dm}{m}=\frac{4}{x}.
\]
Integrating we get:
\[
\ln|m(x)|=4\ln|x|+C
\]
We can simplify the work a bit by noticing that we only need \emph{one} particular solution for $m(x)$, so we can choose $C=0$:
\[
\ln|m(x)|=4\ln|x|.
\]
It's now easy to see that this can be solved algebraically to yield:
\[
m(x)=x^4.
\]
Let's now go back to~\ref{eq:integrating-factor-example} and plug in the function $m(x)$ we found above:
\[
x^4y'+4x^3y=x^4.
\]
As expected, this equation presents the same pattern observed in equation~\ref{eq:linear-de-example1}, and we can rewrite it as:
\[
\frac{d}{dx}\left[x^4y(x)\right] = x^4.
\]
Integrating this we get:
\[
x^4y(x)=\frac{x^5}{5}+C,
\]
and we get the following explicit general solution:
\[
y(x)=\frac{x}{6}+\frac{C}{x^4}.
\]
In the next section we will see how we can simplify this solution method.
 
\subsection*{A Streamlined Method}
 
The goal of this section is to present the \textit{method of integrating factors}, a streamlined procedure for the solution of a linear first order differential equation. The method is simply a reorganization of the procedure illustrated in the previous section.
 
To find the general solution of a first order linear differential equation such as~\ref{eq:linear-first-order-de}, we can proceed as follows:
 
\begin{enumerate}
\item Compute $\alpha(x)=\int p(x)\,dx$. It is safe to ignore the constant of integration here.
\item Let $m(x)=e^{\alpha(x)}$, which is called the \emph{integrating factor}.
\item Multiply both sides of~\ref{eq:linear-first-order-de}, obtaining the equation:
\[
m(x)y'+m(x)p(x)y=m(x)f(x)
\]
\item Rewrite the left-hand side of this equation as a derivative, leaving the right-hand side unchanged:
\[
\frac{d}{dt}[m(x)y(x)]=m(x)f(x)
\]
\item Integrate:
\[
m(x)y(x)=\int m(x)f(x)\,dx
\]
Notice that, here, you have to add an integration constant to the antiderivative.
\item Solve the equation explicitly for $y(x)$. Notice that this is always possible because $m(x)\ne 0$ for all $x$, since $m(x)$ is an exponential.
\end{enumerate}
 
Let's now consider some examples.
 
\begin{example}\label{ex:linearex3}
Find the general solution of the DE:
\[
y'+3y=e^{-x}
\]
 
\begin{explanation}
For this equation, we have $p(x)=3$ and $f(x)=e^{-2x}$. Following the steps outlined above:
 
\begin{enumerate}
\item Compute $\alpha(x)=\int p(x)\,dx=\int 3\,dx=3x$. (No constant of integration needed here!)
\item Compute the integrating factor: $m(x)=e^{\alpha(x)}=e^{3x}$.
\item Multiply both sides of the equation by the integrating factor $m(x)$:
\[
3y'+3e^{3x}=e^{3x}e^{-x}=e^{2x}.
\]
\item Rewrite the left-hand side as $\frac{d}{dx}[m(x)y(x)]$:
\[
\frac{d}{dx}\left[e^{3x}y(x)\right]=e^{2x}
\]
\item Integrate:
\[
e^{3x}y(x)=\int e^{2x}\,dx=\answer{\frac{1}{2}}e^{2x}+C.
\]
\item Solve explicitly for $y(x)$:
\[
y(x)=e^{-3x}\left(\frac{1}{2}e^{2x}+C\right)=\frac{1}{2}e^{\answer{-x}}+Ce^{\answer{-3x}}.
\]
\end{enumerate}
We conclude that the general solution of the initial value problem is:
\[
y(x)=\frac{1}{2}e^{-x}+Ce^{-3x}.
\]
\end{explanation}
\end{example}
 
\begin{example}\label{ex:linearex4}
Find the solution of the IVP:
\[
ty'-y=t^2e^{-t},\quad y(2)=3
\]
 
\begin{explanation}
We first need the general solution of the DE. Start by dividing both sides of the equation by $t$:
\[
y'-\frac{1}{t}y=te^{-t}
\]
Notice that the equation has a singularity at $t=0$, since the coefficient of $y$ has a vertical asymptote when $t-0$. This means that solutions to the DE will be only defined on an interval that does not contain zero. Since the initial condition is specified at $t=2$, which is positive, it is natural to assume that $t>0$.
 
So, we have: $p(t)=-\frac{1}{t}$ and $f(t)=te^{-t}$. We are now ready to find the general solution by the method of integrating factors:
\begin{enumerate}
\item Compute $\alpha(t)=\int p(t)\,dt=\int -\frac{1}{t}\,dt=-\ln t$, where we used the assumption that $t>0$.
\item Evaluate $m(t)=e^{\alpha(t)}=e^{-\ln t}=(e^{\ln t})^{-1}=\frac{1}{t}$.
\item Multiply both sides of the equation by $m(t)$:
\[
\frac{1}{t}y'-\frac{1}{t^2}y=e^{-t}.
\]
\item Rewrite the left-hand side as $\frac{d}{dt}[m(t)y(t)]$:
\[
\frac{d}{dt}\left[\frac{1}{t}y(t)\right]=e^{-t}
\]
\item Integrate:
\[
\frac{1}{t}y(t)=\int e^{-t}=-e^{-t}+C
\]
\item Solve for $y(t)$:
\[
y(t)=-te^{-t}+Ct
\]
 
% \[
% y(t)=\frac{C-e^{-t}}{t}
% \]
% Notice that, as expected, solutions will not be defined at $t=0$.
 
\end{enumerate}
Once we have the general solution, we can plug in the initial condition:
\[
y(2)=-2e^{-2}-2C=3
\]
Solving for $C$ we get:
\[
C = \frac{3}{2}+e^{-2}
\]
Using this constant, we can now write the particular solution to the IVP:
\[
y(t)=-te^{-t}+(1.5+e^{-2})t, \quad t>0
\]
 
We conclude this example by providing an interactive Sage cell which will display the particular solution to the IVP where $y(a)=b$.  If you change the values of $a$ or $b$, you will get a different solution curve, but notice that these curves are valid only for $t>0$ or $t<0$.
 
\begin{sageCell}
@interact
def _(a=input_box(1, width=5),b=input_box(1, width=5)):
    x,y=var('x,y')
    v=plot_slope_field((x*e^(-x)+(y/x)),(x,-1,5),(y,-5,5),headaxislength=3, headlength=3)
    c=(b/a)+e^(-a)
    d=plot(-x*e^(-x)+c*x,(x,-1,3))
    p=point((a,b),rgbcolor=hue(1))
    show(v+d+p)
\end{sageCell}
 
% \begin{image}
% {\def\length{sqrt(1+(x+y)^2)}
% \begin{tikzpicture}
%   \begin{axis}[
%       xmin=0, xmax=3,ymin=0,ymax=4,domain=0.1:6,view={0}{90},
%       axis lines =center, xlabel=$t$, ylabel=$y$,
%       every axis y label/.style={at=(current axis.above origin),anchor=south},
%       every axis x label/.style={at=(current axis.right of origin),anchor=west},
%       axis on top,
%     ]
%     \addplot3 [blue, quiver={u={1/\length}, v={(x*e^(-x)+(y/x))/(\length)},scale arrows=.2},samples=20] {0};
%     \addplot[blue,very thick]{-x*e^(-x)+(1.5+e^(-2))*x};
%     \filldraw[fill=red]  (2,3) circle (0.05);
 
% ]  \end{axis}
% \end{tikzpicture}}
% \end{image}
 
\end{explanation}
 
\end{example}
 
\section*{Text Source}
Felipe L. Martins, Lecture Notes
\end{document}
