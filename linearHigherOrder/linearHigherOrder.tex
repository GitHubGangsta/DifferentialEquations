\documentclass{ximera}

%% You can put user macros here
%% However, you cannot make new environments

\listfiles

\graphicspath{{./}{firstExample/}{secondExample/}}

\usepackage{tikz}
\usepackage{tkz-euclide}
\usepackage{tikz-3dplot}
\usepackage{tikz-cd}
\usetikzlibrary{shapes.geometric}
\usetikzlibrary{arrows}
\usetikzlibrary{decorations.pathmorphing,patterns}
\usetkzobj{all}
\pgfplotsset{compat=1.13} % prevents compile error.

\renewcommand{\vec}[1]{\mathbf{#1}}
\newcommand{\RR}{\mathbb{R}}
\newcommand{\dfn}{\textit}
\newcommand{\dotp}{\cdot}
\newcommand{\id}{\text{id}}
\newcommand\norm[1]{\left\lVert#1\right\rVert}
 
\newtheorem{general}{Generalization}
\newtheorem{initprob}{Exploration Problem}

\tikzstyle geometryDiagrams=[ultra thick,color=blue!50!black]

\usepackage{mathtools}

\title{Introduction to Linear Higher Order Equations}%\label{Module 7-ADEF}


\begin{document}

\begin{abstract}

\end{abstract}

\maketitle

\section*{Introduction to Linear Higher Order Equations}

An $n$th order differential equation is said to be \dfn{linear} if it
can be written in the form
\begin{equation}\label{eq:9.1.1}
y^{(n)}+p_1(x)y^{(n-1)}+\cdots+p_n(x)y=f(x).
\end{equation}
We considered equations of this form with $n=1$ in
Section~2.1 and with $n=2$ in
Chapter~5. In this chapter $n$ is an arbitrary
positive integer.

In this section we sketch the general theory of linear $n$th order
equations. Since this theory has already been discussed for $n=2$ in
Sections~5.1\; and 5.3, we'll omit
proofs.

For convenience, we consider linear differential equations written
as
\begin{equation}\label{eq:9.1.2}
P_0(x)y^{(n)}+P_1(x)y^{(n-1)}+\cdots+P_n(x)y=F(x),
\end{equation}
which can be rewritten as \eqref{eq:9.1.1} on any interval on
which $P_0$ has no zeros, with $p_1=P_1/P_0$, \dots, $p_n=P_n/P_0$ and
$f=F/P_0$. For simplicity, throughout this chapter we'll
abbreviate the left side of \eqref{eq:9.1.2} by $Ly$; that is,
$$
Ly=P_0y^{(n)}+P_1y^{(n-1)}+\cdots+P_ny.
$$
We say that the equation $Ly=F$ is \dfn{normal} on $(a,b)$ if
$P_0$, $P_1$, \dots, $P_n$ and $F$ are continuous on $(a,b)$ and $P_0$ has
no zeros on $(a,b)$. If this is so then $Ly=F$ can be
written as \eqref{eq:9.1.1} with $p_1, \dots, p_n$ and $f$ continuous on
$(a,b)$.

The next theorem is analogous to
Theorem~\ref{thmtype:5.3.1}.

\begin{theorem}\label{thmtype:9.1.1}
Suppose   $Ly=F$ is normal on $(a,b)$, let $x_0$ be a point in
$(a,b),$
and let $k_0$, $k_1$, \dots, $k_{n-1}$ be arbitrary real numbers. Then
the initial value problem
$$
Ly=F, \quad  y(x_0)=k_0,\quad y'(x_0)=k_1,\dots,\quad y^{(n-1)}(x_0)=k_{n-1}
$$
has a unique solution on $(a,b)$.
\end{theorem}

\subsection*{Homogeneous Equations}

Eqn.~\eqref{eq:9.1.2} is said to be \dfn{homogeneous} if $F\equiv0$ and \dfn{nonhomogeneous} otherwise. Since $y\equiv0$ is obviously a
solution of $Ly=0$, we call it the \dfn{trivial} solution. Any other
solution is \dfn{nontrivial}.



If $y_1, y_2, \dots, y_n$ are defined on $(a,b)$ and
$c_1, c_2, \dots, c_n$ are constants, then
\begin{equation}\label{eq:9.1.3}
y=c_1y_1+c_2y_2+\cdots+c_ny_n
\end{equation}
is a \dfn{linear combination} of $\{y_1,y_2\dots,y_n\}$. It's
easy
 to show that if $y_1, y_2, \dots, y_n$ are solutions of
$Ly=0$ on $(a,b)$, then so is any linear combination of
$\{y_1,y_2,\dots,y_n\}$. (See the proof of Theorem~\ref{thmtype:5.1.2}.) We
say that $\{y_1,y_2,\dots,y_n\}$ is a \dfn{fundamental set of
solutions of $Ly=0$ on} $(a,b)$ if every solution of $Ly=0$ on
$(a,b)$ can be written as a linear combination of
$\{y_1,y_2,\dots,y_n\}$, as in \eqref{eq:9.1.3}. In this case we say that
\eqref{eq:9.1.3} is the \dfn{general solution of $Ly=0$ on
$(a,b)$}.


It can be shown (Exercises~\ref{exer:9.1.14} and \ref{exer:9.1.15}) that if
the equation $Ly=0$ is normal on $(a,b)$ then it has
infinitely many fundamental  sets of solutions  on $(a,b)$.
The next definition will help to identify fundamental sets of
solutions of $Ly=0$.

We say that $\{y_1,y_2,\dots,y_n\}$ is \dfn{linearly independent} on
$(a,b)$ if the only constants $c_1, c_2, \dots, c_n$ such that
\begin{equation}\label{eq:9.1.4}
c_1y_1(x)+c_2y_2(x)+\cdots+c_ny_n(x)=0,\quad a<x<b,
\end{equation}
are $c_1=c_2=\cdots=c_n=0$. If \eqref{eq:9.1.4} holds for some set of
constants $c_1, c_2, \dots, c_n$ that are not all zero, then
$\{y_1,y_2,\dots,y_n\}$ is \dfn{linearly dependent} on $(a,b)$

The next theorem is analogous to
Theorem~\ref{thmtype:5.1.3}.

\begin{theorem}\label{thmtype:9.1.2}
If  $Ly=0$ is normal on $(a,b)$, then a set
$\{y_1,y_2,\dots,y_n\}$ of $n$ solutions of $Ly=0$ on $(a,b)$ is a
fundamental set if and only if it's linearly independent on $(a,b)$.
\end{theorem}

\begin{example}\label{example:9.1.1}
The equation
\begin{equation}\label{eq:9.1.5}
x^3y'''-x^2y''-2xy'+6y=0
\end{equation}
is normal and has the solutions $y_1=x^2$, $y_2=x^3$, and $y_3=1/x$ on
$(-\infty,0)$ and $(0,\infty)$.  Show that $\{y_1,y_2,y_3\}$
is linearly independent on $(-\infty, 0)$ and $(0,\infty)$. Then find
the general solution of \eqref{eq:9.1.5} on $(-\infty, 0)$ and
$(0,\infty)$.


\begin{explanation}
Suppose
\begin{equation} \label{eq:9.1.6}
c_1x^2+c_2x^3+\frac{c_3}{x}=0
\end{equation}
on $(0,\infty)$. We must show that $c_1=c_2=c_3=0$. Differentiating
\eqref{eq:9.1.6} twice yields  the system
\begin{equation} \label{eq:9.1.7}
\begin{array}{rcl}
c_1x^2+c_2x^3+\frac{c_3}{x}&=&0\\
2c_1x+3c_2x^2-\frac{c_3}{x^2}&=&0\\
2c_1+6c_2x+\frac{2c_3}{x^3}&=&0.
\end{array}
\end{equation}
If \eqref{eq:9.1.7} holds for all $x$ in $(0,\infty)$, then it
certainly  holds at $x=1$;  therefore,
\begin{equation} \label{eq:9.1.8}
\begin{array}{rcl}
c_1+c_2+c_3&=&0\\
2c_1+3c_2-c_3&=&0\\
2c_1+6c_2+2c_3&=&0.
\end{array}
\end{equation}
By solving this system directly, you can verify that it has only the
trivial solution $c_1=c_2=c_3=0$;     however, for our purposes it's more
useful to recall from linear algebra that a homogeneous linear system
of $n$ equations in $n$ unknowns has only the trivial solution if its
determinant is nonzero. Since the determinant of \eqref{eq:9.1.8} is
$$
\begin{vmatrix}1&1&1\\2&3&-1\\2&6&2\end{vmatrix}=
\begin{vmatrix}1&0&0\\2&1&-3\\2&4&0\end{vmatrix}=12,
$$
it follows that \eqref{eq:9.1.8} has only the trivial solution, so
$\{y_1,y_2,y_3\}$ is linearly independent on $(0,\infty)$. Now
Theorem~\ref{thmtype:9.1.2} implies that
$$
y=c_1x^2+c_2x^3+\frac{c_3}{x}
$$
is the general solution of \eqref{eq:9.1.5} on $(0,\infty)$.
To see that this is also true
 on $(-\infty,0)$,  assume that \eqref{eq:9.1.6} holds on
$(-\infty,0)$. Setting $x=-1$ in \eqref{eq:9.1.7} yields
\begin{eqnarray*}
c_1-c_2-c_3&=&0\\
-2c_1+3c_2-c_3&=&0\\
2c_1-6c_2-2c_3&=&0.
\end{eqnarray*}
Since the determinant of this system is
$$
\begin{vmatrix}1&-1&-1\\-2&3&-1\\2&-6&-2\end{vmatrix}=
\begin{vmatrix}1&0&0\\-2&1&-3\\2&-4&0\end{vmatrix}=-12,
$$
it follows that $c_1=c_2=c_3=0$; that is, $\{y_1,y_2,y_3\}$
is linearly independent on $(-\infty,0)$.

\end{explanation}
\end{example}

\begin{example}\label{example:9.1.2}
The equation
\begin{equation}\label{eq:9.1.9}
y^{(4)}+y'''-7y''-y'+6y=0
\end{equation}
is normal and has the solutions $y_1=e^x$, $y_2=e^{-x}$, $y_3=e^{2x}$,
and $y_4=e^{-3x}$ on $(-\infty,\infty)$. (Verify.) Show that
$\{y_1,y_2,y_3,y_4\}$ is linearly independent on $(-\infty,\infty)$.
Then find the general solution of \eqref{eq:9.1.9}.

\begin{explanation}
Suppose   $c_1$, $c_2$, $c_3$, and $c_4$
are constants such that
\begin{equation} \label{eq:9.1.10}
c_1e^x+c_2e^{-x}+c_3e^{2x}+c_4e^{-3x}=0
\end{equation}
for all $x$. We must show that $c_1=c_2=c_3=c_4=0$. Differentiating
\eqref{eq:9.1.10} three times yields the system
\begin{equation} \label{eq:9.1.11}
\begin{array}{rcl}
c_1e^x+c_2e^{-x}+c_3e^{2x}+c_4e^{-3x}&=&0\\
c_1e^x-c_2e^{-x}+2c_3e^{2x}-3c_4e^{-3x}&=&0\\
c_1e^x+c_2e^{-x}+4c_3e^{2x}+9c_4e^{-3x}&=&0\\
c_1e^x-c_2e^{-x}+8c_3e^{2x}-27c_4e^{-3x}&=&0.
\end{array}
\end{equation}
If \eqref{eq:9.1.11} holds for all $x$, then it certainly holds for $x=0$.
Therefore
$$
\begin{array}{rcl}
c_1+c_2+c_3+c_4&=&0\\
c_1-c_2+2c_3-3c_4&=&0\\
c_1+c_2+4c_3+9c_4&=&0\\
c_1-c_2+8c_3-27c_4&=&0.
\end{array}
$$
The determinant of this system is

\begin{equation} \label{eq:9.1.12}
\begin{array}{ll}
&\begin{vmatrix}1&1&1&1\\1&-1&2&-3\\1&1&4&9\\
1&-1&8&-27\end{vmatrix}=
\begin{vmatrix}1&1&1&1\\0&-2&1&-4\\0&0&3&8\\
0&-2&7&-28\end{vmatrix}\\
=&\begin{vmatrix}-2&1&-4\\0&3&8\\
-2&7&-28\end{vmatrix}
=\begin{vmatrix}-2&1&-4\\0&3&8
\\0&6&-24\end{vmatrix}\\
=&-2\begin{vmatrix}3&8\\6&-24\end{vmatrix}\\
=&240
\end{array}
\end{equation}

so the system has only the trivial solution $c_1=c_2=c_3=c_4=0$.
Now Theorem~\ref{thmtype:9.1.2} implies that
$$
y=c_1e^x+c_2e^{-x}+c_3e^{2x}+c_4e^{-3x}
$$
is the general solution of \eqref{eq:9.1.9}.

\end{explanation}
\end{example}

\subsection*{The Wronskian}

We can use the method used in Examples~\ref{example:9.1.1} and
\ref{example:9.1.2} to test $n$ solutions $\{y_1,y_2,\dots,y_n\}$ of any
$n$th order equation $Ly=0$ for linear independence on an interval
$(a,b)$ on which the equation is normal. Thus, if $c_1, c_2 ,\dots, c_n$ are constants such that
$$
c_1y_1+c_2y_2+\cdots+c_ny_n=0,\quad a<x<b,
$$
then differentiating $n-1$ times leads to the $n\times n$
system of equations
\begin{equation} \label{eq:9.1.13}
\begin{array}{rcl}
c_1y_1(x)+c_2y_2(x)+&\cdots&+c_ny_n(x)=0\\
c_1y_1'(x)+c_2y_2'(x)+&\cdots&+c_ny_n'(x)=0\\
&\vdots&
&&\\
c_1y_1^{(n-1)}(x)+c_2y_2^{(n-1)}(x)+&\cdots&+c_ny_n^{(n-1)}(x)
=0
\end{array}
\end{equation}
for $c_1, c_2, \dots, c_n$. For a fixed $x$,  the determinant of this
system is
$$
W(x)=\begin{vmatrix}
y_1(x)&y_2(x)&\cdots&y_n(x)\\
y'_1(x)&y'_2(x)&\cdots&y_n'(x)\\
\vdots&\vdots&\ddots&\vdots\\
y_1^{(n-1)}(x)&y_2^{(n-1)}(x)&\cdots&y_n^{(n-1)}(x)
\end{vmatrix}
$$
We call this determinant the
\href{http://www-history.mcs.st-and.ac.uk/Mathematicians/Wronski.html}{Wronskian} of
$\{y_1,y_2,\dots,y_n\}$. If $W(x)\neq 0$ for some $x$ in $(a,b)$ then
 the system \eqref{eq:9.1.13} has only the trivial solution
$c_1=c_2=\cdots=c_n=0$, and Theorem~\ref{thmtype:9.1.2} implies that
$$
y=c_1y_1+c_2y_2+\cdots+c_ny_n
$$
is the general solution of $Ly=0$ on $(a,b)$.


The next theorem generalizes Theorem~\ref{thmtype:5.1.4}.
%The proof is sketched in (Exercises~\ref{exer:9.1.17}--\ref{exer:9.1.20}).

\begin{theorem}\label{thmtype:9.1.3}
 Suppose the   homogeneous linear $n$th order equation
\begin{equation}\label{eq:9.1.14}
P_0(x)y^{(n)}+P_1(x)y^{n-1}+\cdots+P_n(x)y=0
\end{equation}
is normal on $(a,b),$ let $y_1, y_2, \dots, y_n$ be solutions of
\eqref{eq:9.1.14} on $(a,b)$, and let $x_0$ be in $(a,b)$. Then the
Wronskian of $\{y_1,y_2,\dots,y_n\}$ is given by
\begin{equation}  \label{eq:9.1.15}
W(x)=W(x_0)\exp\left\{-\int^x_{x_0}\frac{P_1(t)}{P_0(t)}\,
dt\right\},\quad a<x<b.
\end{equation}
Therefore,  either $W$ has no zeros in  $(a,b)$ or $W\equiv0$
on  $(a,b)$.
\end{theorem}


Formula \eqref{eq:9.1.15} is
 \href{http://www-history.mcs.st-and.ac.uk/Mathematicians/Abel.html}{Abel's formula}.

The next theorem is analogous to
Theorem~\ref{thmtype:5.1.6}.

\begin{theorem}\label{thmtype:9.1.4}
Suppose $Ly=0$ is normal on $(a,b)$ and let $y_1, y_2, \dots, y_n$
be $n$ solutions of $Ly=0$ on $(a,b)$. Then the following statements
are equivalent;  that is, they are either all true or all false:
\begin{enumerate}
\item\label{item:9.1.4a} % (a)
The general solution of $Ly=0$ on $(a,b)$ is
$y=c_1y_1+c_2y_2+\cdots+c_ny_n$.
\item\label{item:9.1.4b} % (b)
 $\{y_1,y_2,\dots,y_n\}$ is a fundamental set of solutions of
$Ly=0$ on $(a,b)$.
\item\label{item:9.1.4c} % (c)
$\{y_1,y_2,\dots,y_n\}$ is linearly independent on $(a,b)$.
\item\label{item:9.1.4d} % (d)
The Wronskian of $\{y_1,y_2,\dots,y_n\}$ is nonzero at some point in
$(a,b)$.
\item\label{item:9.1.4e} % (e)
The Wronskian of $\{y_1,y_2,\dots,y_n\}$ is nonzero at all points in
$(a,b)$.
\end{enumerate}
\end{theorem}

\begin{example}\label{example:9.1.3}
In Example~\ref{example:9.1.1} we saw that the solutions $y_1=x^2$,
$y_2=x^3$, and $y_3=1/x$ of
$$
x^3y'''-x^2y''-2xy'+6y=0
$$
are linearly independent on $(-\infty,0)$ and $(0,\infty)$. Calculate
the Wronskian  of $\{y_1,y_2,y_3\}$.

\begin{explanation}
If $x\neq 0$,  then
$$
W(x)=\begin{vmatrix}
x^2&x^3&1/x\\
2x&3x^2&-1/x^2\\
2&6x&2/x^3\end{vmatrix}
=2x^3\begin{vmatrix}
1&x&1/x^3\\2&3x&-1/x^3
\\1&3x&1/x^3\end{vmatrix},
$$
where we factored $x^2$, $x$, and $2$ out of the first, second, and
third  rows of
$W(x)$, respectively. Adding the second row of the last determinant
to the first and third rows yields
$$
W(x)=2x^3\begin{vmatrix}
3&4x&0\\
2&3x&-1/x^3\\
3&6x&0\end{vmatrix}
=2x^3\left(\frac{1}{x^3}\right)\begin{vmatrix}3&4x\\3&6x
\end{vmatrix}=12x
$$
Therefore $W(x)\neq 0$ on $(-\infty,0)$ and $(0,\infty)$.
\end{explanation}
\end{example}

\begin{example}\label{example:9.1.4} In
Example~\ref{example:9.1.2} we saw that the solutions $y_1=e^x$,
$y_2=e^{-x}$,
$y_3=e^{2x}$, and $y_4=e^{-3x}$ of
$$
y^{(4)}+y'''-7y''-y'+6y=0
$$
are linearly independent on every open interval.
Calculate the Wronskian  of $\{y_1,y_2,y_3,y_4\}$.

\begin{explanation}
For all $x$,
$$
W(x)=\begin{vmatrix}
e^x&e^{-x}&e^{2x}&e^{-3x}\\
e^x&-e^{-x}&2e^{2x}&-3e^{-3x}\\
e^x&e^{-x}&4e^{2x}&9e^{-3x}\\
e^x&-e^{-x}&8e^{2x}&-27e^{-3x}
\end{vmatrix}
$$
Factoring the exponential common factor from each row yields
$$
W(x)=e^{-x}\begin{vmatrix}1&1&1&1\\1&-1&2&-3\\1&1&4&9\\
1&-1&8&-27\end{vmatrix}=240e^{-x}
$$
from \eqref{eq:9.1.12}.
\end{explanation}
\end{example}

\begin{remark}
Under the assumptions of Theorem~\ref{thmtype:9.1.4}, it isn't
necessary to obtain a formula for $W(x)$. Just evaluate $W(x)$
at a convenient point in $(a,b)$, as we did in
Examples~\ref{example:9.1.1} and \ref{example:9.1.2}.
\end{remark}



\begin{theorem}\label{thmtype:9.1.5}
Suppose $c$ is in $(a,b)$ and $\alpha_{1}, \alpha_{2}, \dots$,
are real numbers, not all zero.
Under the assumptions of Theorem~\ref{thmtype:10.3.3}, suppose
$y_{1}$ and $y_{2}$ are  solutions  of \eqref{eq:5.1.35}   such that
\begin{equation} \label{eq:9.1.16}
\alpha y_{i}(c)+ y_{i}'(c)+\cdots +y_{i}^{(n-1)}(c)=0,\quad 1\leq i\leq n.
\end{equation}
Then $\{y_{1},y_{2},\dots y_{n}\}$ isn't  linearly independent on $(a,b).$
\end{theorem}


\begin{proof} Since $\alpha_{1}, \alpha_{2}, \dots, \alpha_{n}$
are not all zero,
\eqref{eq:9.1.14} implies that
$$
\begin{vmatrix}
y_{1}(c)&y_{1}'(c)&\cdots&y_{1}^{(n-1)}(c)\\
y_{2}(c)&y_{2}'(c)&\cdots&y_{2}^{(n-1)}(c)\\
\vdots&\vdots&\ddots&\vdots\\
y_{n}(c)&y_{n}'(c)&\cdots&y_{n}^{(n-1)}(c)\\
\end{vmatrix}=0,
$$
so
$$
\begin{vmatrix}
y_{1}(c)&y_{2}(c)&\cdots& y_{n}(c)\\
y_{1}'(c)&y_{2}'(c)&\cdots& y_{n}'(c)\\
\vdots&\vdots&\ddots&\vdots\\
y_{1}^{(n-1)}(c)&y_{2}^{(n-1)}(c)(c)&\cdots& y_{n}^{(n-1)}(c)(c)\\
\end{vmatrix}=0
$$

and Theorem~\ref{thmtype:9.1.4} implies the stated conclusion.
\end{proof}




\subsection*{General Solution of a Nonhomogeneous Equation}

The next theorem is analogous to
Theorem~\ref{thmtype:5.3.2}.
It shows how to find the general solution of $Ly=F$ if we know a particular
solution of $Ly=F$ and a fundamental set of solutions of the \dfn{complementary equation} $Ly=0$.

\begin{theorem}\label{thmtype:9.1.6}
Suppose $Ly=F$ is normal on $(a,b)$. Let $y_p$ be a particular
solution of $Ly=F$ on $(a,b)$, and let $\{y_1,y_2,\dots,y_n\}$ be a
fundamental set of solutions of the complementary equation $Ly=0$ on
$(a,b)$. Then $y$ is a solution of $Ly=F$ on $(a,b)$ if and only if
$$
y=y_p+c_1y_1+c_2y_2+\cdots+c_ny_n,
$$
where $c_1,c_2,\dots,c_n$  are constants.
\end{theorem}

The next theorem is analogous to Theorem~\ref{thmtype:5.3.2}.

\begin{theorem}[The Principle of
Superposition]\label{thmtype:9.1.7}
Suppose for each $i=1, 2, \dots, r$, the function $y_{p_i}$ is a
particular solution of $Ly=F_i$ on  $(a,b)$. Then
$$
y_p=y_{p_1}+y_{p_2}+\cdots+y_{p_r}
$$
is a particular  solution of
$$
Ly=F_1(x)+F_2(x)+\cdots+F_r(x)
$$
on $(a,b)$.
\end{theorem}

We'll apply  Theorems~\ref{thmtype:9.1.6} and \ref{thmtype:9.1.7}
throughout the rest of this chapter.


\section*{Text Source}
Trench, William F., "Elementary Differential Equations" (2013). Faculty Authored and Edited Books \& CDs. 8. (CC-BY-NC-SA)

\href{https://digitalcommons.trinity.edu/mono/8/}{https://digitalcommons.trinity.edu/mono/8/}


\end{document}