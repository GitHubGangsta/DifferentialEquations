\documentclass{ximera}

%% You can put user macros here
%% However, you cannot make new environments

\listfiles

\graphicspath{{./}{firstExample/}{secondExample/}}

\usepackage{tikz}
\usepackage{tkz-euclide}
\usepackage{tikz-3dplot}
\usepackage{tikz-cd}
\usetikzlibrary{shapes.geometric}
\usetikzlibrary{arrows}
\usetikzlibrary{decorations.pathmorphing,patterns}
\usetkzobj{all}
\pgfplotsset{compat=1.13} % prevents compile error.

\renewcommand{\vec}[1]{\mathbf{#1}}
\newcommand{\RR}{\mathbb{R}}
\newcommand{\dfn}{\textit}
\newcommand{\dotp}{\cdot}
\newcommand{\id}{\text{id}}
\newcommand\norm[1]{\left\lVert#1\right\rVert}
 
\newtheorem{general}{Generalization}
\newtheorem{initprob}{Exploration Problem}

\tikzstyle geometryDiagrams=[ultra thick,color=blue!50!black]

\usepackage{mathtools}




\title{Reduction of Order}


\begin{document}

\begin{abstract}

\end{abstract}

\maketitle

\section*{Reduction of Order}

In this section we give a method for finding the general
solution of
\begin{equation} \label{eq:5.6.1}
P_0(x)y''+P_1(x)y'+P_2(x)y=F(x)
\end{equation}
if we know  a nontrivial solution $y_1$ of the complementary equation
\begin{equation} \label{eq:5.6.2}
P_0(x)y''+P_1(x)y'+P_2(x)y=0.
\end{equation}
The method is called \dfn{reduction of order} because it reduces
the task of solving \eqref{eq:5.6.1} to solving a first order equation.
Unlike the method of undetermined coefficients, it does not require
$P_0$, $P_1$, and $P_2$ to be constants, or $F$ to be of any special
form.

By now you shouldn't be surprised that we look for
solutions of \eqref{eq:5.6.1} in the form
\begin{equation} \label{eq:5.6.3}
y=uy_1
\end{equation}
where $u$ is to be determined so that $y$ satisfies \eqref{eq:5.6.1}.
Substituting \eqref{eq:5.6.3} and
$$\begin{array}{rcl}
y'&=& u'y_1+uy_1' \\
y''&=& u''y_1+2u'y_1'+uy_1''
\end{array}$$
into \eqref{eq:5.6.1} yields
$$
P_0(x)(u''y_1+2u'y_1'+uy_1'')+P_1(x)(u'y_1+uy_1')+P_2(x)uy_1=F(x).
$$
Collecting the coefficients of $u$, $u'$, and $u''$ yields
\begin{equation} \label{eq:5.6.4}
(P_0y_1)u''+(2P_0y_1'+P_1y_1)u'+(P_0y_1''+P_1y_1'+P_2y_1)
u=F.
\end{equation}
However, the coefficient of $u$ is zero, since $y_1$ satisfies
\eqref{eq:5.6.2}. Therefore \eqref{eq:5.6.4} reduces to
\begin{equation} \label{eq:5.6.5}
Q_0(x)u''+Q_1(x)u'=F,
\end{equation}
with
$$
Q_0=P_0y_1 \quad\mbox{and}\quad Q_1=2P_0y_1'+P_1y_1.
$$
(It isn't worthwhile to memorize the formulas for $Q_0$ and $Q_1$!)
Since \eqref{eq:5.6.5} is a linear first order equation in $u'$, we can
solve it for $u'$ by variation of parameters as in
Section~1.2, integrate the solution to obtain $u$, and then
obtain $y$ from \eqref{eq:5.6.3}.

\begin{example}\label{example:5.6.1} 
\begin{enumerate}
\item \label{item:5.6.1a} % (a)
 Find the general solution of
\begin{equation} \label{eq:5.6.6}
xy''-(2x+1)y'+(x+1)y=x^2,
\end{equation}
given that $y_1=e^x$ is a solution of the complementary equation
\begin{equation} \label{eq:5.6.7}
xy''-(2x+1)y'+(x+1)y=0.
\end{equation}
\item \label{item:5.6.1b}% (b)
As a byproduct of \ref{item:5.6.1a}, find a fundamental set of solutions of
\eqref{eq:5.6.7}.
\end{enumerate}

\begin{explanation} \ref{item:5.6.1a}
If
$y=ue^x$, then $y'=u'e^x+ue^x$ and
$y''=u''e^x+2u'e^x+ue^x$, so
\begin{eqnarray*}
xy''-(2x+1)y'+(x+1)y&=&x(u''e^x+2u'e^x+ue^x)\\
&&-(2x+1)(u'e^x+ue^x)+(x+1)ue^x\\
&=&(xu''-u')e^x.
\end{eqnarray*}
Therefore $y=ue^x$ is a solution of \eqref{eq:5.6.6} if and only if
$$
(xu''-u')e^x=x^2,
$$
which is a first order equation in $u'$. We rewrite it as
\begin{equation} \label{eq:5.6.8}
u''-\frac{u'}{x}=xe^{-x}.
\end{equation}
To focus on how we apply variation of parameters to this equation, we
temporarily write $z=u'$, so that \eqref{eq:5.6.8} becomes
\begin{equation} \label{eq:5.6.9}
z'-\frac{z}{x}=xe^{-x}.
\end{equation}
We leave it to you to show (by separation of variables) that $z_1=x$
is a solution of the complementary equation
$$
z'-\frac{z}{x}=0
$$
for \eqref{eq:5.6.9}. By applying variation of parameters as in
Section~1.2, we can now see that every solution of
\eqref{eq:5.6.9} is of the form
$$
z=vx\quad\mbox{where}\quad
v'x=xe^{-x}, \quad\mbox{so}\quad v'=e^{-x} \quad\mbox{and}\quad
v=-e^{-x}+C_1.
$$
Since $u'=z=vx$,   $u$ is a solution of \eqref{eq:5.6.8} if and only if
$$
u'=vx=-xe^{-x}+C_1x.
$$
Integrating this yields
$$
u=(x+1)e^{-x}+\frac{C_1}{2}x^2+C_2.
$$
Therefore the general solution of  \eqref{eq:5.6.6} is
\begin{equation} \label{eq:5.6.10}
y=ue^x=x+1+\frac{C_1}{2}x^2e^x+C_2e^x.
\end{equation}

\ref{item:5.6.1b} By letting $C_1=C_2=0$ in \eqref{eq:5.6.10}, we see that
$y_{p_1}=x+1$ is a solution of \eqref{eq:5.6.6}. By letting $C_1=2$ and
$C_2=0$, we see that $y_{p_2}=x+1+x^2e^x$ is also a solution of
\eqref{eq:5.6.6}. Since the difference of two solutions of \eqref{eq:5.6.6} is
a solution of \eqref{eq:5.6.7},
$y_2=y_{p_1}-y_{p_2}=x^2e^x$ is a solution of \eqref{eq:5.6.7}. Since
$y_2/y_1$ is nonconstant and we already know  that $y_1=e^x$ is a
solution of \eqref{eq:5.6.6}, Theorem~\ref{thmtype:5.1.6} implies that
$\{e^x,x^2e^x\}$ is a fundamental set of solutions of
\eqref{eq:5.6.7}.
\end{explanation}
\end{example}

Although \eqref{eq:5.6.10} is a correct form for the general solution of
\eqref{eq:5.6.6}, it's silly to leave the arbitrary coefficient of
$x^2e^x$ as $C_1/2$ where $C_1$ is an arbitrary constant. Moreover, it's
sensible to make the subscripts of the coefficients of $y_1=e^x$ and
$y_2=x^2e^x$ consistent with the subscripts of the functions
themselves. Therefore we rewrite \eqref{eq:5.6.10} as $$
y=x+1+c_1e^x+c_2x^2e^x $$ by simply renaming the arbitrary constants.
We'll also do this in the next two examples, and in the answers to
the exercises.

\begin{example}\label{example:5.6.2}
\begin{enumerate}
\item \label{item:5.6.2a} %(a)
Find the general solution of
$$
x^2y''+xy'-y=x^2+1,
$$
 given that $y_1=x$ is a solution of the complementary
equation
\begin{equation}  \label{eq:5.6.11}
x^2y''+xy'-y=0.
\end{equation}
  As a byproduct of this result, find a fundamental set of
solutions of \eqref{eq:5.6.11}.

\item \label{item:5.6.2b}%(b)
Solve the initial value problem
\begin{equation}  \label{eq:5.6.12}
x^2y''+xy'-y=x^2+1, \quad   y(1)=2,\;  y'(1)=-3.
\end{equation}
\end{enumerate}


\begin{explanation}\ref{item:5.6.2a} If $y=ux$, then $y'=u'x+u$ and $y''=u''x+2u'$, so
\begin{eqnarray*}
x^2y''+xy'-y&=&x^2(u''x+2u')+x(u'x+u)-ux\\
&=&x^3u''+3x^2u'.
\end{eqnarray*}
Therefore $y=ux$ is a solution of \eqref{eq:5.6.12} if and only if
$$
x^3u''+3x^2u'=x^2+1,
$$
which is a first order equation in $u'$.
We rewrite it as
\begin{equation} \label{eq:5.6.13}
u''+\frac{3}{x}u'=\frac{1}{x}+\frac{1}{x^3}.
\end{equation}
To focus on how we apply variation of parameters to this equation, we
temporarily write $z=u'$, so that \eqref{eq:5.6.13} becomes
\begin{equation} \label{eq:5.6.14}
z'+\frac{3}{x}z=\frac{1}{x}+\frac{1}{x^3}.
\end{equation}
We leave it to you to show by separation of variables that
$z_1=1/x^3$ is a solution of the complementary equation
$$
z'+\frac{3}{x}z=0
$$
for \eqref{eq:5.6.14}. By variation of parameters, every solution of
\eqref{eq:5.6.14} is of the form
$$
z=\frac{v}{x^3}\quad\mbox{where}\quad
\frac{v'}{x^3}=\frac{1}{x}+\frac{1}{x^3}, \quad\mbox{so}\quad
 v'=x^2+1 \quad\mbox{and}\quad v=\frac{x^3}{3}+x+C_1.
$$
Since $u'=z=v/x^3$,   $u$ is a solution of \eqref{eq:5.6.14} if and only
if
$$
u'=\frac{v}{x^3}=\frac{1}{3}+\frac{1}{x^2}+\frac{C_1}{x^3}.
$$
Integrating this yields
$$
u=\frac{x}{3}-\frac{1}{x}-\frac{C_1}{2x^2}+C_2.
$$
Therefore the general solution of  \eqref{eq:5.6.12} is
\begin{equation} \label{eq:5.6.15}
y=ux=\frac{x^2}{3}-1-\frac{C_1}{2x}+C_2x.
\end{equation}
Reasoning as in  the solution of
Example~\ref{example:5.6.1}\ref{item:5.6.1a}, we  conclude that $y_1=x$ and $y_2=1/x$
form a fundamental set of solutions for \eqref{eq:5.6.11}.

As we explained above, we rename the constants in \eqref{eq:5.6.15} and
rewrite it as
\begin{equation}  \label{eq:5.6.16}
y=\frac{x^2}{3}-1+c_1x+\frac{c_2}{x}.
\end{equation}

\ref{item:5.6.2b}   Differentiating  \eqref{eq:5.6.16} yields
\begin{equation} \label{eq:5.6.17}
y'=\frac{2x}{3}+c_1-\frac{c_2}{x^2}.
\end{equation}
 Setting $x=1$ in  \eqref{eq:5.6.16} and  \eqref{eq:5.6.17} and imposing the
initial conditions $y(1)=2$ and $y'(1)=-3$  yields
\begin{eqnarray*}
c_1+c_2&=& \frac{8}{3} \\
c_1-c_2&=& -\frac{11}{3}.
\end{eqnarray*}
Solving these  equations yields $c_1=-1/2$, $c_2=19/6$.
Therefore the solution of \eqref{eq:5.6.12} is
$$
y=\frac{x^2}{3}-1-\frac{x}{2}+\frac{19}{6x}.
$$
\end{explanation}
\end{example}


Using reduction of order to  find the general solution of
a homogeneous linear second order  equation leads to a
homogeneous linear first  order  equation in $u'$ that can be solved
by separation of variables. The next example illustrates this.

\begin{example}\label{example:5.6.3}
 Find the general solution and a fundamental set  of solutions of
\begin{equation}  \label{eq:5.6.18}
x^2y''-3xy'+3y=0,
\end{equation}
 given that $y_1=x$ is a solution.


\begin{explanation}
 If $y=ux$ then $y'=u'x+u$ and $y''=u''x+2u'$, so
\begin{eqnarray*}
x^2y''-3xy'+3y&=&x^2(u''x+2u')-3x(u'x+u)+3ux\\
&=&x^3u''-x^2u'.
\end{eqnarray*}
Therefore $y=ux$ is a solution of \eqref{eq:5.6.18} if and only if
$$
x^3u''-x^2u'=0.
$$
Separating the variables $u'$ and $x$ yields
$$
\frac{u''}{u'}=\frac{1}{x},
$$
so
$$
\ln|u'|=\ln|x|+k,\quad\mbox{or, equivalently,}\quad u'=C_1x.
$$
Therefore
$$
u=\frac{C_1}{2}x^2+C_2,
$$
so the general solution of \eqref{eq:5.6.18} is
$$
y=ux=\frac{C_1}{2}x^3+C_2x,
$$
which we rewrite as
$$
y=c_1x+c_2x^3.
$$
Therefore $\{x,x^3\}$ is  a fundamental set of solutions of
\eqref{eq:5.6.18}.
\end{explanation}
\end{example}


\section*{Text Source}
Trench, William F., "Elementary Differential Equations" (2013). Faculty Authored and Edited Books \& CDs. 8. (CC-BY-NC-SA)

\href{https://digitalcommons.trinity.edu/mono/8/}{https://digitalcommons.trinity.edu/mono/8/}

\end{document}


