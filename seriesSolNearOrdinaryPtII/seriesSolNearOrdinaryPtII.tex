\documentclass{ximera}

%% You can put user macros here
%% However, you cannot make new environments

\listfiles

\graphicspath{{./}{firstExample/}{secondExample/}}

\usepackage{tikz}
\usepackage{tkz-euclide}
\usepackage{tikz-3dplot}
\usepackage{tikz-cd}
\usetikzlibrary{shapes.geometric}
\usetikzlibrary{arrows}
\usetikzlibrary{decorations.pathmorphing,patterns}
\usetkzobj{all}
\pgfplotsset{compat=1.13} % prevents compile error.

\renewcommand{\vec}[1]{\mathbf{#1}}
\newcommand{\RR}{\mathbb{R}}
\newcommand{\dfn}{\textit}
\newcommand{\dotp}{\cdot}
\newcommand{\id}{\text{id}}
\newcommand\norm[1]{\left\lVert#1\right\rVert}
 
\newtheorem{general}{Generalization}
\newtheorem{initprob}{Exploration Problem}

\tikzstyle geometryDiagrams=[ultra thick,color=blue!50!black]

\usepackage{mathtools}

\title{Series Solutions Near an Ordinary Point II}%\label{Module 7-ADEF}


\begin{document}

\begin{abstract}
We consider the utilization of power series to determine solutions to more general differential equations.
\end{abstract}

\maketitle

\section*{Series Solutions Near an Ordinary Point II}

In this section we continue to find series solutions
$$
y=\sum_{n=0}^\infty a_n(x-x_0)^n
$$
of initial value problems
\begin{equation} \label{eq:7.3.1}
P_0(x)y''+P_1(x)y'+P_2(x)y=0,\quad y(x_0)=a_0,\quad y'(x_0)=a_1,
\end{equation}
where $P_0,P_1$, and $P_2$ are polynomials and $P_0(x_0)\neq0$,
so $x_0$ is an ordinary point of \eqref{eq:7.3.1}.  However, here we
consider cases where the differential equation in \eqref{eq:7.3.1}
is not of the form
$$
\left(1+\alpha(x-x_0)^2\right)y''+\beta(x-x_0) y'+\gamma y=0,
$$
so Theorem~\ref{thmtype:7.2.2} does not apply, and the computation of the
coefficients $\{a_n\}$ is more complicated. For the equations
considered here it's difficult or impossible to obtain an explicit
formula for $a_n$ in terms of $n$. Nevertheless, we can calculate as
many coefficients as we wish. The next three examples illustrate
this.

\begin{example}\label{example:7.3.1}
Find the coefficients $a_0, \dots, a_7$ in the  series solution
$y=\sum^\infty_{n=0}
a_nx^n$  of the initial value problem
\begin{equation} \label{eq:7.3.2}
(1+x+2x^2)y''+(1+7x)y'+2y=0,\quad y(0)=-1,\quad y'(0)=-2.
\end{equation}


\begin{explanation}
Here
$$
Ly=(1+x+2x^2)y''+(1+7x)y'+2y.
$$
The zeros $(-1\pm i\sqrt7)/4$ of $P_0(x)=1+x+2x^2$ have absolute value
$1/\sqrt2$, so Theorem~\ref{thmtype:7.2.2} implies that the series
solution
converges to the solution of \eqref{eq:7.3.2} on $(-1/\sqrt2,1/\sqrt2)$.
Since
$$
y=\sum^\infty_{n=0} a_nx^n,\quad y'=\sum^\infty_{n=1} n
a_nx^{n-1}\quad\mbox{ and }\quad  y''=\sum^\infty_{n=2}n(n-1)a_nx^{n-2},
$$
\begin{eqnarray*}
Ly&=&\sum^\infty_{n=2}n(n-1)a_nx^{n-2}+\sum^\infty_{n=2}n(n-1)a_nx^{n-1}
+2\sum^\infty_{n=2}n(n-1)a_nx^n\\
&&+\sum^\infty_{n=1}na_nx^{n-1}+7\sum^\infty_{n=1}na_nx^n+2\sum^\infty_{n=0}
a_nx^n.
\end{eqnarray*}
Shifting indices so  the general term in each
series is a constant multiple of $x^n$ yields
\begin{eqnarray*}
Ly&=&\sum^\infty_{n=0}(n+2)(n+1)a_{n+2}x^n+\sum^\infty_{n=0}(n+1)na_{n+1}x^n
+2\sum^\infty_{n=0}n(n-1)a_nx^n\\
&&+\sum^\infty_{n=0}(n+1)a_{n+1}x^n+7\sum^\infty_{n=0}na_nx^n+
2\sum^\infty_{n=0}a_nx^n
=\sum^\infty_{n=0}b_nx^n,
\end{eqnarray*}
where
$$
b_n=(n+2)(n+1)a_{n+2}+(n+1)^2a_{n+1}+(n+2)(2n+1)a_n.
$$
Therefore $y=\sum^\infty_{n=0}a_nx^n$ is a solution of $Ly=0$
if and only if
\begin{equation} \label{eq:7.3.3}
a_{n+2}=-\frac{n+1}{n+2}\,a_{n+1}-\frac{2n+1}{n+1}\,a_n,\,n\geq0.
\end{equation}
From the initial conditions in \eqref{eq:7.3.2}, $a_0=y(0)=-1$ and
$a_1=y'(0)=-2$.
Setting $n=0$  in \eqref{eq:7.3.3} yields
$$
a_2=-\frac{1}{2}a_1-a_0=-\frac{1}{2}(-2)-(-1)=2.
$$
Setting $n=1$  in \eqref{eq:7.3.3} yields
$$
a_3=-\frac{2}{3}a_2-\frac{3}{2}a_1=-\frac{2}{3}(2)-\frac{3}{2}(-2)=\frac{5}{3}.
$$
We leave it to you to compute $a_4,a_5,a_6,a_7$ from \eqref{eq:7.3.3} and
show that
$$
y=-1-2x+2x^2+\frac{5}{3}x^3-\frac{55}{12}x^4+\frac{3}{4}x^5+\frac{61}{8}x^6-
\frac{443}{56}x^7+\cdots .
$$
We also leave it to you %(Exercise~\ref{exer:7.3.13}) 
to verify numerically
that the Taylor polynomials $T_N(x)=\sum_{n=0}^Na_nx^n$ converge
to the solution of \eqref{eq:7.3.2}
on $(-1/\sqrt2,1/\sqrt2)$.
\end{explanation}
\end{example}

\begin{example}\label{example:7.3.2}
Find  the coefficients $a_0, \dots, a_5$ in the series
solution
$$
y=\sum^\infty_{n=0} a_n(x+1)^n
$$
 of the initial value problem
\begin{equation} \label{eq:7.3.4}
(3+x)y''+(1+2x)y'-(2-x)y=0,\quad y(-1)=2,\quad y'(-1)=-3.
\end{equation}
\begin{explanation}
Since the desired series is in powers of $x+1$ we rewrite
the differential equation in \eqref{eq:7.3.4} as $Ly=0$, with
$$
Ly=\left(2+(x+1)\right)y''-\left(1-2(x+1)\right)y'-\left(3-(x+1)\right)y.
$$
Since
$$
y=\sum^\infty_{n=0} a_n(x+1)^n,\quad y'=\sum^\infty_{n=1} n
a_n(x+1)^{n-1}\quad\mbox{ and }\quad
y''=\sum^\infty_{n=2}n(n-1)a_n(x+1)^{n-2},
$$
\begin{eqnarray*}
Ly&=&2\sum^\infty_{n=2}n(n-1)a_n(x+1)^{n-2}+\sum^\infty_{n=2}n(n-1)a_n(x+1)^{n-1}
\\&&-\sum^\infty_{n=1}na_n(x+1)^{n-1}+2\sum^\infty_{n=1}na_n(x+1)^n\\
&&-3\sum^\infty_{n=0}a_n(x+1)^n+\sum_{n=0}^\infty a_n(x+1)^{n+1}.
\end{eqnarray*}
Shifting indices so that  the general term in each
series is a constant multiple of $(x+1)^n$ yields
\begin{eqnarray*}
Ly&=&2\sum^\infty_{n=0}(n+2)(n+1)a_{n+2}(x+1)^n+\sum^\infty_{n=0}(n+1)na_{n+1}
(x+1)^n\\&&-\sum^\infty_{n=0}(n+1)a_{n+1}(x+1)^n
+\sum^\infty_{n=0}(2n-3)a_n(x+1)^n+\sum^\infty_{n=1}a_{n-1}(x+1)^n\\
&=&\sum^\infty_{n=0}b_n(x+1)^n,
\end{eqnarray*}
where
$$
b_0=4a_2-a_1-3a_0
$$
and
$$
b_n=2(n+2)(n+1)a_{n+2}+(n^2-1)a_{n+1}+(2n-3)a_n+a_{n-1},\quad n\geq1.
$$
Therefore $y=\sum^\infty_{n=0}a_n(x+1)^n$ is a solution of $Ly=0$
if and only if
\begin{equation} \label{eq:7.3.5}
a_2=\frac{1}{4}(a_1+3a_0)
\end{equation}
and
\begin{equation} \label{eq:7.3.6}
a_{n+2}=-\frac{1}{2(n+2)(n+1)}\left[(n^2-1)a_{n+1}+(2n-3)a_n+a_{n-1}\right],
\quad n\geq1.
\end{equation}
From the initial conditions in \eqref{eq:7.3.4}, $a_0=y(-1)=2$ and
$a_1=y'(-1)=-3$. We leave it to you to  compute $a_2, \dots, a_5$
with  \eqref{eq:7.3.5} and \eqref{eq:7.3.6} and show that
the solution of \eqref{eq:7.3.4} is
$$
y=-2-3(x+1)+\frac{3}{4}(x+1)^2-\frac{5}{12}(x+1)^3+\frac{7}{48}(x+1)^4
-\frac{1}{60}(x+1)^5+\cdots.
$$
We also leave it to you %(Exercise~\ref{exer:7.3.14}) 
to verify numerically
that the Taylor polynomials $T_N(x)=\sum_{n=0}^Na_nx^n$ converge to
the solution of \eqref{eq:7.3.4}  on the interval of
convergence of the power series solution.
\end{explanation}
\end{example}

\begin{example}\label{example:7.3.3}
Find the coefficients $a_0, \dots, a_5$ in the series
solution $y=\sum^\infty_{n=0}
a_nx^n$   of the initial value problem
\begin{equation} \label{eq:7.3.7}
y''+3xy'+(4+2x^2)y=0,\quad y(0)=2,\quad y'(0)=-3.
\end{equation}
\begin{explanation}
Here
$$
Ly=y''+3xy'+(4+2x^2)y.
$$
Since
$$
y=\sum^\infty_{n=0} a_nx^n,\quad y'=\sum^\infty_{n=1} n
a_nx^{n-1},\quad\mbox{and}\quad
y''=\sum^\infty_{n=2}n(n-1)a_nx^{n-2},
$$
\begin{eqnarray*}
Ly&=&\sum^\infty_{n=2}n(n-1)a_nx^{n-2}
+3\sum^\infty_{n=1}na_nx^n+4\sum^\infty_{n=0}a_nx^n+2\sum^\infty_{n=0}
a_nx^{n+2}.
\end{eqnarray*}
Shifting indices so  that the general term in each
series is a constant multiple of $x^n$ yields
$$
Ly=\sum^\infty_{n=0}(n+2)(n+1)a_{n+2}x^n+\sum^\infty_{n=0}(3n+4)a_nx^n
+2\sum^\infty_{n=2}a_{n-2}x^n=\sum_{n=0}^\infty b_nx^n
$$
where
$$
b_0=2a_2+4a_0,\quad b_1=6a_3+7a_1,
$$
and
$$
b_n=(n+2)(n+1)a_{n+2}+(3n+4)a_n+2a_{n-2},\quad n\geq2.
$$
Therefore $y=\sum^\infty_{n=0}a_nx^n$ is a solution of $Ly=0$
if and only if
\begin{equation} \label{eq:7.3.8}
a_2=-2a_0,\quad a_3=-\frac{7}{6}a_1,
\end{equation}
and
\begin{equation} \label{eq:7.3.9}
a_{n+2}=-\frac{1}{(n+2)(n+1)}\left[(3n+4)a_n+2a_{n-2}\right],\quad n\geq2.
\end{equation}
From the initial conditions in \eqref{eq:7.3.7}, $a_0=y(0)=2$ and
$a_1=y'(0)=-3$. We leave it to you to compute $a_2, \dots, a_5$ with
 \eqref{eq:7.3.8} and \eqref{eq:7.3.9} and show that the solution of
\eqref{eq:7.3.7} is
$$
y=2-3x-4x^2+\frac{7}{2}x^3+3x^4-\frac{79}{40}x^5+\cdots.
$$
We also leave it to you 
%(Exercise~\ref{exer:7.3.15}) 
to verify numerically
that the Taylor polynomials $T_N(x)=\sum_{n=0}^Na_nx^n$ converge to
the solution of \eqref{eq:7.3.9} on the interval of
convergence of the power series solution.
\end{explanation}
\end{example}

\section*{Text Source}
Trench, William F., "Elementary Differential Equations" (2013). Faculty Authored and Edited Books \& CDs. 8. (CC-BY-NC-SA)

\href{https://digitalcommons.trinity.edu/mono/8/}{https://digitalcommons.trinity.edu/mono/8/}

\end{document}