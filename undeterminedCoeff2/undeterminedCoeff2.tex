\documentclass{ximera}

%% You can put user macros here
%% However, you cannot make new environments

\listfiles

\graphicspath{{./}{firstExample/}{secondExample/}}

\usepackage{tikz}
\usepackage{tkz-euclide}
\usepackage{tikz-3dplot}
\usepackage{tikz-cd}
\usetikzlibrary{shapes.geometric}
\usetikzlibrary{arrows}
\usetikzlibrary{decorations.pathmorphing,patterns}
\usetkzobj{all}
\pgfplotsset{compat=1.13} % prevents compile error.

\renewcommand{\vec}[1]{\mathbf{#1}}
\newcommand{\RR}{\mathbb{R}}
\newcommand{\dfn}{\textit}
\newcommand{\dotp}{\cdot}
\newcommand{\id}{\text{id}}
\newcommand\norm[1]{\left\lVert#1\right\rVert}
 
\newtheorem{general}{Generalization}
\newtheorem{initprob}{Exploration Problem}

\tikzstyle geometryDiagrams=[ultra thick,color=blue!50!black]

\usepackage{mathtools}




\title{The Method of Undetermined Coefficients II}


\begin{document}

\begin{abstract}

\end{abstract}

\maketitle

\section*{The Method of Undetermined Coefficients II}

In this section we consider the constant coefficient equation
\begin{equation} \label{eq:5.5.1}
ay''+by'+cy=e^{\lambda x}\left(P(x)\cos \omega x+Q(x)\sin \omega x\right)
\end{equation}
where $\lambda$ and $\omega$ are real numbers, $\omega\neq 0$, and $P$
and $Q$ are polynomials. We want to find a particular solution of
\eqref{eq:5.5.1}. As in the previous module, the procedure that we
will use is called \textit{the method of undetermined coefficients}.

\subsection*{Forcing Functions Without Exponential Factors}

We begin with the case where $\lambda=0$ in \eqref{eq:5.5.1};   thus, we
we
want to find a particular solution of
\begin{equation} \label{eq:5.5.2}
ay''+by'+cy=P(x)\cos\omega x+Q(x)\sin\omega x,
\end{equation}
where $P$ and $Q$ are polynomials.

Differentiating  $x^r\cos\omega x$ and $x^r\sin\omega x$ yields
$$
\frac{d}{dx}x^r\cos\omega x=-\omega x^r\sin\omega x+
rx^{r-1}\cos\omega x 
$$
and
$$
\frac{d}{dx}x^r\sin\omega x=\omega x^r\cos\omega x+
rx^{r-1}\sin\omega x.
$$
This implies  that if
$$
y_p=A(x)\cos\omega x+B(x)\sin\omega x
$$
where $A$ and $B$ are polynomials, then
$$
ay_p''+by_p'+cy_p=F(x)\cos\omega x+G(x)\sin\omega x,
$$
where $F$ and $G$ are polynomials with coefficients that can be
expressed in terms of the coefficients of $A$ and $B$. This suggests
that we try to choose $A$ and $B$ so that $F=P$ and $G=Q$,
respectively. Then $y_p$ will be a particular solution of
\eqref{eq:5.5.2}. The next theorem tells us how to choose the proper
form for $y_p$. %For the proof see Exercise~\ref{exer:5.5.37}.

\begin{theorem}\label{thmtype:5.5.1}
Suppose $\omega$ is a positive number and $P$ and $Q$ are
polynomials. Let $k$ be the larger of the degrees of $P$ and $Q$. Then
the equation
$$
ay''+by'+cy=P(x)\cos \omega x+Q(x)\sin \omega x
$$
has a particular solution
\begin{equation} \label{eq:5.5.3}
y_p=A(x)\cos\omega x+B(x)\sin\omega x,
\end{equation}
where
$$
A(x)=A_0+A_1x+\cdots+A_kx^k\quad \mbox{and}\quad
B(x)=B_0+B_1x+\cdots+B_kx^k,
$$
provided that $\cos\omega x$ and $\sin\omega x$ are
not solutions of  the complementary equation. The solutions of
$$
a(y''+\omega^2y)=P(x)\cos \omega x+Q(x)\sin \omega x
$$
(for which $\cos\omega x$ and $\sin\omega x$ are solutions of the
complementary equation) are of the form \eqref{eq:5.5.3},
where
$$
A(x)=A_0x+A_1x^2+\cdots+A_kx^{k+1}\quad \mbox{and}\quad
B(x)=B_0x+B_1x^2+\cdots+B_kx^{k+1}.
$$
\end{theorem}

%For an analog of this theorem that's applicable to \eqref{eq:5.5.1}, see Exercise~\ref{exer:5.5.38}.

\begin{example}\label{example:5.5.1}
Find a particular solution of
\begin{equation} \label{eq:5.5.4}
y''-2y'+y=5\cos2x+10\sin2x.
\end{equation}


\begin{explanation}
In \eqref{eq:5.5.4} the coefficients of $\cos2x$ and $\sin2x$ are both zero
degree polynomials (constants). Therefore Theorem~\ref{thmtype:5.5.1}
implies that \eqref{eq:5.5.4} has a particular solution
$$
y_p=A\cos2x+B\sin2x.
$$
Since
$$
y_p'=-2A\sin2x+2B\cos2x\quad\mbox{and}\quad
y_p''=-4(A\cos2x+B\sin2x),
$$
replacing $y$ by $y_p$ in  \eqref{eq:5.5.4} yields
\begin{eqnarray*}
y_p''-2y_p'+y_p&=&-4(A\cos2x+B\sin2x)-4(-A\sin2x+B\cos2x)
\\ &&+(A\cos2x+B\sin2x)\\ &=&
(-3A-4B)\cos2x+(4A-3B)\sin2x.
\end{eqnarray*}
Equating the coefficients of $\cos2x$ and $\sin2x$ here with the
corresponding coefficients on the right side of \eqref{eq:5.5.4}
shows that $y_p$ is a solution of \eqref{eq:5.5.4}  if
\begin{eqnarray*}
-3A-4B&=&5\\
4A-3B&=&10
\end{eqnarray*}
Solving these equations yields $A=1$, $B=-2$.  Therefore
$$
y_p=\cos2x-2\sin2x
$$
is a particular solution of \eqref{eq:5.5.4}.
\end{explanation}
\end{example}

\begin{example}\label{example:5.5.2}
Find a particular solution of
\begin{equation} \label{eq:5.5.5}
y''+4y=8\cos2x+12\sin2x.
\end{equation}


\begin{explanation}
The procedure used in  Example~\ref{example:5.5.1} doesn't work here;
substituting $y_p=A\cos2x+B\sin2x$ for $y$ in \eqref{eq:5.5.5} yields
$$
y_p''+4y_p=-4(A\cos2x+B\sin2x) +4(A\cos2x+B\sin2x)=0
$$
for any choice of $A$ and $B$, since $\cos2x$ and $\sin2x$ are both
solutions of the complementary equation for \eqref{eq:5.5.5}. We're
dealing with the second case mentioned in Theorem~\ref{thmtype:5.5.1}, and
should therefore try a particular solution of the form
\begin{equation} \label{eq:5.5.6}
y_p=x(A\cos2x+B\sin2x).
\end{equation}
Then
\begin{eqnarray*}
y_p'&=&A\cos2x+B\sin2x+2x(-A\sin2x+B\cos2x)
\\ 
y_p''&=&-4A\sin2x+4B\cos2x-4x(A\cos2x+B\sin2x)\\
&=&-4A\sin2x+4B\cos2x-4y_p \mbox{ (see \eqref{eq:5.5.6})},
\end{eqnarray*}
so
$$
y_p''+4y_p=-4A\sin2x+4B\cos2x.
$$
Therefore $y_p$ is a solution of \eqref{eq:5.5.5} if
$$
-4A\sin2x+4B\cos2x=8\cos2x+12\sin2x,
$$
which holds if $A=-3$ and $B=2$.  Therefore
$$
y_p=-x(3\cos2x-2\sin2x)
$$
is a particular solution of \eqref{eq:5.5.5}.
\end{explanation}
\end{example}

\begin{example}\label{example:5.5.3}
Find a particular solution of
\begin{equation} \label{eq:5.5.7}
y''+3y'+2y=(16+20x)\cos x+10\sin x.
\end{equation}

\begin{explanation}
The coefficients of $\cos x$ and $\sin x$ in \eqref{eq:5.5.7} are
polynomials of degree one and zero, respectively. Therefore
Theorem~\ref{thmtype:5.5.1} tells us to look for a particular solution of
\eqref{eq:5.5.7} of the form
\begin{equation} \label{eq:5.5.8}
y_p=(A_0+A_1x)\cos x+(B_0+B_1x)\sin x.
\end{equation}
Then
\begin{equation} \label{eq:5.5.9}
y_p'=(A_1+B_0+B_1x)\cos x+(B_1-A_0-A_1x)\sin x
\end{equation}
and
\begin{equation} \label{eq:5.5.10}
 y_p''=(2B_1-A_0-A_1x)\cos x-(2A_1+B_0+B_1x)\sin x,
\end{equation}
so
\begin{equation} \label{eq:5.5.11}
\begin{array}{rcl}
y_p''+3y_p'+2y_p&=&\left[A_0+3 A_1+3 B_0+2 B_1+(A_1+3 B_1)x\right]\cos
x\\ &&+ \left[B_0+3 B_1-3 A_0-2 A_1+(B_1-3 A_1)x\right]\sin x.
\end{array}
\end{equation}
Comparing the coefficients of $x\cos x$, $x\sin x$, $\cos x$, and
$\sin x$ here with the corresponding coefficients in \eqref{eq:5.5.7}
shows that $y_p$ is a solution of \eqref{eq:5.5.7} if
$$
\begin{array}{rcr}
A_1+3B_1&=&20\\
-3A_1+B_1&=&0\\
A_0+3B_0+3A_1+2B_1&=&16\\
-3A_0+B_0-2A_1+3B_1&=&10.
\end{array}
$$
Solving the first two equations yields $A_1=2$, $B_1=6$.
Substituting these into the last two equations  yields
\begin{eqnarray*}
A_0+3B_0&=&16-3A_1-2B_1=-2\\
-3A_0+B_0&=&10+2A_1-3B_1=-4
 \end{eqnarray*}
Solving these equations yields $A_0=1$, $B_0=-1$.
Substituting $A_0=1$, $A_1=2$, $B_0=-1$, $B_1=6$ into
\eqref{eq:5.5.8} shows that
$$
y_p=(1+2x)\cos x-(1-6x)\sin x
$$
is a particular solution of  \eqref{eq:5.5.7}.
\end{explanation}
\end{example}


\subsection*{A Useful Observation}

In \eqref{eq:5.5.9}, \eqref{eq:5.5.10}, and \eqref{eq:5.5.11} the polynomials
multiplying $\sin x$ can be obtained by replacing $A_0,A_1,B_0$,
and $B_1$ by $B_0$, $B_1$, $-A_0$, and $-A_1$, respectively, in the
polynomials multiplying $\cos x$. %An analogous result applies in
%general, as follows (Exercise~\ref{exer:5.5.36}).

\begin{theorem}\label{thmtype:5.5.2}
If
$$
y_p=A(x)\cos\omega x+B(x)\sin\omega x,
$$
where $A(x)$ and $B(x)$ are polynomials with coefficients
$A_0$ \dots, $A_k$ and $B_0$, \dots, $B_k,$ then the polynomials
multiplying $\sin\omega x$ in
$$
y_p',\quad y_p'',\quad  ay_p''+by_p'+cy_p
\quad\mbox{and}\quad y_p''+\omega^2 y_p
$$
can be obtained by replacing $A_0$, \dots$,$ $A_k$ by $B_0,$ \dots$,$ $B_k$
and $B_0,$ \dots$,$  $B_k$ by $-A_0,$ \dots$,$ $-A_k$ in the corresponding
polynomials multiplying $\cos\omega x$.
\end{theorem}

We won't use this theorem in our examples, but we recommend that
you use it to check your manipulations when you work on the  exercises.

\begin{example}\label{example:5.5.4}
Find a particular solution of
\begin{equation} \label{eq:5.5.12}
y''+y=(8-4x)\cos x-(8+8x)\sin x.
\end{equation}

\begin{explanation}
According to  Theorem~\ref{thmtype:5.5.1}, we should look for a
particular solution of the form
\begin{equation} \label{eq:5.5.13}
y_p=(A_0x+A_1x^2)\cos x+(B_0x+B_1x^2)\sin x,
\end{equation}
since $\cos x$ and $\sin x$ are solutions of the complementary
equation. However, let's  try
\begin{equation} \label{eq:5.5.14}
y_p=(A_0+A_1x)\cos x+(B_0+B_1x)\sin x
\end{equation}
first, so you can see why it doesn't work. From \eqref{eq:5.5.10},
$$
 y_p''=(2B_1-A_0-A_1x)\cos x-(2A_1+B_0+B_1x)\sin x,
$$
which together with \eqref{eq:5.5.14} implies that
$$
y_p''+y_p=2B_1\cos x-2A_1\sin x.
$$
Since the right side of this equation does not contain $x\cos x$
or $x\sin x$,  \eqref{eq:5.5.14} can't satisfy
\eqref{eq:5.5.12} no matter how we choose $A_0$, $A_1$, $B_0$, and $B_1$.

Now let $y_p$ be as in \eqref{eq:5.5.13}. Then
\begin{eqnarray*}
y_p'&=&\left[A_0+(2A_1+B_0)x+B_1x^2\right]\cos x\\ &&
+\left[B_0+(2B_1-A_0)x-A_1x^2\right]\sin x\\  
y_p''&=&
\left[2A_1+2B_0-(A_0-4B_1)x-A_1x^2\right]\cos x\\ &&+
\left[2B_1-2A_0-(B_0+4A_1)x-B_1x^2\right]\sin x,
\end{eqnarray*}
so
$$
y_p''+y_p=(2A_1+2B_0+4B_1x)\cos x+(2B_1-2A_0-4A_1x)\sin x.
$$
Comparing the  coefficients of $\cos x$ and
$\sin x$ here  with the corresponding coefficients  in \eqref{eq:5.5.12}
shows that
$y_p$ is a solution of \eqref{eq:5.5.12} if
$$
\begin{array}{rcr}
4B_1&=&-4\\
-4A_1&=&-8\\
2B_0+2A_1&=&8\\
-2A_0+2B_1&=&-8
\end{array}
$$
The solution of this system is  $A_1=2$, $B_1=-1$, $A_0=3$, $B_0=2$.
  Therefore
$$
y_p=x\left[(3+2x)\cos x+(2-x)\sin x\right]
$$
is a particular solution of \eqref{eq:5.5.12}.
\end{explanation}
\end{example}



\subsection*{Forcing Functions with Exponential Factors}

To find a particular solution of
\begin{equation} \label{eq:5.5.15}
ay''+by'+cy=e^{\lambda x}\left(P(x)\cos \omega x+Q(x)\sin \omega x\right)
\end{equation}
when $\lambda\neq 0$, we recall from the previous module that
substituting $y=ue^{\lambda x}$ into \eqref{eq:5.5.15} will produce a
constant coefficient equation for $u$  with  the forcing
function  $P(x)\cos \omega x+Q(x)\sin \omega x$. We can find a
particular solution $u_p$ of this equation by the procedure that we
used in Examples~\ref{example:5.5.1}--\ref{example:5.5.4}. Then
$y_p=u_pe^{\lambda x}$ is a particular solution of \eqref{eq:5.5.15}.

\begin{example}\label{example:5.5.5}
Find a particular solution of
\begin{equation} \label{eq:5.5.16}
y''-3y'+2y=e^{-2x}\left[2\cos 3x-(34-150x)\sin 3x\right].
\end{equation}


\begin{explanation}
Let $y=ue^{-2x}$. Then
\begin{eqnarray*}
y''-3y'+2y&=&e^{-2x}\left[(u''-4u'+4u)-3(u'-2u)+2u\right]\\
&=&e^{-2x}(u''-7u'+12u)\\ &=&
e^{-2x}\left[2\cos 3x-(34-150x)\sin 3x\right]
\end{eqnarray*}
if
 \begin{equation} \label{eq:5.5.17}
u''-7u'+12u=2\cos 3x-(34-150x)\sin 3x.
\end{equation}
Since $\cos3x$ and $\sin3x$ aren't solutions of
the complementary equation
$$
u''-7u'+12u=0,
$$
 Theorem~\ref{thmtype:5.5.1} tells us to look for a particular solution
of \eqref{eq:5.5.17} of the form
\begin{equation} \label{eq:5.5.18}
u_p=(A_0+A_1x)\cos 3x +(B_0+B_1x)\sin 3x.
\end{equation}
Then
\begin{eqnarray*}
u_p'&=&(A_1+3B_0+3B_1x)\cos 3x+(B_1-3A_0-3A_1x)\sin 3x\\
u_p''&=&(-9A_0+6B_1-9A_1x)\cos 3x-(9B_0+6A_1+9B_1x)\sin 3x
\end{eqnarray*}
so
\begin{eqnarray*}
u_p''-7u_p'+12u_p&=&\left[3A_0-21B_0-7A_1+6B_1+(3A_1-21B_1)x\right]\cos
3x\\ &&+\left[21A_0+3B_0-6A_1-7B_1+(21A_1+3B_1)x\right]\sin 3x.
\end{eqnarray*}
Comparing the coefficients of $x\cos 3x$, $x\sin 3x$, $\cos 3x$, and
$\sin 3x$ here with the corresponding coefficients on the right side
of \eqref{eq:5.5.17} shows that $u_p$ is a solution of \eqref{eq:5.5.17} if
\begin{equation} \label{eq:5.5.19}
\begin{array}{rcr}
3A_1-21B_1&=&0\\
21A_1+3B_1&=&150\\
3A_0-21B_0-7A_1+6B_1&=&2\\
21A_0+3B_0-6A_1-7B_1&=&-34
\end{array}
\end{equation}
Solving the first two equations yields $A_1=7$, $B_1=1$.
Substituting these values into the last two equations of \eqref{eq:5.5.19}
yields
\begin{eqnarray*}
3A_0-21B_0&=&2+7A_1-6B_1=45\\
21A_0+3B_0&=&-34+6A_1+7B_1=15.
\end{eqnarray*}
Solving this system  yields  $A_0=1$, $B_0=-2$.
Substituting $A_0=1$, $A_1=7$, $B_0=-2$, and $B_1=1$ into
\eqref{eq:5.5.18} shows that
$$
u_p=(1+7x)\cos 3x-(2-x)\sin 3x
$$
is a particular solution of  \eqref{eq:5.5.17}. Therefore
$$
y_p=e^{-2x}\left[(1+7x)\cos 3x-(2-x)\sin 3x\right]
$$
is a particular solution of   \eqref{eq:5.5.16}.
\end{explanation}
\end{example}

\begin{example}\label{example:5.5.6} 
Find a particular solution of
\begin{equation} \label{eq:5.5.20}
y''+2y'+5y=e^{-x}\left[(6-16x)\cos2x-(8+8x)\sin2x\right].
\end{equation}


\begin{explanation}
Let $y=ue^{-x}$. Then
\begin{eqnarray*}
y''+2y'+5y&=&e^{-x}\left[(u''-2u'+u)+2(u'-u)+5u\right]\\
&=&e^{-x}(u''+4u)\\ &=&
e^{-x}\left[(6-16x)\cos2x-(8+8x)\sin2x\right]
\end{eqnarray*}
if
\begin{equation} \label{eq:5.5.21}
u''+4u=(6-16x)\cos2x-(8+8x)\sin2x.
\end{equation}
Since $\cos2x$  and $\sin2x$ are solutions of
the complementary equation
$$
u''+4u=0,
$$
 Theorem~\ref{thmtype:5.5.1} tells us to look for a particular solution
of \eqref{eq:5.5.21} of the form
$$
u_p=(A_0x+A_1x^2)\cos2x+(B_0x+B_1x^2)\sin2x.
$$
Then
\begin{eqnarray*}
u_p'&=&\left[A_0+(2A_1+2B_0)x+2B_1x^2\right]\cos2x \\ &&
+\left[B_0+(2B_1-2A_0)x-2A_1x^2\right]\sin2x\\ 
u_p''&=&\left[2A_1+4B_0-(4A_0-8B_1)x-4A_1x^2\right]\cos2x\\ &&
+\left[2B_1-4A_0-(4B_0+8A_1)x-4B_1x^2\right]\sin2x,
\end{eqnarray*}
so
$$
u_p''+4u_p=(2A_1+4B_0+8B_1x)\cos2x+(2B_1-4A_0-8A_1x)\sin2x.
$$
Equating the  coefficients of $x\cos2x$,
$x\sin2x$, $\cos2x$, and $\sin2x$  here with the
corresponding
coefficients on the right side of \eqref{eq:5.5.21} shows that $u_p$
is a solution of \eqref{eq:5.5.21} if
\begin{equation} \label{eq:5.5.22}
\begin{array}{rcr}
8B_1&=&-16\\
-8A_1&=&-8\\
4B_0+2A_1&=&6\\
-4A_0+2B_1&=&-8
\end{array}
\end{equation}
The solution of this system is  $A_1=1$, $B_1=-2$, $B_0=1$, $A_0=1$.
Therefore
$$
u_p=x[(1+x)\cos2x+(1-2x)\sin2x]
$$
is a particular solution of  \eqref{eq:5.5.21}, and
$$
y_p=xe^{-x}\left[(1+x)\cos2x+(1-2x)\sin2x\right]
$$
is a particular solution of  \eqref{eq:5.5.20}.
\end{explanation}
\end{example}

You can also find a particular solution of \eqref{eq:5.5.20}
by substituting
$$
y_p=xe^{-x}\left[(A_0+A_1x)\cos 2x +(B_0+B_1x)\sin 2x\right]
$$
for $y$  in \eqref{eq:5.5.20}
and equating the  coefficients of $xe^{-x}\cos2x$,
$xe^{-x}\sin2x$, $e^{-x}\cos2x$, and $e^{-x}\sin2x$  in the
resulting expression for
$$
y_p''+2y_p'+5y_p
$$
with the corresponding coefficients on the right side of
\eqref{eq:5.5.20}. %(See Exercise~\ref{exer:5.5.38}). 
This leads to the
same system \eqref{eq:5.5.22} of equations for $A_0$, $A_1$, $B_0$, and
$B_1$ that we obtained in Example~\ref{example:5.5.6}. However, if you try
this approach you'll see that deriving \eqref{eq:5.5.22} this way is
much more tedious than the way we did it in Example~\ref{example:5.5.6}.

\section*{Text Source}
Trench, William F., "Elementary Differential Equations" (2013). Faculty Authored and Edited Books \& CDs. 8. (CC-BY-NC-SA)

\href{https://digitalcommons.trinity.edu/mono/8/}{https://digitalcommons.trinity.edu/mono/8/}

\end{document}


